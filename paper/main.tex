%!TEX program = pdflatex

%% ORGANIZAÇÃO DO PAPER
% Seções 2–3: o que a teoria diz que deveria acontecer
% Seção 4: o que parece ter acontecido nos dados
% Seção 5: como vamos testar isso formalmente
% Seção 6: o que os testes mostram

\documentclass[12pt,a4paper,alf]{abntex2}
\AtBeginDocument{\selectlanguage{brazilian}}
\emergencystretch=2em
\raggedbottom

% Teste vscode para github para overleaf
% remove o "capítulo." das seções
\renewcommand{\thesection}{\arabic{section}}
\renewcommand{\thesubsection}{\thesection.\arabic{subsection}}
\renewcommand{\thesubsubsection}{\thesubsection.\arabic{subsubsection}}

% Altera o nome "Figura" para "Gráfico"
\renewcommand{\figurename}{Gráfico}

% ================== CODIFICAÇÃO ==================
\usepackage[T1]{fontenc}
\usepackage[utf8]{inputenc}

% ================== PACOTES MATEMÁTICOS E GRÁFICOS ==================
\usepackage{amsmath,amssymb,amsfonts}
\usepackage{graphicx}
\usepackage{booktabs}

% ================== FORMATAÇÃO ==================
\usepackage{geometry}
\usepackage{setspace}
\geometry{margin=2.5cm}
\OnehalfSpacing

% ================== CITAÇÕES E REFERÊNCIAS ==================
\usepackage[alf]{abntex2cite}

% ================== TEXTO E LINKS ==================
\usepackage{csquotes}
\usepackage{hyperref}

% ================== INFORMAÇÕES DO ARTIGO ==================
\title{Determinância e Política Monetária no Brasil:\\
Uma Avaliação Empírica do Princípio de Taylor (1999--2024)}
\author{
Fernando Souza de Vieira\\
Rafaela Dezidério Rocha
}
\date{}

% ***********************************************************
\begin{document}

\maketitle
\thispagestyle{empty}

\begin{abstract}
Este artigo investiga em que medida a política monetária brasileira, sob o regime de metas de inflação vigente desde 1999, tem seguido regras compatíveis com o Princípio de Taylor e com a determinância do equilíbrio em modelos Novo-Keynesianos. O estudo estima diferentes especificações da Regra de Taylor — \textit{backward-looking}, \textit{forward-looking} no estilo Clarida--Galí--Gertler e versões híbridas com suavização da taxa de juros — utilizando dados da taxa Selic, inflação (IPCA), hiato do produto e expectativas de inflação (Focus), no período 1999--2024. A metodologia combina MQO com erros robustos, GMM com instrumentos defasados, testes de quebras estruturais (Bai--Perron) e avaliação formal da condição de Bullard--Mitra, que relaciona os parâmetros estimados à determinância do modelo Novo-Keynesiano. Os resultados permitem identificar se, e em quais subperíodos, a reação da política monetária à inflação foi suficientemente forte ($\phi_\pi > 1$) para garantir unicidade do equilíbrio e ancoragem das expectativas. A contribuição central do artigo consiste em estabelecer um diálogo sistemático entre o arcabouço Novo-Keynesiano e a experiência brasileira recente, destacando como diferentes regimes de política monetária se relacionam com a estabilidade nominal e a credibilidade do Banco Central.
\end{abstract}

\pagestyle{plain}

% ***********************************************************

% ===========================================================
\section{Introdução}
% ===========================================================

Desde 1999, o regime de metas para a inflação consolidou um novo arcabouço de política monetária no Brasil, ancorado na transparência da atuação do Banco Central e na necessidade de compatibilizar estabilidade nominal e crescimento econômico. Ao longo desse período, a condução da taxa Selic passou a ser orientada por uma lógica sistemática de resposta a desvios da inflação em relação à meta e, em menor medida, a flutuações no hiato do produto \cite{carrara2012regime}. Embora tal lógica nem sempre seja explicitada em regras formais, a literatura reconhece que o comportamento histórico do Banco Central brasileiro aproxima-se de uma função de reação do tipo Taylor \cite{de2013modified}.

Conforme será argumentado teoricamente na Seção 3.3, a Regra de Taylor especifica que a taxa básica de juros reage sistematicamente ao desvio da inflação em relação à meta e ao hiato do produto, elevando-se quando inflação e atividade estão acima de seus níveis desejados e reduzindo-se no caso oposto. O Princípio de Taylor, por sua vez, formaliza uma exigência mínima de estabilização: para que a política monetária evite equilíbrios múltiplos e trajetórias instáveis, o coeficiente de resposta da taxa de juros à inflação deve exceder a unidade. Essa condição, aparentemente simples, é decisiva em modelos Novo-Keynesianos, nos quais a dinâmica macroeconômica depende criticamente de expectativas e de rigidez nominal \cite{taylor1993}.

% -----------------------------------------------------------
\subsection{Contexto e Motivação}

Segundo \citeonline[p.~120]{balbinoPoliticaMonetariaBrasileira2011}, a credibilidade da autoridade monetária, elemento central do regime de metas, está diretamente associada à capacidade de reagir de forma consistente a choques inflacionários. Nesse sentido, a violação do Princípio de Taylor pode gerar instabilidade nominal, indeterminação do equilíbrio e sensibilidade excessiva a choques exógenos e revisões nas expectativas. Essa necessidade de reação consistente conecta-se diretamente à estabilidade dinâmica discutida na literatura Novo-Keynesiana: quando a autoridade monetária não reage de forma suficientemente forte à inflação, abre-se espaço para desancoragem de expectativas e equilíbrios autorrealizáveis \cite{woodford2003interest,bullard2002learning}.

Um elemento adicional que condiciona a eficácia da política monetária é o risco de dominância fiscal. Quando a dinâmica da dívida pública restringe o espaço de atuação do Banco Central, aumentos da taxa Selic podem comprometer a sustentabilidade fiscal, reduzindo a autonomia da política monetária e dificultando a implementação de respostas agressivas à inflação. Em tais cenários, a condição $\phi_{\pi} > 1$ pode tornar-se inviável na prática, ainda que desejável do ponto de vista teórico \cite{kumhof2010simple}.

% -----------------------------------------------------------
\subsection{Problema de Pesquisa}

O problema de pesquisa que orienta este artigo pode ser sintetizado da seguinte forma:

\begin{itemize}
\item[] \textbf{Até que ponto a política monetária brasileira, no período 1999--2024, reagiu à inflação de forma suficientemente forte para satisfazer o Princípio de Taylor e garantir determinância no sentido do modelo Novo-Keynesiano?} Ademais, como esse padrão de reação variou entre os diferentes presidentes do Banco Central?
\end{itemize}

% -----------------------------------------------------------
\subsection{Objeto do Estudo}

\textbf{O objeto deste estudo consiste na estimativa da função de reação da taxa Selic à inflação e ao hiato do produto no regime de metas de inflação brasileiro}, a partir de diferentes formulações empíricas da Regra de Taylor: retrospectivas, prospectivas e híbridas com inércia. Essa função de reação é empregada para verificar a validade empírica do Princípio de Taylor e, de forma complementar, avaliar se os parâmetros obtidos asseguram determinância do equilíbrio no modelo Novo-Keynesiano.

% -----------------------------------------------------------
\subsection{Objetivos}

\subsubsection*{Objetivo Geral}

Temos como \textbf{objetivo geral do artigo avaliar empiricamente se a política monetária brasileira tem satisfeito o Princípio de Taylor e se os parâmetros estimados são compatíveis com a determinância (local) do equilíbrio no modelo Novo-Keynesiano.}

\subsubsection*{Objetivos Específicos}

\begin{itemize}
\item Estimar diferentes especificações da Regra de Taylor.
\item Verificar se os coeficientes estimados satisfazem $\hat{\phi}_\pi > 1$.
\item Avaliar a condição de Bullard--Mitra para os parâmetros estimados.
\item Explorar implicações dinâmicas por meio de simulações em modelo NK.
\end{itemize}

% -----------------------------------------------------------
\subsection{Estrutura do Artigo}

\textbf{Além desta introdução, o artigo organiza-se em outras três partes. A primeira parte reúne as Seções 2, 3 e 4}: a Seção 2 revisa a literatura sobre regras monetárias e determinância, com ênfase em evidência internacional e para o Brasil, apresentando como as diferentes formulações da Regra de Taylor têm sido utilizadas e estimadas empiricamente; a Seção 3, por sua vez, desenvolve o arcabouço Novo-Keynesiano que fundamenta a análise, mostrando como essas mesmas regras operam dentro do modelo, em particular sua interação com a NKPC, a equação IS e a condição de Bullard--Mitra; e a Seção 4 contextualiza o regime de metas brasileiro, descrevendo presidentes do Banco Central, regimes de política monetária e fatos estilizados que motivam a estratificação da análise. \textbf{A segunda parte compreende as Seções 5, 6 e 7}: a Seção 5 detalha dados e metodologia, a Seção 6 apresenta os resultados empíricos e a Seção 7 discute suas implicações para credibilidade, dominância fiscal e estabilidade monetária. \textbf{A terceira parte contém a Seção 8}, com as conclusões e considerações finais.

% ===========================================================
\section{Regras de Política e Estabilidade: Evidências Teóricas e Empíricas}
% ===========================================================

A presente seção organiza a discussão de política monetária a partir das diferentes formulações da Regra de Taylor, que ajudam a entender como bancos centrais ajustam juros em função da inflação e da atividade. Em seguida, são apresentadas as contribuições sobre determinância e estabilidade, que delimitam quando essas respostas são suficientes para garantir um equilíbrio único no modelo Novo-Keynesiano. Evidências internacionais e brasileiras complementam esse panorama, mostrando como esses mecanismos se manifestaram em distintos regimes e períodos. Essa sequência, da formulação das regras às suas implicações teóricas e empíricas, busca oferecer o enquadramento necessário para interpretar os resultados estimados posteriormente, com base no arcabouço teórico, que os formaliza em maior profundidade na seção 3.

% -----------------------------------------------------------
\subsection{Regras de Política Monetária}

As contribuições de \citeonline{taylor1993}, que, juntamente com a Curva de Phillips Novo-Keynesiana e a equação IS dinâmica, compõem o núcleo da análise moderna de política monetária ao fornecerem o mecanismo de fechamento do modelo, podem ser compreendidas no contexto da transição metodológica produzida pela crítica de Lucas e pela consolidação dos modelos com expectativas racionais. Taylor enfatiza que a avaliação \textit{ex ante} de regras de política só se tornou factível quando a literatura reconheceu as limitações dos modelos tradicionais, bem como os avanços teóricos que demonstravam a superioridade de regras sistemáticas sobre a discricionariedade. É nesse contexto que ele identifica os principais fatores que motivaram a formulação de regras de política operacionalizáveis, afirmando que “\textit{the Lucas critique showing that traditional econometric policy evaluation was flawed, the recognition that rational expectations does not imply monetary policy ineffectiveness, and the finding that credibility has empirically significant benefits}” foram determinantes para esse avanço \cite[p.~195--196]{taylor1993}.

Essas motivações e alicerces teóricos dão sentido à formulação da Regra de Taylor: uma especificação simples e operacional que relaciona a taxa básica de juros a desvios da inflação e do hiato do produto, oferecendo uma resposta sistemática e previsível em contraste com o “\textit{feeling}” discricionário. Em sua análise histórica subsequente, \citeonline{taylor1999} mostrou que períodos em que a política monetária reagiu de forma insuficiente aos desvios da inflação, isto é, com coeficientes $\phi_{\pi}$ persistentemente inferiores à unidade, coincidiram com maior instabilidade macroeconômica, ao passo que períodos caracterizados por respostas mais agressivas estiveram associados a maior estabilidade. Assim, o conjunto de seus trabalhos estabelece não apenas uma regra prática, mas um arcabouço conceitual que vincula credibilidade, expectativas e determinância do equilíbrio, desempenhando papel central no fechamento dos modelos Novo-Keynesianos.

Aprofundando essa análise sob a ótica das expectativas racionais, \citeonline{clarida2000monetary} expandiram o arcabouço de Taylor ao estimar uma função de reação \textit{forward-looking} (quando a autoridade monetária reage às expectativas de inflação futura e não aos dados passados) para a economia dos EUA, dividindo a amostra nas eras pré-Volcker (1960--1979) e Volcker-Greenspan (1979--1996). Os autores demonstraram formalmente que a instabilidade macroeconômica do primeiro período decorreu de uma violação sistemática do Princípio de Taylor ($\phi_{\pi} < 1$), o que permitiu a emergência de equilíbrios indeterminados sujeitos a \textit{sunspots}\space (profecias autorrealizáveis). A contribuição crucial desse trabalho reside na incorporação explícita das expectativas de inflação futura na função de reação do Banco Central, estabelecendo que a estabilidade não depende apenas da resposta aos dados observados, mas da capacidade da autoridade monetária de ancorar as expectativas dos agentes através de uma postura agressiva ($\phi_{\pi} > 1$) diante de choques inflacionários previstos. Essa abordagem fundamenta diretamente a estratégia empírica do presente estudo, que busca testar se a condução da política monetária no Brasil, ao reagir às expectativas de inflação (Focus), cumpriu as condições de determinância exigidas para evitar a propagação de instabilidade não fundamental.

Uma síntese natural das contribuições anteriores encontra-se em \citeonline{woodford2003interest}, cuja formulação microfundamentada do modelo Novo-Keynesiano apresenta o arcabouço teórico dominante para a análise de política monetária. Woodford demonstrou que a eficácia das regras de política depende crucialmente do modo como elas ancoram expectativas, não somente de inflação futura, mas também do próprio estado da economia ao longo do tempo. Em seu tratamento normativo, a autoridade monetária minimiza a perda social derivada de fundamentos microeconômicos, o que leva a regras ótimas caracterizadas por forte resposta às expectativas de inflação e por um compromisso intertemporal capaz de eliminar equilíbrios indeterminados. Assim, embora a Regra de Taylor represente uma aproximação reduzida dessas regras ótimas, Woodford mostra que, de fato, políticas que reagem de maneira suficientemente agressiva à inflação (p. ex., $\phi_{\pi}$ >1) são as que garantem a unicidade do equilíbrio e a estabilidade macroeconômica. Desse modo, a contribuição de Woodford consolida o elo conceitual entre microfundamentação, expectativas e determinância, oferecendo a justificativa teórica definitiva para o uso de regras de Taylor na avaliação empírica da política monetária.

% -----------------------------------------------------------
\subsection{Determinância e Estabilidade}

A análise da eficácia das regras de política monetária em modelos dinâmicos exige, além da especificação da função de reação da autoridade monetária, a verificação das condições sob as quais o sistema econômico converge para uma trajetória estável e única. Nesse contexto, a contribuição de \citeonline{blanchard1980solution} estabeleceu o arcabouço matemático fundamental para a solução de modelos lineares com expectativas racionais, permitindo a verificação formal da estabilidade de regras como a de Taylor. 

Os autores demonstraram que a existência e a unicidade do equilíbrio dependem da relação entre o número de autovalores explosivos da matriz de transição do sistema e o número de variáveis não predeterminadas (\textit{forward-looking}). Especificamente, para que haja determinância (equilíbrio único e estável), o número de raízes instáveis, isto é, os componentes dinâmicos que levariam o sistema a uma trajetória explosiva, deve ser exatamente igual ao número de variáveis de salto, ou seja, aquelas que podem mudar de valor instantaneamente e de forma \enquote{livre} em resposta a novas informações ou choques, porque dependem inteiramente das expectativas futuras dos agentes, como no caso da inflação e o hiato do produto. Caso essa condição não seja satisfeita, o modelo pode apresentar indeterminação, permitindo múltiplos equilíbrios e flutuações guiadas por \textit{sunspots}, ou inexistência de solução estável (trajetórias explosivas).

Avançando sobre essa base, \citeonline{bullard2002learning} investigaram a estabilidade do equilíbrio Novo-Keynesiano sob a hipótese de que os agentes não possuem conhecimento perfeito da estrutura da economia, mas formam expectativas através de processos de aprendizagem adaptativa (\textit{adaptive learning}). Ao analisarem diferentes especificações da regra de juros, os autores derivaram condições para determinância e para estabilidade sob aprendizagem (\textit{E-stability}). No caso do modelo NK canônico com regra de juros linear reagindo à inflação e ao hiato do produto, uma condição amplamente utilizada para garantir estabilidade do equilíbrio (e, sob hipóteses usuais, determinância local) pode ser expressa pela desigualdade abaixo:

\begin{equation}
\kappa(\phi_\pi - 1) + (1-\beta)\phi_y > 0
\end{equation}

Essa expressão formaliza analiticamente o Princípio de Taylor no contexto de modelos microfundamentados: para garantir a determinância e a convergência da aprendizagem dos agentes para o equilíbrio de expectativas racionais, a autoridade monetária deve reagir à inflação de forma agressiva ($\phi_\pi > 1$), ajustada pela resposta da autoridade monetária ao hiato do produto ($\phi_y$) e pelos parâmetros estruturais da economia, como o fator de desconto ($\beta$) e a inclinação da Curva de Phillips ($\kappa$). 

Posto isso, o trabalho de Bullard e Mitra é central para este estudo, pois conecta a estimação empírica dos coeficientes $\phi_\pi$ e $\phi_y$ à capacidade efetiva do Banco Central de ancorar expectativas e evitar a propagação de instabilidade não fundamental. No contexto empírico deste artigo, essa condição será utilizada para avaliar se os coeficientes 
$\hat{\phi}_{\pi}$ e $\hat{\phi}_{y}$, estimados a partir de diferentes especificações da Regra de Taylor, 
são compatíveis com a determinância do equilíbrio NK.

% -----------------------------------------------------------
\subsection{Evidências Internacionais}

A literatura empírica internacional, ao aplicar o arcabouço de regras de Taylor em economias avançadas, consolidou o entendimento de que a estabilidade macroeconômica depende crucialmente do cumprimento do Princípio de Taylor. Trabalhos como o de \citeonline{smets2007shocks}, utilizando um modelo DSGE estimado por métodos bayesianos para a economia norte-americana no período pós--1984 (``Grande Moderação''), encontraram coeficientes de reação à inflação significativamente superiores à unidade (em torno de 2,0 no longo prazo), corroborando a hipótese de que uma política monetária ativa foi determinante para a ancoragem das expectativas e a redução da volatilidade do produto. Essa evidência contrasta com os resultados para o período da ``Grande Inflação'' (pré--1979), onde estimativas de \citeonline{lubik2004testing} indicam que a política monetária situava-se na região de indeterminação, permitindo flutuações guiadas por profecias autorrealizáveis (\textit{sunspots}).

Complementarmente, a aplicação de regras de Taylor em outros contextos regionais reforça, com nuances importantes, o papel central da resposta sistemática à inflação, ao mesmo tempo em que amplia a agenda para incluir também riscos de instabilidade financeira. \citeonline{moura2010can} mostram que, em economias latino-americanas como Chile, México e Brasil, estimativas de regras de Taylor frequentemente apontam para $\phi_\pi > 1$ em regimes de metas de inflação mais consolidados, mas revelam forte heterogeneidade temporal e institucional, sugerindo que a aderência ao Princípio de Taylor é mais frágil em ambientes de maior vulnerabilidade fiscal e cambial. Em paralelo, \citeonline{kahn2010taylor} e \citeonline{kafer2014taylor} destacam que desvios persistentes em relação à regra de Taylor podem estar associados à formação de desequilíbrios financeiros, defendendo a incorporação de variáveis financeiras nas funções de reação, em particular no contexto do Federal Reserve e da área do euro. Esses resultados situam a problemática deste artigo em um quadro mais amplo: ainda que o foco aqui recaia sobre determinância e ancoragem de expectativas em modelos Novo-Keynesianos, a força da reação da política monetária à inflação, medida por $\phi_\pi$, também dialoga com o debate contemporâneo sobre a compatibilização entre estabilidade de preços, estabilidade financeira e credibilidade em economias avançadas e emergentes.

Além da magnitude da resposta à inflação, a literatura também enfatiza a importância da informação utilizada pelo Banco Central. \citeonline{orphanides2001monetary} demonstrou que avaliações de regras de política baseadas em dados revisados (\textit{ex post}) podem levar a conclusões enganosas sobre a conduta da autoridade monetária, uma vez que as decisões são tomadas com base em dados em tempo real. Seus resultados indicam que especificações \textit{forward-looking}, que incorporam as previsões disponíveis no momento da decisão, descrevem com maior precisão o comportamento histórico do \textit{Federal Reserve}. Essa constatação reforça a relevância de utilizar medidas de expectativas de mercado, como as do Boletim Focus no caso brasileiro, para capturar adequadamente o conjunto de informações que condiciona a reação da política monetária e a determinação do equilíbrio.

% -----------------------------------------------------------
\subsection{Evidências para o Brasil}

Ecoando a consolidação internacional do arcabouço de regras de Taylor, a literatura empírica brasileira também produziu um vasto corpo de evidências sobre a condução da política monetária sob o regime de metas, buscando verificar se o Banco Central do Brasil aderiu aos princípios de estabilização consagrados nos modelos Novo-Keynesianos. Desde a implementação do regime, discutida normativamente por \citeonline{giambiagi2002metas} como um mecanismo de disciplina intertemporal para a estabilidade de preços, artigos como o de \citeonline{minella2003inflation} documentaram uma reação vigorosa da taxa Selic aos desvios das expectativas de inflação ($\phi_\pi > 1$). Essa evidência inicial sugeriu que, a exemplo das economias avançadas durante a ``Grande Moderação'', a autoridade monetária brasileira internalizou rapidamente a necessidade de uma postura ativa para construir credibilidade e ancorar as expectativas nominais, achado corroborado em estudos mais recentes, como \citeonline{vanzelotti2023estimando}, para amostras ampliadas.

Entretanto, diferentemente do caso norte-americano ou europeu, a aplicação da Regra de Taylor ao contexto brasileiro exige a consideração de idiossincrasias estruturais de uma economia emergente. \citeonline{barbosa2016taxa}, por exemplo, argumentam que a postura da política monetária deve ser avaliada à luz de uma taxa de juros natural variável e historicamente elevada, influenciada pelo risco-país e pelas condições externas. Adicionalmente, \citeonline{aragon2010nonlinearities} demonstraram que a função de reação do Banco Central pode apresentar não linearidades, caracterizada por preferências assimétricas que toleram menos desvios positivos da inflação do que quedas no produto. Essas nuances são reforçadas por \citeonline{areosa2007inflation}, que destacam como a forte inércia inflacionária e o repasse cambial no Brasil impõem restrições mais severas à calibração da regra de juros do que as observadas em modelos de economia fechada.

Apesar desses avanços robustos na caracterização das especificidades locais, observa-se uma lacuna importante quando se compara a literatura nacional à fronteira da pesquisa internacional descrita na seção anterior: a conexão explícita entre as estimativas empíricas e as condições teóricas de estabilidade dinâmica. Enquanto autores como \citeonline{lubik2004testing} testam formalmente a determinância do equilíbrio para os EUA, poucos estudos no Brasil transpõem a estimação dos coeficientes $\phi_\pi$ e $\phi_y$ para uma avaliação rigorosa da condição de \citeonline{bullard2002learning}. Em particular, carece-se de uma análise sistemática que segmente os mandatos presidenciais do Banco Central para verificar se, em cada gestão específica, a combinação de parâmetros satisfez as condições de unicidade do equilíbrio, evitando a vulnerabilidade da economia a flutuações guiadas por profecias autorrealizáveis. É justamente nessa interseção entre a evidência econométrica dos regimes brasileiros e a teoria de estabilidade sob aprendizagem que este artigo se insere.

% ***********************************************************

% ===========================================================
\section{Arcabouço Teórico}
% ===========================================================

Esta seção apresenta o modelo Novo-Keynesiano padrão que fundamenta a análise empírica, estruturando as relações dinâmicas entre inflação, produto e taxa de juros. Inicialmente, descreve-se a oferta agregada da economia por meio da Curva de Phillips Novo-Keynesiana (NKPC), derivada de fundamentos microeconômicos de rigidez de preços, seguida pela demanda agregada representada pela equação IS dinâmica. Com o sistema econômico definido, discutem-se as diferentes especificações da Regra de Taylor que fecham o modelo, detalhando como a reação do Banco Central a choques condiciona a estabilidade macroeconômica. Por fim, formalizam-se as condições de determinância de Bullard e Mitra (2002), estabelecendo o critério teórico exato para avaliar se os parâmetros estimados na seção empírica são compatíveis com a unicidade do equilíbrio ou se permitem instabilidade.

% -----------------------------------------------------------
\subsection{Curva de Phillips Novo-Keynesiana (NKPC)}

No modelo Novo-Keynesiano, a dinâmica da oferta agregada é descrita pela Curva de Phillips Novo-Keynesiana (NKPC), que relaciona a inflação corrente às expectativas de inflação futura e ao hiato do produto. Essa relação é derivada a partir de fundamentos microeconômicos, resultando do problema de precificação intertemporal das firmas em um ambiente com concorrência monopolística e rigidez nominal de preços.

Seguindo \citeonline{gali2015monetary}, assume-se rigidez de preços à la \citeonline{calvo1983staggered}: em cada período, apenas uma fração $(1-\theta)$ das firmas pode reajustar seus preços de forma ótima, enquanto a fração $\theta$ mantém os preços previamente fixados. O parâmetro $\theta \in [0,1)$ mede o grau de rigidez nominal da economia, determinando a duração esperada dos contratos de preços.

As firmas que reajustam seus preços escolhem um preço ótimo maximizando o valor presente esperado dos lucros, levando em conta que esse preço poderá vigorar por vários períodos. A agregação dessas decisões e a log-linearização em torno de um estado estacionário de inflação nula conduzem à seguinte expressão para a dinâmica inflacionária:

\begin{equation}
\pi_t = \beta E_t \pi_{t+1} + \kappa \tilde{y}_t,
\label{eq:nkpc}
\end{equation}

\noindent onde $\pi_t$ é a inflação no período $t$, $E_t \pi_{t+1}$ representa a expectativa racional da inflação futura e $\tilde{y}_t \equiv y_t - y_t^n$ denota o hiato do produto, definido como a diferença entre o produto efetivo e o produto natural sob preços flexíveis.

A NKPC evidencia dois mecanismos centrais. Primeiro, a inflação é um fenômeno essencialmente prospectivo: como os preços permanecem fixos por múltiplos períodos, as expectativas de inflação futura afetam diretamente a inflação corrente. Segundo, o hiato do produto influencia a inflação por meio de seu impacto sobre os custos marginais reais, refletindo pressões de demanda na economia.

O coeficiente $\kappa$ mede a sensibilidade da inflação ao ciclo econômico e é determinado estruturalmente por:
\begin{equation}
\kappa \equiv \lambda \left(\frac{\sigma(1-\alpha)+\varphi+\alpha}{1-\alpha}\right),
\qquad
\lambda \equiv \frac{(1-\theta)(1-\beta\theta)}{\theta},
\end{equation}
onde $\sigma$ é o inverso da elasticidade intertemporal de substituição do consumo, $\varphi$ é o inverso da elasticidade de Frisch da oferta de trabalho e $\alpha$ mede a participação do capital na função de produção (com rendimentos constantes de escala), de modo que $1-\alpha$ corresponde à participação do trabalho.

Quanto maior a rigidez nominal (isto é, quanto mais $\theta$ se aproxima de 1), menor é $\kappa$, reduzindo a resposta da inflação às flutuações do hiato do produto. Essa característica tem implicações diretas para a condução da política monetária: quanto menor $\kappa$, maior o papel das expectativas na determinação da inflação e maior a necessidade de regras de política que assegurem a estabilidade do equilíbrio macroeconômico, tema explorado na seção seguinte.

% -----------------------------------------------------------
\subsection{Equação IS Dinâmica}

A demanda agregada do modelo é descrita pela Equação IS Dinâmica, que relaciona o hiato do produto à taxa de juros real e às expectativas de atividade futura. Essa relação é obtida diretamente da log-linearização da equação de Euler do consumidor representativo, combinada com a condição de equilíbrio no mercado de bens, conforme detalhado em \citeonline{gali2015monetary}.

Formalmente, a maximização da utilidade intertemporal das famílias implica que o consumo corrente depende negativamente da taxa de juros real ex-ante e positivamente do consumo esperado para o futuro (suavização do consumo). Reescrevendo essa condição em termos de desvios do estado estacionário e expressando o produto em função de seu nível natural, chega-se à seguinte expressão para o hiato do produto:

\begin{equation}
\tilde{y}_t = E_t \tilde{y}_{t+1} - \frac{1}{\sigma} \big( i_t - E_t \pi_{t+1} - r_t^n \big),
\label{eq:dis}
\end{equation}

\noindent onde $\tilde{y}_t$ é o hiato do produto, $i_t$ é a taxa de juros nominal fixada pelo Banco Central, $E_t \pi_{t+1}$ é a inflação esperada e $\sigma$ representa o inverso da elasticidade de substituição intertemporal do consumo (ou coeficiente de aversão relativa ao risco).

A variável $r_t^n$ denota a \textbf{taxa de juros natural} (ou Wickselliana), definida por \citeonline{woodford2005interest} como a taxa de juros real de equilíbrio que prevaleceria na economia se os preços fossem totalmente flexíveis. No modelo básico, $r_t^n$ é determinada inteiramente por choques reais (como produtividade ou preferências) e independe da política monetária.

A equação \eqref{eq:dis} difere fundamentalmente da curva IS tradicional em dois aspectos:

\begin{itemize}
    \item \textbf{Natureza \textit{Forward-Looking}:} O hiato corrente $\tilde{y}_t$ depende não apenas dos juros hoje, mas da expectativa de hiato futuro $E_t \tilde{y}_{t+1}$. Isso significa que a política monetária afeta a economia tanto pela taxa de juros corrente quanto pela sua trajetória esperada (canal das expectativas).
    
    \item \textbf{O Juro Natural como Referência:} O termo relevante para a demanda agregada é o \textit{hiato da taxa de juros} $(i_t - E_t \pi_{t+1} - r_t^n)$. A política monetária é expansionista apenas se a taxa real de juros ($i_t - E_t \pi_{t+1}$) for fixada abaixo da taxa natural $r_t^n$. Caso o Banco Central mantenha a taxa real igual à taxa natural em todos os períodos, o hiato do produto será nulo ($\tilde{y}_t = 0$) e a inflação permanecerá estável, isolando a economia de choques de demanda ineficientes.
\end{itemize}

Essa estrutura teórica destaca que a estabilização macroeconômica exige que o Banco Central monitore e acompanhe as flutuações de $r_t^n$, ajustando a taxa nominal $i_t$ para evitar desvios persistentes em relação ao juro natural. A forma como a autoridade monetária operacionaliza esses ajustes em resposta à inflação e ao produto é descrita pelas Regras de Taylor, detalhadas a seguir.

Assim sendo, esse aspecto decorre do fato de que as equações centrais do modelo novo-keynesiano são intrinsecamente prospectivas. A inflação corrente depende das expectativas de inflação futura, enquanto o hiato do produto é determinado pela expectativa do hiato à frente e pela taxa de juros real esperada. Assim, a política monetária afeta o equilíbrio macroeconômico não apenas por meio de seus efeitos contemporâneos sobre a taxa de juros nominal, mas sobretudo pela forma como influencia a formação de expectativas privadas acerca da trajetória futura da economia.

% -----------------------------------------------------------
\subsection{Regra de Taylor}

A condução da política monetária no modelo Novo-Keynesiano (NK) é tipicamente descrita por meio de uma regra de juros que especifica como o banco central ajusta a taxa de juros nominal em resposta a desvios da inflação e do produto em relação aos seus níveis-alvo. \citeonline{taylor1993} propôs uma regra simples que captura o comportamento sistemático da política monetária nos Estados Unidos, estabelecendo que a taxa de juros nominal deve responder positivamente aos desvios da inflação em relação à meta e ao hiato do produto. Este \textit{framework} foi posteriormente generalizado por \citeonline{clarida2000monetary} para incorporar expectativas \textit{forward-looking} e suavização da taxa de juros, alinhando a regra de política monetária à estrutura intertemporal do modelo NK.

\subsubsection{Princípio de Taylor}

O Princípio de Taylor estabelece uma condição fundamental para a estabilidade macroeconômica sob uma regra de juros: o banco central deve ajustar a taxa de juros nominal mais que proporcionalmente a variações na inflação esperada, de modo que a taxa de juros real se eleve quando há pressões inflacionárias. Formalmente, considerando a regra de juros
\[
i_t = \rho + \phi_\pi E_t\pi_{t+1} + \phi_y \tilde{y}_t,
\]
o Princípio de Taylor requer que \(\phi_\pi > 1\). Em sua forma mais simples, esta condição assegura a determinação do equilíbrio e evita flutuações auto-realizáveis (\textit{sunspot fluctuations}), ao garantir que aumentos na inflação esperada sejam acompanhados por elevações mais que proporcionais da taxa de juros nominal.

A intuição por trás deste princípio é direta: quando \(\phi_\pi < 1\), um aumento na inflação esperada leva o banco central a elevar a taxa nominal em magnitude insuficiente, resultando em queda da taxa real de juros \(r_t = i_t - E_t\pi_{t+1}\). Esta redução estimula a demanda agregada, validando e amplificando as expectativas inflacionárias iniciais. Em contraste, quando \(\phi_\pi > 1\), o banco central eleva a taxa real de juros, contraindo a demanda e contribuindo para a estabilização da inflação. Em modelos mais gerais, com resposta explícita ao hiato do produto e regras antecipativas, a condição de determinância passa a depender de combinações entre \(\phi_\pi\), \(\phi_y\), \(\kappa\) e \(\beta\), conforme discutido por \citeonline{bullard2002learning}. 

Cabe notar que, em modelos novo-keynesianos com expectativas racionais, a introdução de uma regra de política monetária não esgota, por si só, a caracterização do equilíbrio. Embora a regra de Taylor forneça um mecanismo sistemático de reação da autoridade monetária a desvios da inflação e do produto, sua presença não garante, a priori, que o sistema dinâmico resultante admita uma única trajetória de equilíbrio consistente com as expectativas dos agentes. Em particular, em ambientes forward-looking, regras de juros insuficientemente reativas podem ser compatíveis com múltiplos equilíbrios racionais, fenômeno associado à possibilidade de trajetórias autorrealizáveis para inflação e produto.


\subsubsection{Especificações da Regra de Taylor}

Consideramos três especificações principais da regra de juros, que diferem quanto ao horizonte de previsão e ao grau de inércia da taxa de juros.

\paragraph{Regra \textit{Backward-Looking}}

A formulação original de \citeonline{taylor1993} especifica a taxa de juros como função da inflação corrente e do hiato do produto corrente:
\[
i_t = \rho + \phi_\pi \pi_t + \phi_y \tilde{y}_t + \varepsilon_t,
\]
onde \(\rho\) é a taxa de juros real de equilíbrio, \(\pi_t\) é a taxa de inflação no período \(t\), \(\tilde{y}_t\) é o hiato do produto, e \(\varepsilon_t\) é um choque de política monetária. O autor propôs valores específicos para os coeficientes — \(\rho = 2\%\), \(\phi_\pi = 1.5\) e \(\phi_y = 0.5\) — calibrados para aproximar a condução da política monetária do Federal Reserve no período 1987--1992 \cite[p.~202]{taylor1993}.

\paragraph{Regra \textit{Forward-Looking}}

De forma alternativa, \citeonline{clarida2000monetary} argumentam que a política monetária deve ser descrita por uma regra \textit{forward-looking}, na qual o banco central responde não à inflação e ao hiato do produto correntes, mas às suas expectativas:
\[
i_t^* = \rho + \phi_\pi E_t\pi_{t+k} + \phi_y E_t\tilde{y}_{t+q} + \varepsilon_t,
\]
onde \(i_t^*\) denota a taxa de juros alvo, e \(k\) e \(q\) representam os horizontes de previsão para inflação e produto, respectivamente. Esta especificação reconhece que os instrumentos de política monetária afetam a economia com defasagens consideráveis, tornando apropriado que o banco central reaja a projeções das variáveis-alvo. Na prática, a literatura frequentemente assume \(k=q=1\), embora horizontes mais longos possam ser considerados.

\paragraph{Regra com Suavização de Juros}

Evidências empíricas indicam que bancos centrais ajustam suas taxas de juros de forma gradual, evitando mudanças abruptas. Para capturar este comportamento, \citeonline{clarida2000monetary} propõem uma especificação com suavização parcial:
\[
i_t = \theta i_{t-1} + (1-\theta)\left[\rho + \phi_\pi E_t\pi_{t+k} + \phi_y E_t\tilde{y}_{t+q}\right] + \varepsilon_t,
\]
onde \(\theta \in [0,1)\) captura o grau de suavização. Quando \(\theta = 0\), recupera-se a regra sem inércia; quando \(\theta\) é próximo de 1, observa-se forte persistência nos ajustes da taxa de juros.

Complementando o que foi discutido na Seção 2, \citeonline{clarida2000monetary} estimam valores de \(\theta\) entre 0.7 e 0.9 para os Estados Unidos, sugerindo substancial gradualismo na condução da política monetária. A suavização pode refletir incerteza quanto aos efeitos da política, preocupação com a volatilidade dos mercados financeiros ou ainda uma estratégia ótima para o gerenciamento das expectativas em um ambiente \textit{forward-looking}.

\subsubsection{Regra Híbrida e Interpretação da Taxa Real Alvo}

Combinando as características anteriores, a regra híbrida mais geral pode ser escrita como:
\[
i_t = \theta i_{t-1} + (1-\theta)\left[\rho + \phi_\pi E_t\pi_{t+k} + \phi_y E_t\tilde{y}_{t+q}\right] + \varepsilon_t.
\]
Definindo a taxa real de juros alvo implícita na regra como \(r_t^* \equiv i_t^* - E_t\pi_{t+k}\), obtém-se:
\[
r_t^* = \rho + (\phi_\pi - 1)E_t\pi_{t+k} + \phi_y E_t\tilde{y}_{t+q}.
\]

Na prática, e em linha com grande parte da literatura empírica, adota-se frequentemente \(k=1\), caso em que \(E_t\pi_{t+k}=E_t\pi_{t+1}\). Note-se que \(r_t^*\) não corresponde, em geral, à taxa natural de juros no sentido wickselliano (isto é, à taxa real consistente com inflação estável e com o produto em seu nível natural, determinada exclusivamente por fatores reais da economia), mas à taxa real induzida pela reação sistemática da política monetária. Esta expressão evidencia o papel central do Princípio de Taylor: quando \(\phi_\pi > 1\), um aumento na inflação esperada eleva a taxa real de juros, produzindo efeito contracionista; quando \(\phi_\pi < 1\), a política torna-se acomodatícia, com a taxa real caindo em resposta a pressões inflacionárias.

No contexto do modelo Novo-Keynesiano, a regra de Taylor deve ser interpretada não como uma descrição ad hoc da política monetária, mas como o mecanismo de fechamento do sistema dinâmico formado pela Curva de Phillips Novo-Keynesiana e pela equação IS. A capacidade dessa regra de garantir um equilíbrio localmente único e estável depende, portanto, das condições de determinância associadas à estrutura expectacional do modelo, tema ao qual nos voltamos na próxima seção por meio das condições de Blanchard–Kahn e do critério de Bullard–Mitra.

Nesse contexto, a análise do equilíbrio deixa de se restringir à existência de soluções consistentes com as condições de otimalidade e de mercado, passando a envolver a questão da unicidade e da estabilidade dinâmica dessas soluções. Em particular, torna-se central investigar se a regra de política monetária adotada é capaz de selecionar uma única trajetória expectacional compatível com o equilíbrio racional, ou se, alternativamente, o modelo admite múltiplos equilíbrios locais. A próxima subseção dedica-se, portanto, à formalização das condições sob as quais o sistema novo-keynesiano apresenta um equilíbrio determinado.

% -----------------------------------------------------------
\subsection{Condições de Determinância}

A existência de um equilíbrio único e estável no modelo Novo-Keynesiano depende crucialmente da regra de política monetária adotada e de sua capacidade de ancorar as expectativas dos agentes. Formalmente, a dinâmica do sistema linearizado pode ser representada na forma canônica de \citeonline{blanchard1980solution}, onde a unicidade do equilíbrio exige que o número de autovalores instáveis (módulo superior a um) da matriz de transição seja igual ao número de variáveis não-predeterminadas (\textit{forward-looking}), como a inflação e o hiato do produto.

Para o modelo NK linearizado sob uma regra de Taylor que responde a variáveis contemporâneas, uma condição padrão para determinância local e estabilidade sob aprendizagem, conforme \citeonline{bullard2002learning}, é dada por:

\[
\kappa(\phi_\pi - 1) + (1-\beta)\phi_y > 0.
\]

Nesta inequação, derivada da análise dos autovalores da matriz $A_T$ do sistema dinâmico, $\kappa$ representa a inclinação da Curva de Phillips Novo-Keynesiana e $\beta$ o fator de desconto intertemporal. A condição estabelece um \textit{trade-off} entre a resposta à inflação ($\phi_\pi$) e ao hiato do produto ($\phi_y$). O Princípio de Taylor, caracterizado por $\phi_\pi > 1$, é uma condição suficiente para a determinância sempre que $\phi_y \geq 0$, garantindo que a taxa de juros real suba diante de pressões inflacionárias, o que contrai a demanda agregada e estabiliza os preços.

Além de assegurar a unicidade sob expectativas racionais, \citeonline{bullard2002learning} e \citeonline{christiano2018does} argumentam que essa mesma condição é fundamental para a estabilidade sob aprendizagem (\textit{E-stability}). Em modelos onde os agentes não possuem conhecimento perfeito da estrutura da economia e formam expectativas via algoritmos de aprendizagem adaptativa, o cumprimento dessa condição evita trajetórias explosivas e garante a convergência para o Equilíbrio de Expectativas Racionais (REE). A violação dessa regra, por outro lado, abre espaço para a indeterminação real e múltiplos equilíbrios, incluindo trajetórias de \textit{sunspots} onde choques não fundamentais afetam a economia real.

% ***********************************************************

% ===========================================================
\section{Política Monetária no Brasil entre 1999 e 2024}
% ===========================================================

Nesta seção, apresentamos uma análise descritiva da condução da política monetária no Brasil desde a adoção do regime de metas de inflação em 1999 até o presente. Inicialmente, contextualizamos o regime de metas de inflação, destacando sua arquitetura institucional e sistemática de funcionamento. Em seguida, examinamos a evolução macroeconômica do país ao longo dos diferentes mandatos presidenciais do Banco Central, identificando os principais desafios enfrentados e as respostas adotadas. Por fim, sintetizamos fatos estilizados observados em cada regime de política monetária, preparando o terreno para a análise econométrica subsequente.

% -----------------------------------------------------------
\subsection{Regime de Metas de Inflação}

O regime de metas de inflação, adotado no Brasil oficialmente a partir de janeiro de 1999 após o abandono do regime de bandas cambiais, com o objetivo de restabelecer uma âncora nominal crível para a política monetária, e, no formato avaliado neste artigo, vigente até dezembro de 1999 \cite{brasil1999decreto}, seguiu a experiência internacional consolidada ao longo da década de 1990, na qual a estabilidade de preços passou a ser explicitamente definida como objetivo primário da política monetária. Nesse arcabouço, a autoridade monetária anuncia uma meta numérica para a inflação em horizonte relevante, utilizando a taxa de juros de curto prazo como principal instrumento para ancorar as expectativas dos agentes econômicos \cite{bernanke1997inflation,hammond2012state}.

Do ponto de vista institucional, o regime brasileiro é caracterizado por uma meta de inflação definida pelo Conselho Monetário Nacional (CMN), acompanhada de bandas de tolerância, enquanto a condução operacional da política monetária cabe ao Banco Central do Brasil, por meio do Comitê de Política Monetária (Copom). A sistemática do regime enfatiza transparência, comunicação e responsabilização, com a divulgação regular de relatórios, atas e projeções, em linha com as práticas observadas em economias que adotam metas de inflação. Conforme destacado por \citeonline{bernanke1997inflation}, o regime deve ser interpretado como um \textit{framework} de política monetária, e não como uma regra mecânica, preservando flexibilidade para acomodar choques de curto prazo sem comprometer o objetivo de estabilidade de preços no médio prazo.

Embora o arcabouço formal do regime de metas de inflação tenha permanecido inalterado desde sua adoção, a literatura sugere que a condução da política monetária pode refletir variações ao longo do tempo na ênfase atribuída à inflação, ao hiato do produto e à estabilidade financeira, especialmente em economias emergentes sujeitas a choques externos e restrições macroeconômicas específicas \cite{neves2008regime,arestis2009nova}. Essas possíveis variações não configuram rupturas institucionais do regime, mas indicam mudanças na função de reação implícita da autoridade monetária, aspecto que motiva a análise empírica desenvolvida nas seções subsequentes.

% -----------------------------------------------------------
\subsection{Evolução Macroeconômica}

O período 1999–2024, adotado como critério neste trabalho, marca a consolidação do tripé macroeconômico, adotado em resposta ao esgotamento da âncora cambial do Plano Real. Segundo \citeonline{bogdanski2000implementing}, a crise de confiança de 1999 forçou a transição para o câmbio flutuante e exigiu uma nova âncora nominal para as expectativas, resultando na implementação do Regime de Metas para a Inflação. Esse arranjo institucional pautou a reação do Banco Central aos choques subsequentes, estabelecendo a estabilidade de preços como objetivo primário sob um ambiente de maior flexibilidade cambial.

Em consonância com o objeto, problema e objetivos deste trabalho, a análise descritiva por períodos associados às diferentes presidências do Banco Central do Brasil é particularmente relevante em uma economia emergente, caracterizada por elevada exposição a choques externos e recorrentes episódios de incerteza macroeconômica. Evidências recentes, como as de \citeonline{barros2023geopolitical}, indicam que choques de risco geopolítico internacional afetam de forma significativa a atividade econômica, a inflação, a taxa de juros e os prêmios de risco no Brasil, reforçando a importância de considerar a possibilidade de respostas diferenciadas da política monetária ao longo do tempo.

\subsection{Ciclos de Gestão e Choques Estruturais}

A condução da política monetária ao longo desse intervalo foi marcada por cinco administrações principais no Banco Central, cada uma operando sob desafios distintos impostos pela conjuntura governamental e econômica. A gestão de \textbf{Armínio Fraga} (\emph{mar/1999--dez/2002}), durante o segundo governo de Fernando Henrique Cardoso, foi responsável pela implementação e consolidação inicial do regime de metas de inflação. Seguiu-se a longa administração de \textbf{Henrique Meirelles} (\emph{jan/2003--dez/2010}), abrangendo os dois mandatos do governo Lula, período caracterizado pelo acúmulo de reservas internacionais e pelo enfrentamento da Crise Financeira Global de 2008. \textbf{Alexandre Tombini} (\emph{jan/2011--jun/2016}) presidiu a autarquia durante o governo Dilma Rousseff e o início da crise econômica doméstica. \textbf{Ilan Goldfajn} (\emph{jun/2016--fev/2019}), nomeado pelo governo Temer, conduziu o processo de reancoragem das expectativas inflacionárias e o ciclo de flexibilização monetária no pós-recessão. Por fim, \textbf{Roberto Campos Neto} (\emph{fev/2019--dez/2024}), atuando sob os governos Bolsonaro e no início do terceiro mandato de Lula, enfrentou os choques inflacionários globais decorrentes da pandemia de Covid-19 e liderou a transição para a autonomia formal do Banco Central.

Para a identificação dos episódios de ruptura, adotou-se um critério duplo: a cronologia oficial de recessões estabelecida pelo Comitê de Datação de Ciclos Econômicos \cite{codace2020} para a delimitação temporal, combinada à caracterização da natureza dos choques predominante na literatura especializada, exemplificada por \citeonline{barbosa2017}. Sob essa métrica, destacam-se quatro eventos estruturais: a \textbf{crise de confiança de 2002} (\emph{jun/2002--jan/2003}), associada à incerteza eleitoral e à parada súbita de capitais; a \textbf{Crise Financeira Global de 2008} (\emph{set/2008--mar/2009}), que impôs um severo choque de liquidez externa; a \textbf{recessão doméstica de 2015--2016} (\emph{jan/2015--dez/2016}), deflagrada por desajustes fiscais e realinhamento de preços administrados; e a \textbf{pandemia de Covid-19} (\emph{mar/2020--dez/2020}), choque sanitário que exigiu estímulos monetários extraordinários. Inclusive, este cenário de deterioração fiscal impõe o teste prático da hipótese de dominância fiscal levantada na introdução, onde a restrição orçamentária pode ter limitado a capacidade de reação do BC via juros.

\subsection{Indicadores Econômicos}

Esta subseção descreve, de forma sintética, a evolução conjunta de inflação, taxa de juros, expectativas e atividade ao longo do regime de metas, com ênfase nos principais episódios de ruptura destacados na Figura \ref{fig:series_macro}. O objetivo é registrar fatos estilizados e motivar a estratificação por subperíodos na etapa econométrica, sem antecipar inferências causais.

Antes de discutir os episódios, convém explicitar uma dimensão operacional dos dados. A Tabela \ref{tab:desc_stats} resume estatísticas descritivas das séries utilizadas, construídas a partir de bases do BCB e do IBGE. Em particular, parte das variáveis é observada originalmente em frequência mensal ou trimestral (p.ex., IPCA e PIB), enquanto outras são observadas com maior granularidade (p.ex., Selic-meta e expectativas). Para uniformizar o painel e viabilizar comparações e gráficos ao longo do tempo, as séries foram alinhadas em uma base temporal comum, com replicação do valor observado dentro do período correspondente quando necessário. Esse procedimento explica as contagens elevadas (\textit{count}) reportadas na tabela, sem alterar a interpretação econômica das trajetórias ilustradas nas figuras.

\begin{table}[htbp]
\centering
\caption{Estatísticas Descritivas das Variáveis Macroeconômicas (1999--2024)}
\begin{tabular}{lrrrrr}
\toprule
 & IPCA (Mensal) & IPCA (12m) & Selic (Meta) & Focus & PIB (Indice) \\
\midrule
count & 9437.00 & 9437.00 & 9434.00 & 9132.00 & 9437.00 \\
mean & 0.51 & 6.25 & 12.69 & 5.63 & 143.19 \\
std & 0.39 & 2.72 & 5.55 & 1.93 & 23.25 \\
min & $-$0.68 & 1.65 & 2.00 & 1.61 & 99.26 \\
25\% & 0.26 & 4.48 & 9.00 & 4.24 & 120.84 \\
50\% & 0.46 & 5.91 & 12.25 & 5.50 & 153.66 \\
75\% & 0.69 & 7.35 & 16.00 & 6.42 & 161.34 \\
max & 3.02 & 17.24 & 45.00 & 12.41 & 179.53 \\
\bottomrule
\end{tabular}
\label{tab:desc_stats}
\footnotesize

\vspace{0.2cm}
\textbf{Fontes:} Bacen e IBGE, via pacotes Python (sidrapy e python-bcb).
\end{table}

% Crise de Confiança 2002-2003
No início da amostra, destaca-se a chamada \enquote{Crise de Confiança} de 2002--2003, associada à transição do governo Fernando Henrique Cardoso para o governo Luiz Inácio Lula da Silva e ao ajuste de expectativas que se refletiu em forte depreciação cambial. Conforme indicado na Figura \ref{fig:series_macro}, observa-se elevação acentuada da inflação acumulada em 12 meses e aumento expressivo da taxa Selic no período sombreado (jun/2002--jan/2003), em paralelo à deterioração das expectativas medidas pelo Focus. Do ponto de vista descritivo, esse episódio é caracterizado por uma reação monetária contracionista em um ambiente de desancoragem de expectativas, seguida por reconvergência gradual da inflação e das expectativas no ciclo subsequente. Essa dinâmica sugere, como hipótese a ser testada formalmente, que a autoridade monetária buscou reancorar expectativas por meio de um aumento relevante da taxa real ex-ante, compatível com a lógica do Princípio de Taylor no arcabouço novo-keynesiano.

% Crise Financeira Global de 2008--2009
Entre 2004 e 2010, a Figura \ref{fig:series_macro} sugere um intervalo de relativa estabilidade macroeconômica, com crescimento do produto e inflação moderada na maior parte do período, interrompido pela Crise Financeira Global de 2008--2009. O choque externo aparece como queda do nível de atividade e elevação transitória da incerteza, seguida por recuperação relativamente rápida do PIB. Do ponto de vista descritivo, o comportamento da Selic no episódio é consistente com uma resposta contracíclica: redução da taxa nominal em ambiente de desaceleração, com expectativas que, em geral, permanecem mais próximas do alvo quando comparadas aos episódios de maior desancoragem observados em outros momentos da amostra.

% Recessão Doméstica de 2015--2016
A partir de 2014, observa-se uma mudança de regime marcada pela recessão doméstica de 2015--2016. Na Figura \ref{fig:series_macro}, o período sombreado (jan/2015--dez/2016) combina queda persistente do produto com inflação elevada e expectativas pressionadas, configurando um ambiente próximo à estagflação. Essa combinação é particularmente relevante para o objetivo do artigo: em termos de interpretação novo-keynesiana, um cenário com hiato negativo coexistindo com inflação alta é compatível com a presença de choques de oferta, realinhamentos de preços relativos ou deterioração do componente expectacional, fatores que podem reduzir a efetividade de uma resposta monetária convencional. Ainda assim, os dados sugerem uma postura contracionista em parte do intervalo, com juros elevados em meio à inflação e expectativas acima do desejado, motivando investigar se a resposta estimada à inflação foi suficiente para satisfazer $\hat{\phi}_\pi>1$ nesse subperíodo.

\begin{figure}[htbp]
    \centering
    \caption{Dinâmica da Inflação, Taxa de Juros, Expectativas e PIB (1999--2024)}
    \includegraphics[width=1\textwidth]{graficos/series_macroeconomicas.png}
    \label{fig:series_macro}
    \footnotesize
    \vspace{0.2cm}
    \textbf{Fontes:} Bacen e IBGE, via pacotes Python (sidrapy e python-bcb).\\
\end{figure}

% Pandemia de Covid-19
O último grande choque da amostra corresponde à pandemia de Covid-19 (mar/2020--dez/2020). A Figura \ref{fig:series_macro} mostra a contração abrupta do nível de atividade e a resposta inicial de redução da Selic em um contexto de forte queda do produto. Em seguida, a normalização parcial do PIB e a reversão inflacionária de 2021--2022 vêm acompanhadas por elevação expressiva dos juros e por deterioração temporária das expectativas, sugerindo um ciclo de aperto robusto voltado à reancoragem nominal. A avaliação quantitativa da intensidade dessa reação, contudo, depende da estimação formal da função de reação e será tratada na Seção 6.

A Figura \ref{fig:hiato_produto} reforça um ponto metodológico central: embora as estimativas de hiato difiram quanto à magnitude e à volatilidade (notadamente no Filtro de Hamilton), os sinais e os pontos de inflexão do ciclo são amplamente coincidentes \footnote{As estimativas de hiato do produto diferem conforme o critério estatístico adotado para separar tendência e componente cíclico. O filtro de Hodrick--Prescott (HP) extrai a tendência como solução de um problema de suavização penalizando variações na segunda diferença do produto, sendo amplamente utilizado em aplicações macroeconômicas, embora sensível à escolha do parâmetro de suavização e a problemas de viés nas extremidades da amostra \cite{hodrick1997postwar}. O filtro de Baxter--King (BK), por sua vez, é um filtro "passa-banda" que isola flutuações em frequências associadas ao ciclo de negócios, ao custo de perda de observações no início e no fim da amostra \cite{baxter1999measuring}. Por fim, o método proposto por \citeonline{hamilton2018filtering} evita a extração explícita de tendência ao estimar previsões autorregressivas do produto e interpretar os resíduos como componente cíclico, reduzindo vieses de ponta, mas produz séries de hiato tipicamente mais voláteis.}. Em particular, os episódios recessivos de 2015--2016 e 2020 aparecem de forma robusta nas diferentes metodologias, sugerindo que, apesar da incerteza inerente à mensuração do produto potencial, o diagnóstico qualitativo sobre o estado cíclico da economia é relativamente estável.

Por fim, ressalta-se que a análise aqui permanece estritamente descritiva. A co-movimentação entre inflação, expectativas, produto e taxa de juros documentada nas Figuras \ref{fig:series_macro} e \ref{fig:hiato_produto} é consistente com os mecanismos previstos pelo modelo novo-keynesiano, mas não permite inferências causais nem conclusões sobre determinância do equilíbrio. A presença de defasagens, simultaneidade e o papel das expectativas requerem uma abordagem econométrica formal, desenvolvida nas seções seguintes, na qual se testará explicitamente se a resposta da política monetária, em diferentes subperíodos, foi consistente com as condições teóricas de unicidade e estabilidade do equilíbrio.

\begin{figure}[htbp]
    \centering
    \includegraphics[width=1\textwidth]{graficos/hiato_produto.png}
    \caption{Estimativas de Hiato do Produto sob Diferentes Metodologias (1999--2024)}
    \label{fig:hiato_produto}
    \footnotesize
    \textbf{Fonte:} Elaboração própria, baseada em dados do Bacen, coletados via pacotes Python (python-bcb).\\
    \textit{\textbf{Nota:} Comparativo entre Filtro HP, Filtro de Hamilton, Tendência Linear, Quadrática e estimativa da IFI. Valores negativos indicam ociosidade da economia.}
\end{figure}

Os padrões documentados nesta seção sugerem que a condução da política monetária brasileira ao longo do regime de metas não foi invariável no tempo, apresentando diferenças sistemáticas na forma e na intensidade da resposta da taxa de juros a desvios da inflação, do hiato do produto e das expectativas. A evidência descritiva (ver Figura~\ref{fig:series_macro} e Tabela \ref{tab:desc_stats}) aponta para episódios em que o aperto monetário foi suficientemente agressivo para conter processos de desancoragem das expectativas inflacionárias, como no início do regime de metas, em 2002--2003, e mais recentemente no ciclo de alta iniciado em 2021, quando a taxa Selic foi elevada de forma rápida e persistente em resposta à deterioração das expectativas captadas pelo boletim Focus. 

Por outro lado, os dados também indicam períodos em que choques adversos de demanda ou de oferta impuseram trade-offs mais severos à estabilização macroeconômica, como durante a crise financeira internacional de 2008--2009 e a pandemia de COVID-19, quando a política monetária operou em ambiente de elevada incerteza e forte contração do produto, bem como no episódio de 2015--2016, caracterizado por inflação elevada, recessão profunda e dificuldades adicionais de coordenação entre política monetária e fiscal. No entanto, tais inferências não podem ser estabelecidas de forma conclusiva a partir de inspeção visual das séries, uma vez que inflação, atividade, expectativas e juros são determinados de forma conjunta, dinâmica e endógena, com defasagens relevantes. Diante disso, a Seção 5 propõe uma estratégia de identificação formal baseada na estimação de diferentes especificações da Regra de Taylor (retrospectiva, prospectiva e híbrida) explorando segmentações temporais e possíveis quebras estruturais, com o objetivo de avaliar se os parâmetros estimados satisfazem o Princípio de Taylor e são compatíveis com as condições de determinância de \citeonline{bullard2002learning}.

% ***********************************************************

% ===========================================================
\section{Metodologia}
% ===========================================================

\subsection{Dados}

\subsubsection{Variáveis}

\begin{itemize}
    \item Taxa Selic ($i_t$).
    \item Inflação (IPCA).
    \item PIB real e cálculo do hiato do produto.
    \item Expectativas de inflação (Focus).
\end{itemize}

\subsubsection{Fontes}

\begin{itemize}
    \item Banco Central do Brasil (SGS).
    \item IBGE.
    \item FGV/IBRE.
\end{itemize}

\subsubsection{Tratamento dos Dados}

\begin{itemize}
    \item Frequência (mensal ou trimestral).
    \item Ajuste sazonal.
    \item Período amostral e janelas de crise.
\end{itemize}

\subsection{Especificação das Equações}

\begin{itemize}
    \item Regras de Taylor backward (MQO/GLS).
    \item Regras forward (GMM).
    \item Regra híbrida com defasagens de juros (suavização).
\end{itemize}

\subsection{Estratégia Econométrica}

\begin{itemize}
    \item Justificar MQO com erros robustos (Newey--West).
    \item GMM com instrumentos defasados.
    \item Estimação por subperíodos (regimes/presidentes do BCB).
\end{itemize}

\subsection{Teste do Princípio de Taylor e da Condição de Bullard--Mitra}

\begin{itemize}
    \item Verificar empiricamente se $\hat\phi_\pi > 1$.
    \item Avaliar a condição $\kappa(\hat\phi_\pi - 1) + (1-\beta)\hat\phi_y > 0$.
\end{itemize}

\subsection{Quebras Estruturais}

\begin{itemize}
    \item Testes de Bai--Perron.
    \item Datas-chave: 2003, 2011, 2016, 2019.
\end{itemize}

\subsection{Simulações em Dynare (Opcional)}

A implementação computacional do modelo Novo-Keynesiano no software Dynare desempenha um papel complementar à análise econométrica, funcionando como um laboratório para avaliar a dinâmica teórica implícita nos parâmetros estimados. Seus objetivos específicos são:

\begin{itemize}
    \item \textbf{Verificação das Condições de Estabilidade (Blanchard--Kahn):} Utilizar o algoritmo do Dynare para testar computacionalmente se os conjuntos de parâmetros estimados para cada regime ($\hat{\phi}_\pi, \hat{\phi}_y$) satisfazem as condições de existência e unicidade do equilíbrio racional. Isso permite confirmar se regimes com $\hat{\phi}_\pi < 1$ geram, de fato, diagnósticos de indeterminação ou trajetórias explosivas no modelo teórico.
    
    \item \textbf{Análise Dinâmica via Funções de Resposta ao Impulso (IRFs):} Simular a trajetória da inflação, do produto e dos juros diante de choques exógenos (oferta e demanda) sob diferentes regras de política. O objetivo é visualizar como a agressividade da resposta monetária (ou a falta dela) altera a velocidade de convergência da inflação para a meta e a volatilidade do ciclo econômico.
    
    \item \textbf{Exercícios Contrafactuais:} Avaliar cenários hipotéticos, como o comportamento da inflação em períodos de crise caso a regra de política monetária adotada fosse a de um regime de maior credibilidade (ex: aplicar os parâmetros do período $T_2$ aos choques do período $T_1$). Isso ajuda a isolar a contribuição da conduta do Banco Central para a estabilidade macroeconômica, segregando-a dos choques estruturais.
\end{itemize}


\begin{itemize}
    \item Equações: NKPC, IS, Regra de Taylor.
    \item Calibração de $\beta, \sigma, \kappa, \phi_\pi, \phi_y$.
    \item Funções de resposta a impulso (IRFs) sob diferentes regimes.
\end{itemize}

% ***********************************************************

% ===========================================================
\section{Resultados}
% ===========================================================

\subsection{Estimações da Regra de Taylor}

\begin{itemize}
    \item Tabelas com $\hat\phi_\pi$, $\hat\phi_y$, termo constante e, se incluída, inércia na taxa de juros.
    \item Resultados para a amostra completa (1999--2024).
\end{itemize}

\subsection{Teste do Princípio de Taylor}

\begin{itemize}
    \item Verificação de $\hat\phi_\pi > 1$ na amostra completa.
    \item Discussão sobre significância estatística e intervalos de confiança.
\end{itemize}

\subsection{Resultados por Regimes de Política Monetária}

\begin{itemize}
    \item Estimações separadas por presidente do Banco Central: Fraga, Meirelles, Tombini, Goldfajn e Campos Neto.
    \item Comparação dos coeficientes $\hat\phi_\pi$ e $\hat\phi_y$ entre regimes.
    \item Identificação de mudanças na força da reação da política monetária ao longo do tempo.
    \item Verificação do cumprimento (ou não) do Princípio de Taylor em cada regime.
    \item Implicações para a condição de determinância do modelo em subperíodos distintos.
\end{itemize}

\subsection{Simulações (Opcional)}

\begin{itemize}
    \item IRFs para choques monetários sob diferentes valores de $\phi_\pi$ e $\phi_y$.
    \item Comparação entre regimes determinantes e indeterminantes.
\end{itemize}

% ***********************************************************

% ===========================================================
\section{Discussão}
% ===========================================================

\subsection{Interpretação Econômica}

\begin{itemize}
    \item Implicações de $\phi_\pi > 1$ para credibilidade e eficácia da política monetária.
    \item Relação entre reação da taxa de juros, estabilização da inflação e volatilidade do hiato do produto.
\end{itemize}

\subsection{Discussão dos Resultados por Regimes}

\begin{itemize}
    \item Interpretação econômica das diferenças entre regimes.
    \item Relação entre arcabouço institucional, composição do Copom e parâmetros estimados.
    \item Possíveis explicações para mudanças na força da reação à inflação.
    \item Implicações para credibilidade, transparência e comunicação da política monetária.
    \item Comparação com a literatura brasileira de política monetária por períodos.
\end{itemize}

\subsection{Comparação com a Literatura}

\begin{itemize}
    \item Convergência ou divergência dos resultados com estudos prévios para o Brasil.
    \item Diferenças metodológicas em relação à literatura internacional (CGG, Bullard--Mitra etc.).
\end{itemize}

\subsection{Implicações de Política}

\begin{itemize}
    \item Relevância da forte reação à inflação em regimes de metas.
    \item Papel da comunicação, \textit{forward guidance} e credibilidade na sustentação de $\phi_\pi > 1$.
\end{itemize}

\subsection{Limitações}

\begin{itemize}
    \item Medição do hiato do produto.
    \item Qualidade e horizonte das expectativas de inflação.
    \item Simplicidade da regra estimada frente a modelos DSGE completos.
\end{itemize}

% ***********************************************************

% ===========================================================
\section{Considerações Finais}
% ===========================================================

\begin{itemize}
    \item Responder ao problema de pesquisa, sintetizando em que medida a política monetária brasileira satisfez o Princípio de Taylor e a condição de determinância.
    \item Avaliar a compatibilidade do regime de metas com a determinância ao longo dos diferentes regimes de política monetária.
    \item Destacar a contribuição teórica e empírica do artigo para o debate sobre estabilidade nominal, credibilidade e dominância fiscal.
    \item Sugerir extensões (DSGE estimado, interação fiscal-monetária, comparações internacionais).
\end{itemize}

\clearpage
\pagenumbering{Roman}
\bibliographystyle{abntex2-alf}
\bibliography{library}

\end{document}