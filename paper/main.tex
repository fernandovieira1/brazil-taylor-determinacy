%!TEX program = pdflatex

%% ORGANIZAÇÃO DO PAPER
% Seções 2–3: o que a teoria diz que deveria acontecer
% Seção 4: o que parece ter acontecido nos dados
% Seção 5: como vamos testar isso formalmente
% Seção 6: o que os testes mostram

% abntex2/memoir já carrega `url` e `hyperref`; passe opções ANTES do \documentclass
% para habilitar quebras de link/URL sem causar option clash.
% \PassOptionsToPackage{hyphens}{url}
% \PassOptionsToPackage{unicode,breaklinks,hidelinks}{hyperref}

\documentclass[12pt,a4paper]{abntex2}

% --- CITAÇÕES E REFERÊNCIAS ---
% Pacote para citações no padrão ABNT, necessário para \citeonline
\usepackage[alf]{abntex2cite}

% --- CONFIGURAÇÃO DE IDIOMA E LEGENDAS ---
% Observação: em algumas combinações de TeXLive/babel/abntex2, o idioma pode
% ser ativado como 'brazil' (deprecated) em vez de 'brazilian'. Por isso,
% aplicamos a configuração para ambos.
\addto\captionsbrazilian{%
\renewcommand{\figurename}{Gráfico}%
\renewcommand{\listfigurename}{Lista de Gráficos}%
}
\addto\captionsbrazil{%
\renewcommand{\figurename}{Gráfico}%
\renewcommand{\listfigurename}{Lista de Gráficos}%
}

% Força o texto do rótulo no caption (independe de \figurename).
\makeatletter
\AtBeginDocument{%
    \renewcommand{\fnum@figure}{Gráfico~\thefigure}%
}
\makeatother

% --- ESTRUTURA DO DOCUMENTO ---
\emergencystretch=2em
\raggedbottom

% Remove o "capítulo." das seções
\renewcommand{\thesection}{\arabic{section}}
\renewcommand{\thesubsection}{\thesection.\arabic{subsection}}
\renewcommand{\thesubsubsection}{\thesubsection.\arabic{subsubsection}}

% ================== CODIFICAÇÃO ==================
\usepackage[T1]{fontenc}
\usepackage[utf8]{inputenc}

% ================== PACOTES MATEMÁTICOS, TABELAS E GRÁFICOS ==================
\usepackage{amsmath,amssymb,amsfonts}
\usepackage{graphicx}
\usepackage{booktabs}
\usepackage{tabularx}
\usepackage{float}

% ================== FORMATAÇÃO ==================
\usepackage{geometry}
\usepackage{setspace}
\geometry{margin=2.5cm}
\OnehalfSpacing
\usepackage{xcolor}

% ================== TEXTO E LINKS ==================
\usepackage{csquotes}
% Quebra URLs longas (melhora links no fim de linha/página)
\usepackage{xurl}
\Urlmuskip=0mu plus 1mu\relax

% ================== INFORMAÇÕES DO ARTIGO ==================
\title{Determinância e Política Monetária no Brasil:\
Uma Avaliação Empírica do Princípio de Taylor (1999--2024)}
\author{
Fernando Souza de Vieira\\
Rafaela Dezidério Rocha
}
\date{}

% ***********************************************************
\begin{document}

%% ANTES DA COMPILAÇÃO FINAL, VERIFICAR (CTRL+F) '<->'
% <-> SIGNIFICA QUE FALTA AJUSTAR ALGO

\maketitle
\thispagestyle{empty}

\begin{abstract}
\textcolor{magenta}{<-> MUDAR DEPOIS DE TERMINAR} - Este artigo avalia empiricamente a consistência da política monetária brasileira com o Princípio de Taylor no período 1999–2024, utilizando diferentes especificações da regra de juros estimadas por MQO e GMM. Os resultados indicam que, embora a reação da taxa de juros à inflação seja estatisticamente significativa, a elevada persistência da taxa básica e a fragilidade das respostas de longo prazo tornam a classificação do regime como determinante sensível à interpretação dos parâmetros associados à suavização da taxa de juros. Esses achados sugerem que o cumprimento empírico do Princípio de Taylor não é suficiente, por si só, para garantir determinância robusta no arcabouço Novo-Keynesiano. O estudo estima diferentes especificações da Regra de Taylor -- \textit{backward-looking}, \textit{forward-looking} no estilo Clarida--Galí--Gertler e versões híbridas com suavização da taxa de juros -- utilizando dados da taxa Selic, inflação (IPCA), hiato do produto e expectativas de inflação (Focus), no período 1999--2024. A metodologia combina MQO com erros robustos, GMM com instrumentos defasados, testes de quebras estruturais (Bai--Perron) e avaliação formal da condição de Bullard--Mitra, que relaciona os parâmetros estimados à determinância do modelo Novo-Keynesiano. Os resultados permitem identificar se, e em quais subperíodos, a reação da política monetária à inflação foi suficientemente forte ($\phi_\pi > 1$) para garantir unicidade do equilíbrio e ancoragem das expectativas. A contribuição central do artigo consiste em estabelecer um diálogo sistemático entre o arcabouço Novo-Keynesiano e a experiência brasileira recente, destacando como diferentes regimes de política monetária se relacionam com a estabilidade nominal e a credibilidade do Banco Central \textcolor{magenta}{<-> MUDAR DEPOIS DE TERMINAR}.
\end{abstract}

\pagestyle{plain}

% ***********************************************************

% ===========================================================
\section{Introdução}
% ===========================================================
\label{sec:introducao}

Desde 1999, o regime de metas para a inflação consolidou um novo arcabouço de política monetária no Brasil, ancorado na transparência da atuação do Banco Central e na necessidade de compatibilizar estabilidade nominal e crescimento econômico. Ao longo desse período, a condução da taxa Selic passou a ser orientada por uma lógica sistemática de resposta a desvios da inflação em relação à meta e, em menor medida, a flutuações no hiato do produto \cite{carrara2012regime}. Embora tal lógica nem sempre seja explicitada em regras formais, a literatura reconhece que o comportamento histórico do Banco Central brasileiro aproxima-se de uma função de reação do tipo Taylor \cite{de2013modified}.

Conforme será argumentado teoricamente na Seção~\ref{subsec:regra_taylor}, a Regra de Taylor especifica que a taxa básica de juros reage sistematicamente ao desvio da inflação em relação à meta e ao hiato do produto, elevando-se quando inflação e atividade estão acima de seus níveis desejados e reduzindo-se no caso oposto. O Princípio de Taylor, por sua vez, formaliza uma exigência mínima de estabilização: para que a política monetária evite equilíbrios múltiplos e trajetórias instáveis, o coeficiente de resposta da taxa de juros à inflação deve exceder a unidade. Essa condição, aparentemente simples, é decisiva em modelos Novo-Keynesianos, nos quais a dinâmica macroeconômica depende criticamente de expectativas e de rigidez nominal \cite{taylor1993}.

% -----------------------------------------------------------
\subsection{Contexto e Motivação}
\label{subsec:contexto_motivacao}

Segundo \citeonline[p.~120]{balbinoPoliticaMonetariaBrasileira2011}, a credibilidade da autoridade monetária, elemento central do regime de metas, está diretamente associada à capacidade de reagir de forma consistente a choques inflacionários. Nesse sentido, a violação do Princípio de Taylor pode gerar instabilidade nominal, indeterminação do equilíbrio e sensibilidade excessiva a choques exógenos e revisões nas expectativas. Essa necessidade de reação consistente conecta-se diretamente à estabilidade dinâmica discutida na literatura Novo-Keynesiana: quando a autoridade monetária não reage de forma suficientemente forte à inflação, abre-se espaço para desancoragem de expectativas e equilíbrios autorrealizáveis \cite{woodford2003interest,bullard2002learning}.

Um elemento adicional que condiciona a eficácia da política monetária é o risco de dominância fiscal. Quando a dinâmica da dívida pública restringe o espaço de atuação do Banco Central, aumentos da taxa Selic podem comprometer a sustentabilidade fiscal, reduzindo a autonomia da política monetária e dificultando a implementação de respostas agressivas à inflação. Em tais cenários, a condição $\phi_{\pi} > 1$ pode tornar-se inviável na prática, ainda que desejável do ponto de vista teórico \cite{kumhof2010simple}.

% -----------------------------------------------------------
\subsection{Problema de Pesquisa}
\label{subsec:problema_pesquisa}

O problema de pesquisa que orienta este artigo pode ser sintetizado da seguinte forma:

\begin{itemize}
\item[] \textbf{Até que ponto a política monetária brasileira, no período 1999--2024, reagiu à inflação de forma suficientemente forte para satisfazer o Princípio de Taylor e garantir determinância no sentido do modelo Novo-Keynesiano?} 

%% SE DER TEMPO, ESTIMAR POR PERÍODOS E PRESIDENTES DO BC
% Ademais, como esse padrão de reação variou entre os diferentes presidentes do Banco Central?
\end{itemize}

% -----------------------------------------------------------
\subsection{Objeto do Estudo}
\label{subsec:objeto_estudo}

\textbf{O objeto deste estudo consiste na estimativa da função de reação da taxa Selic à inflação e ao hiato do produto no regime de metas de inflação brasileiro}, a partir de diferentes formulações empíricas da Regra de Taylor: retrospectivas, prospectivas e híbridas com inércia. Essa função de reação é empregada para verificar a validade empírica do Princípio de Taylor e, de forma complementar, avaliar se os parâmetros obtidos asseguram determinância do equilíbrio no modelo Novo-Keynesiano.

% -----------------------------------------------------------
\subsection{Objetivos}
\label{subsec:objetivos}

Antes de explicitar os objetivos do estudo, é importante delimitar o escopo e a natureza da contribuição pretendida. Por óbvio, este artigo não tem como objetivo propor extensões teóricas ao arcabouço Novo-Keynesiano nem desenvolver uma nova regra de política monetária. O que se busca é investigar empiricamente o modelo NK aplicado no contexto institucional brasileiro entre 1999 e 2024, conforme apresentado na revisão da literatura (seções \ref{sec:regras_politica_estabilidade} e \ref{sec:arcabouco_teorico}) e discutido ao longo da disciplina, combinando fundamentação teórica, evidência empírica e análise de determinância do equilíbrio. O foco não é avaliar a optimalidade da política monetária brasileira, mas verificar se a reação estimada é compatível com a unicidade e estabilidade do equilíbrio no modelo Novo-Keynesiano.

\subsubsection{Objetivo Geral}
\label{subsubsec:objetivo_geral}

Temos como \textbf{objetivo geral do artigo avaliar empiricamente se a política monetária brasileira tem satisfeito o Princípio de Taylor e se os parâmetros estimados são compatíveis com a determinância (local) do equilíbrio no modelo Novo-Keynesiano.}

\subsubsection{Objetivos Específicos}
\label{subsubsec:objetivo_especificos}

\begin{itemize}
    \item Estimar diferentes especificações da Regra de Taylor.
    \item Verificar se os coeficientes estimados satisfazem $\hat{\phi}_\pi > 1$.
    \item Avaliar a condição de Bullard--Mitra para os parâmetros estimados.
    \item \textcolor{magenta}{ <-> Explorar implicações dinâmicas por meio de simulações em modelo NK <->}. % SE DER TEMPO
\end{itemize}

% -----------------------------------------------------------
\subsection{Estrutura do Artigo}
\label{subsec:estrutura_artigo}

\textbf{Além desta introdução, o artigo organiza-se em outras \underline{três partes}. A \underline{primeira parte}, servindo de revisão e síntese de conteúdo apresentado no curso de Macroeconomia II, reúne as seções \ref{sec:regras_politica_estabilidade}, \ref{sec:arcabouco_teorico}}: de forma contextual, a Seção~\ref{sec:regras_politica_estabilidade} revisa a literatura sobre regras monetárias e determinância, com ênfase em evidência internacional e para o Brasil, apresentando como as diferentes formulações da Regra de Taylor têm sido utilizadas e estimadas empiricamente; complementando a revisão teórica, a Seção~\ref{sec:arcabouco_teorico}, por sua vez, formaliza o arcabouço Novo-Keynesiano que fundamenta a análise, mostrando como essas mesmas regras operam dentro do modelo, em particular sua interação com a NKPC, a equação IS e a condição de Bullard--Mitra. \textbf{A \underline{segunda parte}, que nos permite aplicar e aprofundar os conhecimentos supracitados, compreende as seções \ref{sec:politica_monetaria_brasil_1999_2024}, \ref{sec:metodologia} e \ref{sec:resultados}}: a Seção~\ref{sec:politica_monetaria_brasil_1999_2024} apresenta a conjuntura do regime de metas brasileiro, descrevendo presidentes do Banco Central, regimes de política monetária e fatos estilizados que moldaram a economia brasileira entre 1999 e 2024 e motivam a estratificação e métodos de análise neste trabalho; a Seção~\ref{sec:metodologia} detalha dados e metodologia que serviram de base para responder a pergunta-problema e alcançar os objetivos do trabalho; a Seção~\ref{sec:resultados} apresenta os resultados empíricos, conforme modelagem proposta na seção anterior. \textbf{A \underline{terceira parte} contém a Seção~\ref{sec:consideracoes_finais}, com as conclusões e considerações finais.}

\textcolor{magenta}{VERIFICAR SE PRECISA MEXER AQUI DEPOIS DE TERMINAR O PAPER}

Vale pontuar que a análise empírica desenvolvida na Seção~\ref{sec:metodologia} parte do modelo novo-keynesiano apresentado na Seção~\ref{sec:arcabouco_teorico}, tomando as condições de determinância como referência teórica para a interpretação dos parâmetros estimados. As equações estimadas representam funções de reação reduzidas da política monetária, cujo objetivo não é testar a determinância diretamente, mas verificar se o comportamento empírico da autoridade monetária é compatível com regiões do espaço paramétrico associadas à unicidade do equilíbrio. De resto, a exposição detalhada dos fundamentos teóricos e das escolhas empíricas tem como objetivo não apenas a estimação dos parâmetros de interesse, mas também a explicitação do mecanismo de transmissão subjacente, conforme discutido ao longo do curso.


% ===========================================================
\section{Regras de Política e Estabilidade: Evidências Teóricas e Empíricas}
% ===========================================================
\label{sec:regras_politica_estabilidade}

A presente Seção~organiza a discussão de política monetária a partir das diferentes formulações da Regra de Taylor, que ajudam a entender como bancos centrais ajustam juros em função da inflação e da atividade. Em seguida, são apresentadas as contribuições sobre determinância e estabilidade, que delimitam quando essas respostas são suficientes para garantir um equilíbrio único no modelo Novo-Keynesiano. Evidências internacionais e brasileiras complementam esse panorama, mostrando como esses mecanismos se manifestaram em distintos regimes e períodos. Essa sequência, da formulação das regras às suas implicações teóricas e empíricas, busca oferecer o enquadramento necessário para interpretar os resultados estimados posteriormente, com base no arcabouço teórico, que os formaliza em maior profundidade na Seção~\ref{sec:arcabouco_teorico}.

% -----------------------------------------------------------
\subsection{Regras de Política Monetária}
\label{subsec:regras_politica_monetaria}

As contribuições de \citeonline{taylor1993}, que, juntamente com a Curva de Phillips Novo-Keynesiana e a equação IS dinâmica, compõem o núcleo da análise moderna de política monetária ao fornecerem o mecanismo de fechamento do modelo, podem ser compreendidas no contexto da transição metodológica produzida pela crítica de Lucas e pela consolidação dos modelos com expectativas racionais. Taylor enfatiza que a avaliação \textit{ex ante} de regras de política só se tornou factível quando a literatura reconheceu as limitações dos modelos tradicionais, bem como os avanços teóricos que demonstravam a superioridade de regras sistemáticas sobre a discricionariedade. É nesse contexto que ele identifica os principais fatores que motivaram a formulação de regras de política operacionalizáveis, afirmando que “\textit{the Lucas critique showing that traditional econometric policy evaluation was flawed, the recognition that rational expectations does not imply monetary policy ineffectiveness, and the finding that credibility has empirically significant benefits}” foram determinantes para esse avanço \cite[p.~195--196]{taylor1993}.

Essas motivações e alicerces teóricos dão sentido à formulação da Regra de Taylor: uma especificação simples e operacional que relaciona a taxa básica de juros a desvios da inflação e do hiato do produto, oferecendo uma resposta sistemática e previsível em contraste com o “\textit{feeling}” discricionário. Em sua análise histórica subsequente, \citeonline{taylor1999} mostrou que períodos em que a política monetária reagiu de forma insuficiente aos desvios da inflação, isto é, com coeficientes $\phi_{\pi}$ persistentemente inferiores à unidade, coincidiram com maior instabilidade macroeconômica, ao passo que períodos caracterizados por respostas mais agressivas estiveram associados a maior estabilidade. Assim, o conjunto de seus trabalhos estabelece não apenas uma regra prática, mas um arcabouço conceitual que vincula credibilidade, expectativas e determinância do equilíbrio, desempenhando papel central no fechamento dos modelos Novo-Keynesianos.

Aprofundando essa análise sob a ótica das expectativas racionais, \citeonline{clarida2000monetary} expandiram o arcabouço de Taylor ao estimar uma função de reação \textit{forward-looking} (quando a autoridade monetária reage às expectativas de inflação futura e não aos dados passados) para a economia dos EUA, dividindo a amostra nas eras pré-Volcker (1960--1979) e Volcker-Greenspan (1979--1996). Os autores demonstraram formalmente que a instabilidade macroeconômica do primeiro período decorreu de uma violação sistemática do Princípio de Taylor ($\phi_{\pi} < 1$), o que permitiu a emergência de equilíbrios indeterminados sujeitos a \textit{sunspots}\space (profecias autorrealizáveis). A contribuição crucial desse trabalho reside na incorporação explícita das expectativas de inflação futura na função de reação do Banco Central, estabelecendo que a estabilidade não depende apenas da resposta aos dados observados, mas da capacidade da autoridade monetária de ancorar as expectativas dos agentes através de uma postura agressiva ($\phi_{\pi} > 1$) diante de choques inflacionários previstos. Essa abordagem fundamenta diretamente a estratégia empírica do presente estudo, que busca testar se a condução da política monetária no Brasil, ao reagir às expectativas de inflação (Focus), cumpriu as condições de determinância exigidas para evitar a propagação de instabilidade não fundamental.

Uma síntese natural das contribuições anteriores encontra-se em \citeonline{woodford2003interest}, cuja formulação microfundamentada do modelo Novo-Keynesiano apresenta o arcabouço teórico dominante para a análise de política monetária. Woodford demonstrou que a eficácia das regras de política depende crucialmente do modo como elas ancoram expectativas, não somente de inflação futura, mas também do próprio estado da economia ao longo do tempo. Em seu tratamento normativo, a autoridade monetária minimiza a perda social derivada de fundamentos microeconômicos, o que leva a regras ótimas caracterizadas por forte resposta às expectativas de inflação e por um compromisso intertemporal capaz de eliminar equilíbrios indeterminados. Assim, embora a Regra de Taylor represente uma aproximação reduzida dessas regras ótimas, Woodford mostra que, de fato, políticas que reagem de maneira suficientemente agressiva à inflação (p. ex.: $\phi_{\pi}$ >1) são as que garantem a unicidade do equilíbrio e a estabilidade macroeconômica. Desse modo, a contribuição de Woodford consolida o elo conceitual entre microfundamentação, expectativas e determinância, oferecendo a justificativa teórica definitiva para o uso de regras de Taylor na avaliação empírica da política monetária.

% -----------------------------------------------------------
\subsection{Determinância e Estabilidade}
\label{subsec:determinancia_estabilidade}

A análise da eficácia das regras de política monetária em modelos dinâmicos exige, além da especificação da função de reação da autoridade monetária, a verificação das condições sob as quais o sistema econômico converge para uma trajetória estável e única. Nesse contexto, a contribuição de \citeonline{blanchard1980solution} estabeleceu o arcabouço matemático fundamental para a solução de modelos lineares com expectativas racionais, permitindo a verificação formal da estabilidade de regras como a de Taylor. 

Os autores demonstraram que a existência e a unicidade do equilíbrio dependem da relação entre o número de autovalores explosivos da matriz de transição do sistema e o número de variáveis não predeterminadas (\textit{forward-looking}). Especificamente, para que haja determinância (equilíbrio único e estável), o número de raízes instáveis, isto é, os componentes dinâmicos que levariam o sistema a uma trajetória explosiva, deve ser exatamente igual ao número de variáveis de salto, ou seja, aquelas que podem mudar de valor instantaneamente e de forma \enquote{livre} em resposta a novas informações ou choques, porque dependem inteiramente das expectativas futuras dos agentes, como no caso da inflação e o hiato do produto. Caso essa condição não seja satisfeita, o modelo pode apresentar indeterminação, permitindo múltiplos equilíbrios e flutuações guiadas por \textit{sunspots}, ou inexistência de solução estável (trajetórias explosivas).

Avançando sobre essa base, \citeonline{bullard2002learning} investigaram a estabilidade do equilíbrio Novo-Keynesiano sob a hipótese de que os agentes não possuem conhecimento perfeito da estrutura da economia, mas formam expectativas através de processos de aprendizagem adaptativa (\textit{adaptive learning}). Ao analisarem diferentes especificações da regra de juros, os autores derivaram condições para determinância e para estabilidade sob aprendizagem (\textit{E-stability}). No caso do modelo NK canônico com regra de juros linear reagindo à inflação e ao hiato do produto, uma condição amplamente utilizada para garantir estabilidade do equilíbrio (e, sob hipóteses usuais, determinância local) pode ser expressa pela desigualdade abaixo:

\begin{equation}
\kappa(\phi_\pi - 1) + (1-\beta)\phi_y > 0
\end{equation}

Essa expressão formaliza analiticamente o Princípio de Taylor no contexto de modelos microfundamentados: para garantir a determinância e a convergência da aprendizagem dos agentes para o equilíbrio de expectativas racionais, a autoridade monetária deve reagir à inflação de forma agressiva ($\phi_\pi > 1$), ajustada pela resposta da autoridade monetária ao hiato do produto ($\phi_y$) e pelos parâmetros estruturais da economia, como o fator de desconto ($\beta$) e a inclinação da Curva de Phillips ($\kappa$). 

Posto isso, o trabalho de Bullard e Mitra é central para este estudo, pois conecta a estimação empírica dos coeficientes $\phi_\pi$ e $\phi_y$ à capacidade efetiva do Banco Central de ancorar expectativas e evitar a propagação de instabilidade não fundamental. No contexto empírico deste artigo, essa condição será utilizada para avaliar se os coeficientes 
$\hat{\phi}_{\pi}$ e $\hat{\phi}_{y}$, estimados a partir de diferentes especificações da Regra de Taylor, 
são compatíveis com a determinância do equilíbrio NK.

% -----------------------------------------------------------
\subsection{Evidências Internacionais}
\label{subsec:evidencias_internacionais}

A literatura empírica internacional, ao aplicar o arcabouço de regras de Taylor em economias avançadas, consolidou o entendimento de que a estabilidade macroeconômica depende crucialmente do cumprimento do Princípio de Taylor. Trabalhos como o de \citeonline{smets2007shocks}, utilizando um modelo DSGE estimado por métodos bayesianos para a economia norte-americana no período pós--1984 (``Grande Moderação''), encontraram coeficientes de reação à inflação significativamente superiores à unidade (em torno de 2,0 no longo prazo), corroborando a hipótese de que uma política monetária ativa foi determinante para a ancoragem das expectativas e a redução da volatilidade do produto. Essa evidência contrasta com os resultados para o período da ``Grande Inflação'' (pré--1979), onde estimativas de \citeonline{lubik2004testing} indicam que a política monetária situava-se na região de indeterminação, permitindo flutuações guiadas por profecias autorrealizáveis (\textit{sunspots}).

Complementarmente, a aplicação de regras de Taylor em outros contextos regionais reforça, com nuances importantes, o papel central da resposta sistemática à inflação, ao mesmo tempo em que amplia a agenda para incluir também riscos de instabilidade financeira. \citeonline{moura2010can} mostram que, em economias latino-americanas como Chile, México e Brasil, estimativas de regras de Taylor frequentemente apontam para $\phi_\pi > 1$ em regimes de metas de inflação mais consolidados, mas revelam forte heterogeneidade temporal e institucional, sugerindo que a aderência ao Princípio de Taylor é mais frágil em ambientes de maior vulnerabilidade fiscal e cambial. Em paralelo, \citeonline{kahn2010taylor} e \citeonline{kafer2014taylor} destacam que desvios persistentes em relação à regra de Taylor podem estar associados à formação de desequilíbrios financeiros, defendendo a incorporação de variáveis financeiras nas funções de reação, em particular no contexto do Federal Reserve e da área do euro. Esses resultados situam a problemática deste artigo em um quadro mais amplo: ainda que o foco aqui recaia sobre determinância e ancoragem de expectativas em modelos Novo-Keynesianos, a força da reação da política monetária à inflação, medida por $\phi_\pi$, também dialoga com o debate contemporâneo sobre a compatibilização entre estabilidade de preços, estabilidade financeira e credibilidade em economias avançadas e emergentes.

Além da magnitude da resposta à inflação, a literatura também enfatiza a importância da informação utilizada pelo Banco Central. \citeonline{orphanides2001monetary} demonstrou que avaliações de regras de política baseadas em dados revisados (\textit{ex post}) podem levar a conclusões enganosas sobre a conduta da autoridade monetária, uma vez que as decisões são tomadas com base em dados em tempo real. Seus resultados indicam que especificações \textit{forward-looking}, que incorporam as previsões disponíveis no momento da decisão, descrevem com maior precisão o comportamento histórico do \textit{Federal Reserve}. Essa constatação reforça a relevância de utilizar medidas de expectativas de mercado, como as do Boletim Focus no caso brasileiro, para capturar adequadamente o conjunto de informações que condiciona a reação da política monetária e a determinação do equilíbrio.

% -----------------------------------------------------------
\subsection{Evidências para o Brasil}
\label{subsec:evidencias_brasil}

Ecoando a consolidação internacional do arcabouço de regras de Taylor, a literatura empírica brasileira também produziu um vasto corpo de evidências sobre a condução da política monetária sob o regime de metas, buscando verificar se o Banco Central do Brasil aderiu aos princípios de estabilização consagrados nos modelos Novo-Keynesianos. Desde a implementação do regime, discutida normativamente por \citeonline{giambiagi2002metas} como um mecanismo de disciplina intertemporal para a estabilidade de preços, artigos como o de \citeonline{minella2003inflation} documentaram uma reação vigorosa da taxa Selic aos desvios das expectativas de inflação ($\phi_\pi > 1$). Essa evidência inicial sugeriu que, a exemplo das economias avançadas durante a ``Grande Moderação'', a autoridade monetária brasileira internalizou rapidamente a necessidade de uma postura ativa para construir credibilidade e ancorar as expectativas nominais, achado corroborado em estudos mais recentes, como \citeonline{vanzelotti2023estimando}, para amostras ampliadas.

Entretanto, diferentemente do caso norte-americano ou europeu, a aplicação da Regra de Taylor ao contexto brasileiro exige a consideração de idiossincrasias estruturais de uma economia emergente. \citeonline{barbosa2016taxa}, por exemplo, argumentam que a postura da política monetária deve ser avaliada à luz de uma taxa de juros natural variável e historicamente elevada, influenciada pelo risco-país e pelas condições externas. Adicionalmente, \citeonline{aragon2010nonlinearities} demonstraram que a função de reação do Banco Central pode apresentar não linearidades, caracterizada por preferências assimétricas que toleram menos desvios positivos da inflação do que quedas no produto. Essas nuances são reforçadas por \citeonline{areosa2007inflation}, que destacam como a forte inércia inflacionária e o repasse cambial no Brasil impõem restrições mais severas à calibração da regra de juros do que as observadas em modelos de economia fechada.

Apesar desses avanços robustos na caracterização das especificidades locais, observa-se uma lacuna importante quando se compara a literatura nacional à fronteira da pesquisa internacional descrita na Seção~anterior: a conexão explícita entre as estimativas empíricas e as condições teóricas de estabilidade dinâmica. Enquanto autores como \citeonline{lubik2004testing} testam formalmente a determinância do equilíbrio para os EUA, poucos estudos no Brasil transpõem a estimação dos coeficientes $\phi_\pi$ e $\phi_y$ para uma avaliação rigorosa da condição de \citeonline{bullard2002learning}. Em particular, carece-se de uma análise sistemática que segmente os mandatos presidenciais do Banco Central para verificar se, em cada gestão específica, a combinação de parâmetros satisfez as condições de unicidade do equilíbrio, evitando a vulnerabilidade da economia a flutuações guiadas por profecias autorrealizáveis. É justamente nessa interseção entre a evidência econométrica dos regimes brasileiros e a teoria de estabilidade sob aprendizagem que este artigo se insere.

% ***********************************************************

% ===========================================================
\section{Arcabouço Formal do Modelo Novo-Keynesiano}
% ===========================================================
\label{sec:arcabouco_teorico}

Esta Seção~apresenta o modelo Novo-Keynesiano, padrão que fundamenta a análise empírica, estruturando as relações dinâmicas entre inflação, produto e taxa de juros. Inicialmente, descreve-se a oferta agregada da economia por meio da Curva de Phillips Novo-Keynesiana (NKPC), derivada de fundamentos microeconômicos de rigidez de preços, seguida pela demanda agregada representada pela equação IS dinâmica. Com o sistema econômico definido, discutem-se as diferentes especificações da Regra de Taylor que fecham o modelo, detalhando como a reação do Banco Central a choques condiciona a estabilidade macroeconômica. Por fim, formalizam-se as condições de determinância de \citeonline{bullard2002learning}, estabelecendo o critério teórico exato para avaliar se os parâmetros estimados na Seção empírica são compatíveis com a unicidade do equilíbrio ou se permitem instabilidade.

% -----------------------------------------------------------
\subsection{Curva de Phillips Novo-Keynesiana (NKPC)}
\label{subsec:nkpc}

No modelo Novo-Keynesiano, a dinâmica da oferta agregada é descrita pela Curva de Phillips Novo-Keynesiana (NKPC), que relaciona a inflação corrente às expectativas de inflação futura e ao hiato do produto. Essa relação é derivada a partir de fundamentos microeconômicos, resultando do problema de precificação intertemporal das firmas em um ambiente com concorrência monopolística e rigidez nominal de preços.

Seguindo \citeonline{gali2015monetary}, assume-se rigidez de preços à la \citeonline{calvo1983staggered}: em cada período, apenas uma fração $(1-\theta)$ das firmas pode reajustar seus preços de forma ótima, enquanto a fração $\theta$ mantém os preços previamente fixados. O parâmetro $\theta \in [0,1)$ mede o grau de rigidez nominal da economia, determinando a duração esperada dos contratos de preços.

As firmas que reajustam seus preços escolhem um preço ótimo maximizando o valor presente esperado dos lucros, levando em conta que esse preço poderá vigorar por vários períodos. A agregação dessas decisões e a log-linearização em torno de um estado estacionário de inflação nula conduzem à seguinte expressão para a dinâmica inflacionária:

\begin{equation}   
\pi_t = \beta E_t \pi_{t+1} + \kappa \tilde{y}_t,
\label{eq:nkpc}
\end{equation}

\noindent onde $\pi_t$ é a inflação no período $t$, $E_t \pi_{t+1}$ representa a expectativa racional da inflação futura e $\tilde{y}_t \equiv y_t - y_t^n$ denota o hiato do produto, definido como a diferença entre o produto efetivo e o produto natural sob preços flexíveis.

A NKPC evidencia dois mecanismos centrais. Primeiro, a inflação é um fenômeno essencialmente prospectivo: como os preços permanecem fixos por múltiplos períodos, as expectativas de inflação futura afetam diretamente a inflação corrente. Segundo, o hiato do produto influencia a inflação por meio de seu impacto sobre os custos marginais reais, refletindo pressões de demanda na economia.

O coeficiente $\kappa$ mede a sensibilidade da inflação ao ciclo econômico e é determinado estruturalmente por:
\begin{equation}
\kappa \equiv \lambda \left(\frac{\sigma(1-\alpha)+\varphi+\alpha}{1-\alpha}\right)
\qquad
\lambda \equiv \frac{(1-\theta)(1-\beta\theta)}{\theta}
\end{equation}
onde $\sigma$ é o inverso da elasticidade intertemporal de substituição do consumo, $\varphi$ é o inverso da elasticidade de Frisch da oferta de trabalho e $\alpha$ mede a participação do capital na função de produção (com rendimentos constantes de escala), de modo que $1-\alpha$ corresponde à participação do trabalho.

Quanto maior a rigidez nominal (isto é, quanto mais $\theta$ se aproxima de 1), menor é $\kappa$, reduzindo a resposta da inflação às flutuações do hiato do produto. Essa característica tem implicações diretas para a condução da política monetária: quanto menor $\kappa$, maior o papel das expectativas na determinação da inflação e maior a necessidade de regras de política que assegurem a estabilidade do equilíbrio macroeconômico, tema explorado na Seção seguinte.

% -----------------------------------------------------------
\subsection{Equação IS Dinâmica}
\label{subsec:is_dinamica}

A demanda agregada do modelo é descrita pela Equação IS Dinâmica, que relaciona o hiato do produto à taxa de juros real e às expectativas de atividade futura. Essa relação é obtida diretamente da log-linearização da equação de Euler do consumidor representativo, combinada com a condição de equilíbrio no mercado de bens, conforme detalhado em \citeonline{gali2015monetary}.

Formalmente, a maximização da utilidade intertemporal das famílias implica que o consumo corrente depende negativamente da taxa de juros real ex-ante e positivamente do consumo esperado para o futuro (suavização do consumo). Reescrevendo essa condição em termos de desvios do estado estacionário e expressando o produto em função de seu nível natural, chega-se à seguinte expressão para o hiato do produto:

\begin{equation}
\tilde{y}_t = E_t \tilde{y}_{t+1} - \frac{1}{\sigma} \big( i_t - E_t \pi_{t+1} - r_t^n \big)
\label{eq:dis}
\end{equation}

\noindent onde $\tilde{y}_t$ é o hiato do produto, $i_t$ é a taxa de juros nominal fixada pelo Banco Central, $E_t \pi_{t+1}$ é a inflação esperada e $\sigma$ representa o inverso da elasticidade de substituição intertemporal do consumo (ou coeficiente de aversão relativa ao risco).

A variável $r_t^n$ denota a \textbf{taxa de juros natural} (ou Wickselliana), definida por \citeonline{woodford2005interest} como a taxa de juros real de equilíbrio que prevaleceria na economia se os preços fossem totalmente flexíveis. No modelo básico, $r_t^n$ é determinada inteiramente por choques reais (como produtividade ou preferências) e independe da política monetária.

A equação \eqref{eq:dis} difere fundamentalmente da curva IS tradicional em dois aspectos:

\begin{itemize}
    \item \textbf{Natureza \textit{Forward-Looking}:} O hiato corrente $\tilde{y}_t$ depende não apenas dos juros hoje, mas da expectativa de hiato futuro $E_t \tilde{y}_{t+1}$. Isso significa que a política monetária afeta a economia tanto pela taxa de juros corrente quanto pela sua trajetória esperada (canal das expectativas).
    
    \item \textbf{O Juro Natural como Referência:} O termo relevante para a demanda agregada é o \textit{hiato da taxa de juros} $(i_t - E_t \pi_{t+1} - r_t^n)$. A política monetária é expansionista apenas se a taxa real de juros ($i_t - E_t \pi_{t+1}$) for fixada abaixo da taxa natural $r_t^n$. Caso o Banco Central mantenha a taxa real igual à taxa natural em todos os períodos, o hiato do produto será nulo ($\tilde{y}_t = 0$) e, na ausência de choques de custo, a inflação permanecerá estável, isolando a economia de choques de demanda ineficientes.
\end{itemize}

Essa estrutura teórica destaca que a estabilização macroeconômica exige que o Banco Central monitore e acompanhe as flutuações de $r_t^n$, ajustando a taxa nominal $i_t$ para evitar desvios persistentes em relação ao juro natural. A forma como a autoridade monetária operacionaliza esses ajustes em resposta à inflação e ao produto é descrita pelas Regras de Taylor, detalhadas a seguir.

Assim sendo, esse aspecto decorre do fato de que as equações centrais do modelo novo-keynesiano são intrinsecamente prospectivas. A inflação corrente depende das expectativas de inflação futura, enquanto o hiato do produto é determinado pela expectativa do hiato à frente e pela taxa de juros real esperada. Assim, a política monetária afeta o equilíbrio macroeconômico não apenas por meio de seus efeitos contemporâneos sobre a taxa de juros nominal, mas sobretudo pela forma como influencia a formação de expectativas privadas acerca da trajetória futura da economia.

% -----------------------------------------------------------
\subsection{Regras de Taylor}
\label{subsec:regra_taylor}

A condução da política monetária no modelo Novo-Keynesiano (NK) é tipicamente descrita por meio de uma regra de juros que especifica como o banco central ajusta a taxa de juros nominal em resposta a desvios da inflação e do produto em relação aos seus níveis-alvo. \citeonline{taylor1993} propôs uma regra simples que captura o comportamento sistemático da política monetária nos Estados Unidos, estabelecendo que a taxa de juros nominal deve responder positivamente aos desvios da inflação em relação à meta e ao hiato do produto. Este \textit{framework} foi posteriormente generalizado por \citeonline{clarida2000monetary} para incorporar expectativas \textit{forward-looking} e suavização da taxa de juros, alinhando a regra de política monetária à estrutura intertemporal do modelo NK.

\subsubsection{Princípio de Taylor}
\label{subsubsec:principio_taylor}
\label{subsec:principio_taylor}

O Princípio de Taylor estabelece uma condição fundamental para a estabilidade macroeconômica sob uma regra de juros: o banco central deve ajustar a taxa de juros nominal mais que proporcionalmente a variações na inflação esperada, de modo que a taxa de juros real se eleve quando há pressões inflacionárias. Formalmente, considerando a regra de juros:

\begin{equation}
i_t = \rho + \phi_\pi E_t\pi_{t+1} + \phi_y \tilde{y}_t,
\label{eq:principio_taylor}
\end{equation}
o Princípio de Taylor requer que \(\phi_\pi > 1\). Em sua forma mais simples, esta condição assegura a determinação do equilíbrio e evita flutuações auto-realizáveis (\textit{sunspot fluctuations}), ao garantir que aumentos na inflação esperada sejam acompanhados por elevações mais que proporcionais da taxa de juros nominal.

A intuição por trás deste princípio é direta: quando \(\phi_\pi < 1\), um aumento na inflação esperada leva o banco central a elevar a taxa nominal em magnitude insuficiente, resultando em queda da taxa real de juros \(r_t = i_t - E_t\pi_{t+1}\). Esta redução estimula a demanda agregada, validando e amplificando as expectativas inflacionárias iniciais. Em contraste, quando \(\phi_\pi > 1\), o banco central eleva a taxa real de juros, contraindo a demanda e contribuindo para a estabilização da inflação. Em modelos mais gerais, com resposta explícita ao hiato do produto e regras antecipativas, a condição de determinância passa a depender de combinações entre \(\phi_\pi\), \(\phi_y\), \(\kappa\) e \(\beta\), conforme discutido por \citeonline{bullard2002learning}. 

Cabe notar que, em modelos novo-keynesianos com expectativas racionais, a introdução de uma regra de política monetária não esgota, por si só, a caracterização do equilíbrio. Embora a regra de Taylor forneça um mecanismo sistemático de reação da autoridade monetária a desvios da inflação e do produto, sua presença não garante, a priori, que o sistema dinâmico resultante admita uma única trajetória de equilíbrio consistente com as expectativas dos agentes. Em particular, em ambientes \textit{forward-looking}, regras de juros insuficientemente reativas podem ser compatíveis com múltiplos equilíbrios racionais, fenômeno associado à possibilidade de trajetórias autorrealizáveis para inflação e produto.


\subsubsection{Especificações da Regra de Taylor}
\label{subsubsec:especificacoes_regra_taylor}

Consideramos três especificações principais da regra de juros, que diferem quanto ao horizonte de previsão e ao grau de inércia da taxa de juros. \footnote{A notação das equações declaradas a seguir foi levemente modificada para manter a consistência com o que foi escrito nas diferentes partes deste artigo, mas sem alteração dos sentidos originais. Em suma, a equação~\ref{eq:taylor_backward} em~\ref{par:regra_taylor_backward} segue a equação (1) de \citeonline{taylor1993}; as equações \ref{eq:taylor_forward} em~\ref{par:regra_taylor_forward}, e \ref{eq:hibrida_real_alvo} em \ref{par:taylor_hibrida} seguem respectivamente as equações (1) e (2) de \citeonline{clarida2000monetary}.}

\paragraph{Regra \textit{Backward-Looking}}
\label{par:regra_taylor_backward}
\label{subsec:regra_taylor_backward}

A formulação original de \citeonline{taylor1993} especifica a taxa de juros como função da inflação corrente e do hiato do produto corrente:

\begin{equation}
i_t = \rho + \phi_\pi \pi_t + \phi_y \tilde{y}_t + \varepsilon_t,
\label{eq:taylor_backward}
\end{equation}
onde \(\rho\) representa o intercepto da regra de política monetária, associado ao nível médio da taxa real de juros no estado estacionário (\(\rho = 2\%\), calibrado); \(\phi_\pi\) representa o coeficiente de reação da taxa de juros à inflação corrente (\(\pi_t\)); \(\phi_y\) é o coeficiente de reação ao hiato do produto (\(\tilde{y}_t\)), que mede o desvio do produto em relação ao seu nível potencial; e \(\varepsilon_t\) é um choque de política monetária, capturando fatores não sistemáticos ou imprevistos que afetam a decisão do Banco Central. Assim, \(i_t\) é a taxa de juros nominal definida pela autoridade monetária no período \(t\), \(\pi_t\) é a inflação observada, e \(\tilde{y}_t\) representa o grau de ociosidade ou pressão da atividade econômica sobre os preços). O autor propôs valores específicos para os coeficientes -- \(\rho = 2\%\), \(\phi_\pi = 1.5\) e \(\phi_y = 0.5\) -- calibrados para aproximar a condução da política monetária do Federal Reserve no período 1987--1992 \cite[p.~202]{taylor1993}. Neste ponto, cabe refoçar que a regra é expressa em níveis,
explicitando separadamente a taxa real de juros de equilíbrio e a meta de inflação. Em particular, o autor assume uma taxa real neutra de 2\% ao ano e uma meta implícita de inflação de 2\%, o que permite reescrever a regra em forma reduzida como uma função linear da inflação corrente e do hiato do produto, conforme a equação \eqref{eq:taylor_backward}.


\paragraph{Regra \textit{Forward-Looking}}
\label{par:regra_taylor_forward}
\label{subsec:regra_taylor_forward}

De forma alternativa, \citeonline{clarida2000monetary} argumentam que a política monetária deve ser descrita por uma regra \textit{forward-looking}, na qual o banco central responde não à inflação e ao hiato do produto correntes, mas às suas expectativas:

\begin{equation}
i_t^* = \rho + \phi_\pi E_t[\pi_{t+h_\pi}] + \phi_y E_t[\tilde{y}_{t+h_y}] + \varepsilon_t,
\label{eq:taylor_forward}
\end{equation}
onde \(i_t^*\) denota a taxa de juros alvo definida pelo Banco Central no período \(t\); \(\rho\) é o intercepto, associado ao nível médio da taxa real de juros no estado estacionário; \(\phi_\pi\) é o coeficiente de reação da taxa de juros à expectativa de inflação futura; \(E_t[\pi_{t+h_\pi}]\) representa a expectativa de inflação para o horizonte \(h_\pi\); \(\phi_y\) é o coeficiente de reação ao hiato do produto esperado; \(E_t[\tilde{y}_{t+h_y}]\) é a expectativa do hiato do produto para o horizonte \(h_y\); e \(\varepsilon_t\) é um choque de política monetária, capturando fatores não sistemáticos. A escolha de \(h\) pode variar conforme o contexto institucional e a disponibilidade de informações, mas frequentemente assume-se \(h=1\) ou \(h=2\) trimestres, refletindo o horizonte típico de decisão dos bancos centrais.

\paragraph{Regra Híbrida com Suavização (Inércia)}
\label{par:taylor_hibrida}

Evidências empíricas sugerem que bancos centrais ajustam a taxa básica de juros de forma gradual, evitando variações abruptas que poderiam amplificar a volatilidade macroeconômica e financeira \cite{wagner2000central}. Para capturar esse comportamento, \citeonline{clarida2000monetary} propõem uma regra de política monetária com ajuste parcial, na qual a taxa nominal observada converge gradualmente para uma taxa-alvo implícita. Formalmente, define-se a taxa nominal alvo ($i_t^*$) como:

\begin{equation}
i_t^* = \rho + \phi_\pi E_t[\pi_{t+h_\pi}] + \phi_y E_t[\tilde{y}_{t+h_y}],
\label{eq:hibrida_alvo}
\end{equation}
onde \(i_t^*\) é a taxa de juros alvo definida pelo Banco Central; \(\rho\) representa o intercepto, associado ao nível médio da taxa real de juros no estado estacionário; \(\phi_\pi\) é o coeficiente de reação à expectativa de inflação futura; \(E_t[\pi_{t+h_\pi}]\) indica a expectativa de inflação para o horizonte \(h_\pi\); \(\phi_y\) é o coeficiente de reação ao hiato do produto esperado; \(E_t[\tilde{y}_{t+h_y}]\) corresponde à expectativa do hiato do produto para o horizonte \(h_y\), enquanto a taxa efetivamente praticada pela autoridade monetária obedece à dinâmica

\begin{equation}
i_t = \rho_i\, i_{t-1} + (1-\rho_i)i_t^* + \varepsilon_t,
\label{eq:hibrida_dinamica}
\end{equation}
em que \(\rho_i \in [0,1)\) mede o grau de suavização da política monetária. Valores elevados de \(\rho_i\), tipicamente estimados entre 0{,}7 e 0{,}9, indicam forte persistência nos ajustes da taxa de juros, compatível com estratégias de gradualismo e gerenciamento de expectativas em um ambiente \textit{forward-looking}. Subtraindo a inflação esperada da taxa-alvo nominal, obtém-se a taxa real de juros alvo implícita ($r_t^*$):

\begin{equation}
r_t^* \equiv i_t^* - E_t[\pi_{t+h}] = \rho + (\phi_\pi - 1)E_t[\pi_{t+h_\pi}] + \phi_y E_t[\tilde{y}_{t+h_y}],
\label{eq:hibrida_real_alvo}
\end{equation}
a qual não deve ser confundida com a taxa natural de juros no sentido wickselliano (ver~\ref{subsec:is_dinamica}), mas sim interpretada como a taxa real induzida pela reação sistemática da política monetária.

Essa expressão torna explícito o Princípio de Taylor: quando \(\phi_\pi > 1\), aumentos na inflação esperada elevam a taxa real de juros, produzindo um efeito contracionista compatível com a estabilização macroeconômica; quando \(\phi_\pi < 1\), a política monetária torna-se acomodatícia, com queda da taxa real diante de pressões inflacionárias. No contexto do modelo NK, essa regra desempenha o papel de fechamento do sistema dinâmico formado pela Curva de Phillips Novo-Keynesiana e pela equação IS, sendo sua capacidade de assegurar um equilíbrio localmente único e estável condicionada aos parâmetros estimados e à estrutura expectacional do modelo. Ademais, a presença de expectativas racionais e o uso de variáveis (ou expectativas) potencialmente endógenas ao conjunto de informação dos agentes motivam a estimação dessa regra por meio de métodos baseados em momentos, como o GMM, tema explorado nas seções subsequentes ao tratar das condições de determinância à luz dos critérios de \citeonline{blanchard1980solution} e de \citeonline{bullard2002learning}.

\paragraph{Resumo das Especificações da Regra de Taylor}

A análise empírica considera três classes de regras de política monetária, amplamente utilizadas na literatura Novo-Keynesiana. A especificação \textit{\textbf{backward-looking}} modela a taxa de juros nominal como resposta \textbf{contemporânea} à \underline{inflação observada} e ao \underline{hiato do produto}, capturando um comportamento reativo baseado em informações observadas. A versão \textit{\textbf{forward-looking}}, por sua vez, incorpora \textbf{expectativas racionais}, fazendo a autoridade monetária reagir às \underline{projeções futuras} de inflação e atividade econômica. 

A regra \textit{\textbf{híbrida}} com suavização generaliza essas formulações ao introduzir \textbf{inércia} na taxa de juros, por meio de um mecanismo de \textbf{ajuste parcial} em torno de uma \underline{taxa-alvo implícita}. Nessa estrutura, a regra \textit{forward-looking} define o alvo da política monetária, enquanto o parâmetro de suavização governa a velocidade de convergência da taxa efetiva a esse alvo. Tal especificação permite reconciliar o caráter prospectivo da política monetária com o gradualismo observado empiricamente, sem alterar o conteúdo informacional do Princípio de Taylor, que permanece associado à reação da taxa-alvo à inflação esperada \cite{clarida2000monetary}. 

% -----------------------------------------------------------
\subsection{Condições de Determinância}
\label{subsec:condicoes_determinancia}

A existência de um equilíbrio único e estável no modelo Novo-Keynesiano depende crucialmente da regra de política monetária adotada e de sua capacidade de ancorar as expectativas dos agentes. Formalmente, a dinâmica do sistema linearizado pode ser representada na forma canônica de \citeonline{blanchard1980solution}, onde a unicidade do equilíbrio exige que o número de autovalores instáveis (módulo superior a um) da matriz de transição seja igual ao número de variáveis não-predeterminadas (\textit{forward-looking}), como a inflação e o hiato do produto.

Para o modelo NK linearizado sob uma regra de Taylor que responde a variáveis contemporâneas, uma condição padrão para determinância local e estabilidade sob aprendizagem, conforme \citeonline{bullard2002learning}, é dada por:

\begin{equation}
\kappa(\phi_\pi - 1) + (1-\beta)\phi_y > 0
\end{equation}

Nesta inequação, derivada da análise dos autovalores da matriz $A_T$ do sistema dinâmico, $\kappa$ representa a inclinação da Curva de Phillips Novo-Keynesiana e $\beta$ o fator de desconto intertemporal. A condição estabelece um \textit{trade-off} entre a resposta à inflação ($\phi_\pi$) e ao hiato do produto ($\phi_y$). O Princípio de Taylor, caracterizado por $\phi_\pi > 1$, é uma condição suficiente para a determinância sempre que $\phi_y \geq 0$, garantindo que a taxa de juros real suba diante de pressões inflacionárias, o que contrai a demanda agregada e estabiliza os preços.

Além de assegurar a unicidade sob expectativas racionais, \citeonline{bullard2002learning} e \citeonline{christiano2018does} argumentam que essa mesma condição é fundamental para a estabilidade sob aprendizagem (\textit{E-stability}). Em modelos onde os agentes não possuem conhecimento perfeito da estrutura da economia e formam expectativas via algoritmos de aprendizagem adaptativa, o cumprimento dessa condição evita trajetórias explosivas e garante a convergência para o Equilíbrio de Expectativas Racionais (REE). A violação dessa regra, por outro lado, abre espaço para a indeterminação real e múltiplos equilíbrios, incluindo trajetórias de \textit{sunspots} onde choques não fundamentais afetam a economia real.

% ***********************************************************

% ===========================================================
\section{Política Monetária no Brasil entre 1999 e 2024}
% ===========================================================
\label{sec:politica_monetaria_brasil_1999_2024}

Nesta seção, apresentamos uma análise descritiva da condução da política monetária no Brasil desde a adoção do regime de metas de inflação em 1999 até o período recente. O objetivo é organizar e interpretar os principais fatos estilizados da economia brasileira à luz do arcabouço teórico discutido nas seções anteriores, sem recorrer, neste momento, à formalização econométrica completa.

Para esse fim, utilizamos medidas e procedimentos amplamente empregados na literatura macroeconômica aplicada, como estimativas do hiato do produto, filtros de extração do componente cíclico e indicadores de expectativas, de forma essencialmente descritiva. Os detalhes técnicos associados à construção dessas variáveis, às escolhas metodológicas e aos testes formais são apresentados de maneira sistemática na seção seguinte, dedicada à metodologia empírica.

% -----------------------------------------------------------
\subsection{Regime de Metas de Inflação}
\label{subsec:regime_metas_inflacao}

O regime de metas de inflação, adotado no Brasil oficialmente a partir de janeiro de 1999 após o abandono do regime de bandas cambiais, com o objetivo de restabelecer uma âncora nominal crível para a política monetária, e, no formato avaliado neste artigo, vigente até dezembro de 1999 \cite{brasil1999decreto}, seguiu a experiência internacional consolidada ao longo da década de 1990, na qual a estabilidade de preços passou a ser explicitamente definida como objetivo primário da política monetária. Nesse arcabouço, a autoridade monetária anuncia uma meta numérica para a inflação em horizonte relevante, utilizando a taxa de juros de curto prazo como principal instrumento para ancorar as expectativas dos agentes econômicos \cite{bernanke1997inflation,hammond2012state}.

Do ponto de vista institucional, o regime brasileiro é caracterizado por uma meta de inflação definida pelo Conselho Monetário Nacional (CMN), acompanhada de bandas de tolerância, enquanto a condução operacional da política monetária cabe ao Banco Central do Brasil, por meio do Comitê de Política Monetária (Copom). A sistemática do regime enfatiza transparência, comunicação e responsabilização, com a divulgação regular de relatórios, atas e projeções, em linha com as práticas observadas em economias que adotam metas de inflação. Conforme destacado por \citeonline{bernanke1997inflation}, o regime deve ser interpretado como um \textit{framework} de política monetária, e não como uma regra mecânica, preservando flexibilidade para acomodar choques de curto prazo sem comprometer o objetivo de estabilidade de preços no médio prazo.

Embora o arcabouço formal do regime de metas de inflação tenha permanecido inalterado desde sua adoção, a literatura sugere que a condução da política monetária pode refletir variações ao longo do tempo na ênfase atribuída à inflação, ao hiato do produto e à estabilidade financeira, especialmente em economias emergentes sujeitas a choques externos e restrições macroeconômicas específicas \cite{neves2008regime,arestis2009nova}. Essas possíveis variações não configuram rupturas institucionais do regime, mas indicam mudanças na função de reação implícita da autoridade monetária, aspecto que motiva a análise empírica desenvolvida nas seções subsequentes.

% -----------------------------------------------------------
\subsection{Evolução Macroeconômica}
\label{subsec:evolucao_macroeconomica}

O período 1999–2024, adotado como critério neste trabalho, marca a consolidação do tripé macroeconômico, adotado em resposta ao esgotamento da âncora cambial do Plano Real. Segundo \citeonline{bogdanski2000implementing}, a crise de confiança de 1999 forçou a transição para o câmbio flutuante e exigiu uma nova âncora nominal para as expectativas, resultando na implementação do Regime de Metas para a Inflação. Esse arranjo institucional pautou a reação do Banco Central aos choques subsequentes, estabelecendo a estabilidade de preços como objetivo primário sob um ambiente de maior flexibilidade cambial.

Em consonância com o objeto, problema e objetivos deste trabalho, a análise descritiva por períodos associados às diferentes presidências do Banco Central do Brasil é particularmente relevante em uma economia emergente, caracterizada por elevada exposição a choques externos e recorrentes episódios de incerteza macroeconômica. Evidências recentes, como as de \citeonline{barros2023geopolitical}, indicam que choques de risco geopolítico internacional afetam de forma significativa a atividade econômica, a inflação, a taxa de juros e os prêmios de risco no Brasil, reforçando a importância de considerar a possibilidade de respostas diferenciadas da política monetária ao longo do tempo.

\subsection{Ciclos de Gestão e Choques Estruturais}
\label{subsec:ciclos_gestao_choques_estruturais}

A condução da política monetária ao longo desse intervalo foi marcada por cinco administrações principais no Banco Central, cada uma operando sob desafios distintos impostos pela conjuntura governamental e econômica. A gestão de \textbf{Armínio Fraga} (\emph{mar/1999--dez/2002}), durante o segundo governo de Fernando Henrique Cardoso, foi responsável pela implementação e consolidação inicial do regime de metas de inflação. Seguiu-se a longa administração de \textbf{Henrique Meirelles} (\emph{jan/2003--dez/2010}), abrangendo os dois mandatos do governo Lula, período caracterizado pelo acúmulo de reservas internacionais e pelo enfrentamento da Crise Financeira Global de 2008. \textbf{Alexandre Tombini} (\emph{jan/2011--jun/2016}) presidiu a autarquia durante o governo Dilma Rousseff e o início da crise econômica doméstica. \textbf{Ilan Goldfajn} (\emph{jun/2016--fev/2019}), nomeado pelo governo Temer, conduziu o processo de reancoragem das expectativas inflacionárias e o ciclo de flexibilização monetária no pós-recessão. Por fim, \textbf{Roberto Campos Neto} (\emph{fev/2019--dez/2024}), atuando sob os governos Bolsonaro e no início do terceiro mandato de Lula, enfrentou os choques inflacionários globais decorrentes da pandemia de Covid-19 e liderou a transição para a autonomia formal do Banco Central.

Para a identificação dos episódios de ruptura, adotou-se um critério duplo: a cronologia oficial de recessões estabelecida pelo Comitê de Datação de Ciclos Econômicos \cite{codace2020} para a delimitação temporal, combinada à caracterização da natureza dos choques predominante na literatura especializada, exemplificada por \citeonline{barbosa2017}. Sob essa métrica, destacam-se quatro eventos estruturais: a \textbf{crise de confiança de 2002} (\emph{jun/2002--jan/2003}), associada à incerteza eleitoral e à parada súbita de capitais; a \textbf{Crise Financeira Global de 2008} (\emph{set/2008--mar/2009}), que impôs um severo choque de liquidez externa; a \textbf{recessão doméstica de 2015--2016} (\emph{jan/2015--dez/2016}), deflagrada por desajustes fiscais e realinhamento de preços administrados; e a \textbf{pandemia de Covid-19} (\emph{mar/2020--dez/2020}), choque sanitário que exigiu estímulos monetários extraordinários. Inclusive, este cenário de deterioração fiscal impõe o teste prático da hipótese de dominância fiscal levantada na introdução, onde a restrição orçamentária pode ter limitado a capacidade de reação do BC via juros.

\subsection{Indicadores Econômicos}
\label{subsec:indicadores_economicos}

Esta subseção descreve, de forma sintética, a evolução conjunta de inflação, taxa de juros, expectativas e atividade ao longo do regime de metas, com ênfase nos principais episódios de ruptura destacados no Gráfico \ref{fig:series_macro}. O objetivo é registrar fatos estilizados e motivar a estratificação por subperíodos na etapa econométrica, sem antecipar inferências causais.

Antes de discutir os episódios, convém explicitar uma dimensão operacional dos dados. A Tabela \ref{tab:desc_stats} resume estatísticas descritivas das séries utilizadas, construídas a partir de bases do Bacen e do IBGE. Em particular, parte das variáveis é observada originalmente em frequência mensal ou trimestral (p.ex., IPCA e PIB), enquanto outras são observadas com maior granularidade (p.ex., meta Selic e expectativas). Para uniformizar o painel e viabilizar comparações e Gráficos ao longo do tempo, as séries foram alinhadas em uma base temporal comum, com replicação do valor observado dentro do período correspondente quando necessário. Esse procedimento explica as contagens elevadas (\textit{count}) reportadas na tabela, sem alterar a interpretação econômica das trajetórias ilustradas nos Gráficos.

\begin{table}[htbp]
\centering
\caption{Estatísticas Descritivas das Variáveis Macroeconômicas (1999--2024)}
\begin{tabular}{lrrrrr}
\toprule
 & IPCA (Mensal) & IPCA (12m) & Selic (Meta) & Focus & PIB (Indice) \\
\midrule
count & 9437.00 & 9437.00 & 9434.00 & 9132.00 & 9437.00 \\
mean & 0.51 & 6.25 & 12.69 & 5.63 & 143.19 \\
std & 0.39 & 2.72 & 5.55 & 1.93 & 23.25 \\
min & $-$0.68 & 1.65 & 2.00 & 1.61 & 99.26 \\
25\% & 0.26 & 4.48 & 9.00 & 4.24 & 120.84 \\
50\% & 0.46 & 5.91 & 12.25 & 5.50 & 153.66 \\
75\% & 0.69 & 7.35 & 16.00 & 6.42 & 161.34 \\
max & 3.02 & 17.24 & 45.00 & 12.41 & 179.53 \\
\bottomrule
\end{tabular}
\label{tab:desc_stats}\\
\textcolor{magenta}{<-> No focus não seria bom colocar (12m) para deixar claro que deveria ser comparado com a coluna de inflação de 12m? <->}
\footnotesize

\vspace{0.2cm}
\textbf{Fontes:} Bacen e IBGE, via pacotes Python (sidrapy e python-bcb).
\end{table}

% Crise de Confiança 2002-2003
No início da amostra, destaca-se a chamada \enquote{Crise de Confiança} de 2002--2003, associada à transição do governo Fernando Henrique Cardoso para o governo Luiz Inácio Lula da Silva e ao ajuste de expectativas que se refletiu em forte depreciação cambial. Conforme indicado no Gráfico \ref{fig:series_macro}, observa-se elevação acentuada da inflação acumulada em 12 meses e aumento expressivo da taxa Selic no período sombreado (jun/2002--jan/2003), em paralelo à deterioração das expectativas medidas pelo Focus. Do ponto de vista descritivo, esse episódio é caracterizado por uma reação monetária contracionista em um ambiente de desancoragem de expectativas, seguida por reconvergência gradual da inflação e das expectativas no ciclo subsequente. Essa dinâmica sugere, como hipótese a ser testada formalmente, que a autoridade monetária buscou reancorar expectativas por meio de um aumento relevante da taxa real ex-ante, compatível com a lógica do Princípio de Taylor no arcabouço novo-keynesiano.

% Crise Financeira Global de 2008--2009
Entre 2004 e 2010, o Gráfico \ref{fig:series_macro} sugere um intervalo de relativa estabilidade macroeconômica, com crescimento do produto e inflação moderada na maior parte do período, interrompido pela Crise Financeira Global de 2008--2009. O choque externo aparece como queda do nível de atividade e elevação transitória da incerteza, seguida por recuperação relativamente rápida do PIB. Do ponto de vista descritivo, o comportamento da Selic no episódio é consistente com uma resposta contracíclica: redução da taxa nominal em ambiente de desaceleração, com expectativas que, em geral, permanecem mais próximas do alvo quando comparadas aos episódios de maior desancoragem observados em outros momentos da amostra.

% Recessão Doméstica de 2015--2016
A partir de 2014, observa-se uma mudança de regime marcada pela recessão doméstica de 2015--2016. No Gráfico \ref{fig:series_macro}, o período sombreado (jan/2015--dez/2016) combina queda persistente do produto com inflação elevada e expectativas pressionadas, configurando um ambiente próximo à estagflação. Essa combinação é particularmente relevante para o objetivo do artigo: em termos de interpretação novo-keynesiana, um cenário com hiato negativo coexistindo com inflação alta é compatível com a presença de choques de oferta, realinhamentos de preços relativos ou deterioração do componente expectacional, fatores que podem reduzir a efetividade de uma resposta monetária convencional. Ainda assim, os dados sugerem uma postura contracionista em parte do intervalo, com juros elevados em meio à inflação e expectativas acima do desejado, motivando investigar se a resposta estimada à inflação foi suficiente para satisfazer $\hat{\phi}_\pi>1$ nesse subperíodo.

\begin{figure}[htbp]
    \centering
    \caption{Dinâmica da Inflação, Taxa de Juros, Expectativas e PIB (1999--2024)}
    \includegraphics[width=1\textwidth]{graficos/series_macroeconomicas.png}
    \label{fig:series_macro}
    \footnotesize
    \vspace{0.2cm}
    \textbf{Fontes:} Bacen e IBGE, via pacotes Python (sidrapy e python-bcb).\\
\end{figure}

% Pandemia de Covid-19
O último grande choque da amostra corresponde à pandemia de Covid-19 (mar/2020--dez/2020). O Gráfico \ref{fig:series_macro} mostra a contração abrupta do nível de atividade e a resposta inicial de redução da Selic em um contexto de forte queda do produto. Em seguida, a normalização parcial do PIB e a reversão inflacionária de 2021--2022 vêm acompanhadas por elevação expressiva dos juros e por deterioração temporária das expectativas, sugerindo um ciclo de aperto robusto voltado à reancoragem nominal. A avaliação quantitativa da intensidade dessa reação, contudo, depende da estimação formal da função de reação e será tratada na Seção~\ref{sec:resultados}.

O Gráfico \ref{fig:hiato_produto} reforça um ponto metodológico central: embora as estimativas de hiato difiram quanto à magnitude e à volatilidade (notadamente no Filtro de Hamilton), os sinais e os pontos de inflexão do ciclo são amplamente coincidentes \footnote{As estimativas de hiato do produto dependem do critério adotado para decompor o PIB entre tendência e componente cíclico. O \textbf{filtro de Hodrick--Prescott (HP)} obtém a tendência por meio de um problema de suavização que penaliza variações na segunda diferença do produto, sendo amplamente utilizado, embora sensível ao parâmetro de suavização e sujeito a viés de extremidade \cite{hodrick1997postwar}. O \textbf{filtro de Hamilton} interpreta o componente cíclico como o erro de previsão de um modelo autorregressivo para o produto, dispensando a extração explícita de tendência e mitigando problemas de viés de ponta, ainda que produza séries de hiato tipicamente mais voláteis \cite{hamilton2018filtering}. Como \textit{benchmarks} determinísticos adicionais, utiliza-se \textit{detrending} por \textbf{tendência linear e quadrática} (em geral sobre o log do PIB), que oferecem aproximações parcimoniosas para o crescimento de longo prazo, o que permite, no caso quadrático, variações graduais na taxa de crescimento, e são frequentemente empregados como checagem de robustez, dado que fatos cíclicos podem ser sensíveis à forma de remoção da tendência \cite{canova1998detrending,harvey1993detrending,kydlandprescott1990}. Por fim, inclui-se a estimativa institucional do hiato divulgada pela \citeonline{ifi_hiato_produto} (\textbf{IFI}), construída a partir da combinação de métodos univariados e multivariados aplicados ao PIB e a indicadores do mercado de trabalho, da utilização da capacidade e da produtividade, bem como da abordagem de função de produção. A medida central corresponde à mediana do conjunto de estimativas, reportada com intervalo de plausibilidade, conforme \citeonline{orair2018}, com tendências obtidas a partir de séries históricas ampliadas para reduzir o viés de final de amostra.}. Em particular, os episódios recessivos de 2015--2016 e 2020 aparecem de forma robusta nas diferentes metodologias, sugerindo que, apesar da incerteza inerente à mensuração do produto potencial, o diagnóstico qualitativo sobre o estado cíclico da economia é relativamente estável.

Por fim, ressalta-se que a análise aqui permanece estritamente descritiva. A co-movimentação entre inflação, expectativas, produto e taxa de juros documentada nos Gráficos \ref{fig:series_macro} e \ref{fig:hiato_produto} é consistente com os mecanismos previstos pelo modelo novo-keynesiano, mas não permite inferências causais nem conclusões sobre determinância do equilíbrio. A presença de defasagens, simultaneidade e o papel das expectativas requerem uma abordagem econométrica formal, desenvolvida nas seções seguintes, na qual se testará explicitamente se a resposta da política monetária, em diferentes subperíodos, foi consistente com as condições teóricas de unicidade e estabilidade do equilíbrio.

\begin{figure}[htbp]
    \centering
    \caption{Estimativas de Hiato do Produto sob Diferentes Metodologias (1999--2024)}
    \includegraphics[width=1\textwidth]{graficos/hiato_produto.png}
    \label{fig:hiato_produto}
    \footnotesize
    \textbf{Fonte:} Elaboração própria, baseada em dados do Bacen, coletados via pacotes Python (python-bcb).\\
    \textit{\textbf{Nota:} Comparativo entre Filtro HP, Filtro de Hamilton, Tendência Linear, Quadrática e estimativa da IFI. Valores negativos indicam ociosidade da economia.}
\end{figure}

Os padrões documentados nesta Seção sugerem que a condução da política monetária brasileira ao longo do regime de metas não foi invariável no tempo, apresentando diferenças sistemáticas na forma e na intensidade da resposta da taxa de juros a desvios da inflação, do hiato do produto e das expectativas. A evidência descritiva (ver Gráfico~\ref{fig:series_macro} e Tabela \ref{tab:desc_stats}) aponta para episódios em que o aperto monetário foi suficientemente agressivo para conter processos de desancoragem das expectativas inflacionárias, como no início do regime de metas, em 2002--2003, e mais recentemente no ciclo de alta iniciado em 2021, quando a taxa Selic foi elevada de forma rápida e persistente em resposta à deterioração das expectativas captadas pelo boletim Focus. 

Por outro lado, os dados também indicam períodos em que choques adversos de demanda ou de oferta impuseram trade-offs mais severos à estabilização macroeconômica, como durante a crise financeira internacional de 2008--2009 e a pandemia de COVID-19, quando a política monetária operou em ambiente de elevada incerteza e forte contração do produto, bem como no episódio de 2015--2016, caracterizado por inflação elevada, recessão profunda e dificuldades adicionais de coordenação entre política monetária e fiscal. No entanto, tais inferências não são identificáveis a partir de inspeção visual das séries, uma vez que inflação, atividade, expectativas e juros são determinados de forma conjunta, dinâmica e endógena, com defasagens relevantes. Diante disso, a Seção~\ref{sec:metodologia} propõe uma estratégia de identificação formal baseada na estimação de diferentes especificações da Regra de Taylor (retrospectiva, prospectiva e híbrida) explorando segmentações temporais e possíveis quebras estruturais, com o objetivo de avaliar se os parâmetros estimados satisfazem o Princípio de Taylor e são compatíveis com as condições de determinância de \citeonline{bullard2002learning}. Amiúde, os padrões visuais dos gráficos, as informações descritas e os dados demonstrados não permitem inferir com precisão e causalidade se a política monetária foi, de fato, estabilizadora no sentido NK, pois inflação, expectativas e juros são endógenos e determinados simultaneamente, o que nos leva a testar formalmente essas hipóteses a partir da seção seguinte.

% ***********************************************************

% ===========================================================
\section{Metodologia}
% ===========================================================
\label{sec:metodologia}

Esta Seção~apresenta a estratégia empírica adotada para investigar a condução da política monetária brasileira no período de vigência do regime de metas de inflação. Conforme discutido na Seção~\ref{sec:introducao}, o objeto do trabalho é a reação sistemática do Banco Central do Brasil às flutuações da inflação e da atividade econômica, enquanto o problema central consiste em avaliar se essa reação foi compatível com a estabilidade macroeconômica conforme o modelo Novo-Keynesiano, e tem como objetivo geral avaliar se o Princípio de Taylor foi satisfeito ao longo dos anos e se os parâmetros estimados são compatíveis com a determinância do equilíbrio no modelo. Como objetivos específicos, destacam-se a (i) estimação de diferentes especificações da Regra de Taylor; (ii) verificação do Princípio de Taylor ($\hat{\phi}_\pi > 1$); (iii) avaliação da condição de Bullard--Mitra; e (iv) exploração das implicações dinâmicas por meio de simulações em modelo NK calibrado para a economia brasileira.

% **Objeto**: Reação sistemática da política monetária brasileira (1999--2024).

% **Problema**: Avaliar se a reação da política monetária foi compatível com a estabilidade macroeconômica no modelo NK.

% **Objetivo geral**: Testar o Princípio de Taylor e a condição de determinância.

% **Objetivos específicos**:
%     (i) Estimar diferentes especificações da Regra de Taylor.
%         - >>> Regra backward-looking (MQO);
%         - >>> Regra forward-looking (GMM);
%         - >>> Regra híbrida (GMM);
%     (ii) Verificar se o Princípio de Taylor ($\hat{\phi}_\pi > 1$) é satisfeito.
%         - Teste de Wald unilateral;
%     (iii) Avaliar a condição de Bullard--Mitra.
%         - >>> Via inserção dos resultados do teste de Wald na inequação;
%     (iv) Explorar implicações dinâmicas via simulações em modelo NK calibrado.

A estratégia empírica fundamenta-se no arcabouço teórico apresentado nas Seções~\ref{sec:regras_politica_estabilidade} e~\ref{sec:arcabouco_teorico}, nas quais são formalizadas as equações estruturais do modelo Novo-Keynesiano e as condições necessárias para a determinação do equilíbrio, em particular a condição de Bullard--Mitra. A perspectiva é traduzir os conceitos em especificações econométricas com dados observáveis, por meio da estimação de diferentes versões da regra de Taylor. A Seção está organizada da seguinte forma: inicialmente, descrevem-se a base de dados, as variáveis utilizadas, suas fontes e o tratamento das séries; em seguida, apresentam-se as especificações econométricas e a estratégia de estimação adotada; posteriormente, detalham-se os testes empíricos do Princípio de Taylor e da condição de determinância; por fim, discutem-se as extensões relacionadas à análise por regimes e às quebras estruturais.

% ///////////////////////////
%% PREPARAÇÃO DAS SÉRIES
% ///////////////////////////

% -----------------------------------------------------------
\subsection{Bases de Dados}
% -----------------------------------------------------------
\label{subsec:bases_de_dados}

A base de dados foi construída a partir de fontes oficiais \footnote{O meio de captura dos dados se deu via bibliotecas Python (sidrapy e python-bcb).}. As séries referentes à política monetária e expectativas têm como origem o Sistema Gerenciador de Séries Temporais (SGS) do Banco Central do Brasil (Bacen), abrangendo o histórico da meta da taxa Selic e as medianas diárias do Sistema Expectativas de Mercado (Focus). Os dados de atividade econômica e preços provêm do Instituto Brasileiro de Geografia e Estatística (IBGE), especificamente as séries históricas do Índice Nacional de Preços ao Consumidor Amplo (IPCA) e os números índices do Produto Interno Bruto (PIB). Complementarmente, indicadores de atividade auxiliar e métricas de hiato calculadas por instituições independentes (como o IFI) e pelo próprio corpo técnico do Banco Central foram considerados para fins de robustez. A amostra cobre o período de consolidação do regime de metas, estendendo-se de janeiro de 1999 até dezembro de 2024, englobando diversos ciclos de aperto e relaxamento monetário sob diferentes administrações da autoridade monetária, consoante contetualização feita na Seção~\ref{sec:politica_monetaria_brasil_1999_2024}.

Por este norte, cada especificação é estimada na maior subamostra equilibrada compatível com os requisitos de informação do modelo. Em particular, as regras prospectivas e híbridas são estimadas a partir do primeiro trimestre de 2000, refletindo a disponibilidade de dados sobre expectativas de inflação.

% -----------------------------------------------------------
\subsection{Variáveis}
% -----------------------------------------------------------
\label{subsec:variaveis}

O conjunto de variáveis selecionado para a estimação reflete a estrutura canônica do modelo NK adaptada à frequência e às particularidades da economia brasileira, aproveitando os dados explanados na Seção~\ref{sec:politica_monetaria_brasil_1999_2024}. \textbf{(I)} A \textbf{taxa de juros nominal} ($i_t$) é representada pela \textbf{taxa Selic} definida pelo COPOM; \textbf{(II)} A variável de \textbf{inflação} ($\pi_t$) corresponde à variação acumulada em 12 meses do \textbf{IPCA}, servindo como proxy observável para o alvo do regime de metas; \textbf{(III)} Para capturar o produto da economia ($y_t$), utiliza-se o \textbf{Índice de Volume do PIB} (base 1999=100); \textbf{(IV)} O \textbf{hiato do produto} é estimado por meio de diferentes metodologias de filtragem estatística, incluindo o filtro Hodrick-Prescott (HP), o filtro de Hamilton e tendências determinísticas (linear e quadrática), conforme ilustrado na Seção~\ref{sec:politica_monetaria_brasil_1999_2024} e na subseção a seguir, permitindo mitigar a incerteza inerente a essa variável não observável; \textbf{(V)} Por fim, as \textbf{expectativas inflacionárias} ($E_t \pi_{t+1}$), cruciais para as especificações \textit{forward-looking}, são representadas pela \textbf{mediana} das projeções de mercado do \textbf{Relatório Focus} para o IPCA do ano corrente, capturando a formação de expectativas em tempo real diante de choques exógenos.

% -----------------------------------------------------------
\subsection{Tratamento das Séries} 
% -----------------------------------------------------------
\label{subsec:tratamento_series}

Todas as variáveis são analisadas em \textbf{frequência trimestral}, compatibilizando a periodicidade do PIB, utilizado na construção do hiato do produto (vide subseção~\ref{subsec:indicadores_economicos}), com as séries de inflação, taxa de juros e expectativas. Essa escolha assegura coerência temporal entre as variáveis do modelo e é consistente com o horizonte de transmissão da política monetária, no qual os efeitos das decisões do Copom sobre a atividade e os preços tendem a se materializar ao longo de vários trimestres, e não em flutuações mensais de alta frequência. As séries de atividade econômica foram ajustadas sazonalmente com o objetivo de remover padrões intra-anuais sistemáticos, como efeitos de calendário, que não estão associados ao ciclo econômico nem à condução da política monetária, facilitando a identificação das flutuações cíclicas relevantes para o estudo do hiato do produto \cite{woodford2003interest, gali2015monetary}.

% ARIMA
O \textbf{ajuste sazonal} das séries de atividade econômica foi realizado por meio do procedimento X-13-ARIMA-SEATS, metodologia adotada por instituições oficiais e pela literatura empírica \citeonline{sax2018seasonal}. O método combina pré-ajustes determinísticos via modelagem regARIMA com técnicas de filtragem do tipo X-11, permitindo remover efeitos sazonais e de calendário, bem como reduzir problemas de borda e revisões excessivas das estimativas correntes.\footnote{\textcolor{magenta}{<-> POs detalhes formais do procedimento, incluindo a especificação do modelo regARIMA, a decomposição multiplicativa e os filtros empregados, são apresentados no Apêndice~A. MUDAR ISSO PRA METODOLOGIA?? <->}}

% Teste de Estacionariedade / Raiz Unitária
Em seguida, avaliou-se a propriedade de \textbf{estacionariedade} das séries de inflação, taxa de juros e hiato do produto por meio dos testes de raiz unitária Augmented Dickey--Fuller (ADF) e KPSS. Esses testes são empregados de forma complementar na literatura de séries temporais macroeconômicas, uma vez que partem de hipóteses nulas opostas \cite{dickey1979distribution}. Sua utilização conjunta permite verificar se as transformações adotadas são compatíveis com a modelagem dinâmica implícita no arcabouço Novo-Keynesiano, no qual inflação e hiato são tratados como variáveis estacionárias em torno de seus valores de equilíbrio de longo prazo \cite{kwiatkowski1992testing}.

\textcolor{magenta}{<-> Adicionar um parágrafo explicando como lidamos com casos de raiz unitária na taxa de juros. Se $\rho \approx 1$, a condição de estabilidade muda (a regra precisa ser super-inercial ou responder em diferenças). Deixar claro que estamo cientes de que a fórmula padrão $\frac{\phi}{1-\rho}$ falha nesse caso limite, como falhou no modelo híbrido <->}

% Filtro HP
Em específico, dada a natureza não observável do produto potencial, o hiato do produto foi obtido por meio de procedimentos de \textbf{filtragem estatística}. A especificação adota o filtro de \citeonline{hodrick1997postwar}, com parâmetro de suavização $\lambda = 1600$, padrão para dados trimestrais. O filtro define o componente de tendência como a solução do problema de minimização:
\begin{equation}
\min_{\{\tau_t\}} \sum_{t=1}^{T} (y_t - \tau_t)^2 
\;+\; \lambda \sum_{t=2}^{T-1} \left[(\tau_{t+1}-\tau_t)-(\tau_t-\tau_{t-1})\right]^2,
\end{equation}
onde $y_t$ denota o produto observado e $\tau_t$ sua tendência suavizada.

Reconhecendo a sensibilidade das estimativas do hiato ao método empregado, a robustez dos resultados econométricos é avaliada por meio de especificações alternativas, nas quais o hiato do produto é operacionalizado a partir do filtro de \citeonline{hamilton2018filtering}, de tendências determinísticas lineares e quadráticas, bem como de estimativas institucionais produzidas pela IFI e pelo Banco Central do Brasil. Essa estratégia permite tratar explicitamente a incerteza de mensuração associada ao hiato do produto, avaliando a estabilidade qualitativa dos resultados empíricos frente a diferentes procedimentos de extração do componente cíclico, conforme discutido na Seção~\ref{sec:politica_monetaria_brasil_1999_2024}.

% ///////////////////////////
%% MODELAGEM ECONOMÉTRICA
% ///////////////////////////

% Deixar claro em cada subseção qual objetivo (geral ou específico) está sendo tratado

% -----------------------------------------------------------
\subsection{Especificação Econométrica}
\label{subsec:especificacao_econometrica}
% -----------------------------------------------------------

Esta subseção traduz para o plano econométrico as regras de política discutidas na Seção~\ref{subsec:regra_taylor}, especificando as equações efetivamente estimadas e a estratégia de identificação. Em todas as variantes, a taxa Selic segue uma regra de Taylor com suavização, na qual os parâmetros de reação à inflação e ao hiato do produto têm a mesma interpretação teórica dos coeficientes \(\phi_\pi\) e \(\phi_y\) que entram na condição de determinância apresentada na Seção~\ref{subsec:condicoes_determinancia}. A forma estimável da regra de juros é escrita diretamente em termos de regressões lineares, às quais se impõe posteriormente a interpretação estrutural de longo prazo.

As subseções \ref{subsubsec:regra_backward_mqo} a \ref{subsubsec:regra_hibrida_suavizacao_gmm}, a seguir, descrevem as especificações econométricas correspondentes às diferentes regras de Taylor discutidas na Seção 3.3. Posteriormente, a Seção \ref{subsec:estrategia_estimacao} detalha de forma unificada a estratégia de estimação e os instrumentos utilizados para a identificação dos parâmetros.

% -----------------------------------------------------------
\subsubsection{Regra \textit{Backward-Looking} (MQO).}
\label{subsubsec:regra_backward_mqo}
% -----------------------------------------------------------

Na especificação retrospectiva, o Banco Central reage à inflação e ao hiato do produto correntes, consistentes com a regra \textit{backward} apresentada em \ref{subsec:regra_taylor_backward}. A equação estimada é:

\begin{equation}
  i_t = \beta_0 + \beta_1 i_{t-1} + \beta_2 \pi_t + \beta_3 \tilde y_t + u_t,
  \label{eq:regra_backward_mqo}
\end{equation}
onde \(i_t\) é a meta da taxa Selic; \(\beta_0\) é o intercepto da equação; \(\beta_1\) mede o grau de inércia ou suavização da taxa de juros, sendo \(i_{t-1}\) a taxa Selic do período anterior; \(\beta_2\) é o coeficiente de reação à inflação corrente \(\pi_t\) (medida pelo IPCA); \(\beta_3\) é o coeficiente de reação ao hiato do produto \(\tilde y_t\); e \(u_t\) é um erro de política com \(E(u_t|\mathcal F_{t-1})=0\), dado que o conjunto de informação $\mathcal{F}_{t-1}$ representa o histórico de variáveis macroeconômicas e institucionais observáveis até o período $t-1$. Neste sentido, a condição $E(u_t \mid \mathcal{F}_{t-1}) = 0$ implica que o erro de política captura apenas choques não antecipáveis à luz da informação disponível antes da decisão do Copom. 

Cabe ressaltar que essa especificação \textit{backward-looking} deve ser interpretada como um exercício de referência (\textit{benchmark}), uma vez que a hipótese de predeterminação contemporânea das variáveis explicativas pode ser questionada em ambientes com forte formação de expectativas e transmissão rápida da política monetária. Assim, as evidências obtidas por MQO são complementadas pelas especificações \textit{forward-looking}, estimadas por GMM, que exploram de forma mais direta a estrutura expectacional do modelo novo-keynesiano. Isto dito, a comparação com a forma teórica com suavização (Seção~\ref{par:taylor_hibrida}) permite interpretar
\[
\rho = \beta_1,\qquad
\phi_\pi = \frac{\beta_2}{1-\beta_1},\qquad
\phi_y = \frac{\beta_3}{1-\beta_1},
\]
de modo que \(\phi_\pi\) e \(\phi_y\) são coeficientes de reação de longo prazo à inflação e ao hiato. A estimação é realizada por MQO com matriz de covariância robusta de Newey--West, para acomodar heterocedasticidade e autocorrelação em \(u_t\) \cite{smith1994newey}.

A escolha do método de estimação, inspirado em \citeonline{hubert2019role}, está diretamente relacionada à natureza informacional de cada especificação da regra de política monetária. No caso da regra \textit{backward-looking}, os regressores correspondem a variáveis contemporâneas ou defasadas observáveis no momento da decisão, de modo que, sob a hipótese padrão de que o erro de política representa um choque não antecipável, vale a condição de exogeneidade $E(u_t \mid \mathcal F_{t-1})=0$. Nessas circunstâncias, a estimação por MQO é consistente, sendo a correção de Newey--West suficiente para acomodar heterocedasticidade e autocorrelação serial no termo de erro \footnote{Note-se que, no contexto institucional brasileiro, a decisão do Copom é tomada com base em informações consolidadas referentes ao período anterior e em indicadores de inflação acumulada (ver: \url{https://www.bcb.gov.br/controleinflacao/copom}), de modo que a inflação corrente utilizada na estimação deve ser interpretada como observável e predeterminada no horizonte da decisão de política. Saliente-se também que, ainda que a inflação corrente possa refletir revisões posteriores, a prática institucional do Copom e o uso de indicadores acumulados justificam sua interpretação como predeterminada no horizonte trimestral.}.

% -----------------------------------------------------------
\subsubsection{Regra \textit{forward-looking} (GMM)}
\label{subsubsec:regra_forward_gmm}
% -----------------------------------------------------------

Em contraste, na especificação \textit{forward-looking}, a autoridade monetária reage explicitamente a expectativas de inflação e de hiato do produto, variáveis endógenas que dependem do mesmo conjunto de informação utilizado na formulação da política e que, em geral, estão correlacionadas com o choque monetário contemporâneo. Nessa situação, a condição de exogeneidade forte requerida pelo MQO é violada, tornando necessária a utilização de estimadores baseados em condições de momento. A estimação por GMM permite explorar instrumentos pertencentes ao conjunto de informação disponível antes da decisão de política, assegurando a consistência dos estimadores mesmo na presença de endogeneidade das expectativas. Essa distinção metodológica segue o aplicado em trabalhos da área, em particular nos trabalhos de \citeonline{clarida2000monetary} e em aplicações mais recentes como \citeonline{jayawickrema2019monetary} e \citeonline{hubert2019role}, que adotam MQO em especificações retrospectivas e GMM em regras prospectivas.

Posto isso, na especificação prospectiva a autoridade monetária reage às expectativas de inflação e de hiato do produto, em consonância com a estrutura \textit{forward-looking} da NKPC e da equação IS discutidas nas Seções~\ref{subsec:nkpc} e~\ref{subsec:is_dinamica}. Nesse sentido, estima-se a seguinte regra de juros com suavização:
\begin{equation}
  i_t =  \gamma_0 + \gamma_1 i_{t-1} + \gamma_2 E_t\{\pi_{t+1}\} + \gamma_3 E_t\{\tilde y_{t+1}\} + e_t,
  \label{eq:regra_forward_gmm}
\end{equation}
onde \(i_t\) é a taxa de juros nominal definida pelo Banco Central no período \(t\); \(\gamma_0\) é o intercepto da equação; \(\gamma_1\) representa o grau de inércia ou suavização da taxa de juros, e \(i_{t-1}\) é a taxa de juros nominal observada no período anterior, refletindo o ajuste gradual da política monetária ao longo do tempo; \(\gamma_2\) é o coeficiente de reação à expectativa de inflação futura \(E_t\{\pi_{t+1}\}\); \(\gamma_3\) é o coeficiente de reação à expectativa do hiato do produto \(E_t\{\tilde y_{t+1}\}\); e \(e_t\) é um choque de política monetária, capturando fatores não sistemáticos, com \(E(e_t|\mathcal F_{t-1})=0\). Novamente, o conjunto de informação $\mathcal{F}_{t-1}$ representa o histórico de variáveis macroeconômicas e institucionais observáveis até o período $t-1$, de modo que o termo defasado do instrumento captura o grau de suavização da taxa de juros, enquanto as expectativas condicionais refletem a natureza prospectiva da condução da política monetária.

Empiricamente, \(E_t\{\pi_{t+1}\}\) é aproximado pela mediana das expectativas do mercado coletadas no Relatório Focus para a inflação acumulada em doze meses, ao passo que \(E_t\{\tilde y_{t+1}\}\) é capturado por medidas de hiato prospectivo ou, em especificações alternativas, pelo hiato contemporâneo, utilizado como proxy da informação disponível à autoridade monetária no momento da decisão. A interpretação dos coeficientes de longo prazo segue a parametrização usual:
\[
\rho = \gamma_1, \qquad
\phi_\pi = \frac{\gamma_2}{1-\gamma_1}, \qquad
\phi_y = \frac{\gamma_3}{1-\gamma_1},
\]
de modo que \(\phi_\pi\) e \(\phi_y\) podem ser diretamente avaliados à luz das condições de determinacidade apresentadas na Seção~\ref{subsec:condicoes_determinancia}.

% -----------------------------------------------------------
\subsubsection{Regra Híbrida com Suavização (GMM)}
\label{subsubsec:regra_hibrida_suavizacao_gmm}
% -----------------------------------------------------------

Além das especificações puramente retrospectiva e prospectiva, estima-se uma regra híbrida de política monetária que permite capturar simultaneamente a resposta da autoridade monetária à inflação corrente, às expectativas de inflação futura e ao hiato do produto, bem como a presença de suavização da taxa de juros. Essa formulação está diretamente alinhada à regra teórica híbrida discutida na Seção~\ref{par:taylor_hibrida} e reflete a estrutura informacional implícita nas equações \textit{forward-looking} da NKPC e da equação IS apresentadas nas Seções~\ref{subsec:nkpc} e~\ref{subsec:is_dinamica}.

A equação estimada assume a seguinte forma:
\begin{equation}
  i_t = \delta_0 + \delta_1 i_{t-1} + \delta_2 \pi_t + \delta_3 E_t\{\pi_{t+1}\} + \delta_4 \tilde y_t + v_t,
  \label{eq:regra_hibrida_gmm}
\end{equation}
em que \(i_t\) representa a meta da taxa Selic, \(\pi_t\) é a inflação corrente medida pelo IPCA, \(E_t\{\pi_{t+1}\}\) denota a expectativa condicional de inflação, \(\tilde y_t\) é o hiato do produto e \(v_t\) corresponde a um choque de política monetária.

A inclusão simultânea de inflação corrente e esperada torna a regra híbrida particularmente adequada para avaliar empiricamente se a condução da política monetária responde predominantemente a informações observáveis em tempo real ou a expectativas formadas prospectivamente, conforme discutido na literatura sobre regras de Taylor com informação imperfeita e expectativas racionais. Ao mesmo tempo, a presença do termo defasado da taxa de juros permite capturar a inércia observada empiricamente na condução da política monetária.

Reforçando o que foi dito na seção anterior, do ponto de vista econométrico, a presença de expectativas condicionais implica a violação da condição de exogeneidade forte requerida pelo MQO, uma vez que \(E_t\{\pi_{t+1}\}\) é endógena e potencialmente correlacionada com o choque de política monetária contemporâneo. Nessas circunstâncias, a estimação por GMM é necessária para assegurar a consistência dos estimadores, explorando condições de momento baseadas em instrumentos pertencentes ao conjunto de informação disponível antes da decisão do Copom. Essa estratégia segue a prática consolidada na literatura empírica de regras de Taylor, em particular nos trabalhos de \citeonline{clarida2000monetary}, \citeonline{orphanides2001monetary} e em aplicações recentes como \citeonline{hubert2019role}, que empregam GMM em especificações que incorporam expectativas explícitas.

A interpretação dos coeficientes de longo prazo segue diretamente da parametrização com suavização, definindo-se \(\rho = \delta_1\) e:
\[
\phi_\pi^{b} = \frac{\delta_2}{1-\delta_1},\qquad
\phi_\pi^{f} = \frac{\delta_3}{1-\delta_1},\qquad
\phi_y = \frac{\delta_4}{1-\delta_1},
\]
de modo que \(\phi_\pi = \phi_\pi^{b} + \phi_\pi^{f}\) sintetiza a intensidade total da resposta da política monetária à inflação, combinando componentes \textit{backward-looking} e \textit{forward-looking}. Esses coeficientes podem ser avaliados à luz das condições de determinância apresentadas na Seção~\ref{subsec:condicoes_determinancia}, permitindo verificar se a regra híbrida satisfaz o Princípio de Taylor no longo prazo.

% -----------------------------------------------------------
\subsection{Estratégia de Estimação}
\label{subsec:estrategia_estimacao}
% -----------------------------------------------------------

A escolha do estimador reflete a forma como as variáveis no lado direito se relacionam com o erro de política e com a informação disponível no instante \(t\).

\textcolor{magenta}{<-> EXPLICAR QUANTOS TRIMESTRES TEM NO TOTAL E QUANTOS SOBRARAM APÓS OS LAGS E JUSTIFICAR ISSO. <->}

% -----------------------------------------------------------
\subsubsection{MQO para a Regra \textit{Backward}.}
\label{subsubsec:mqo_regra_backward}
% -----------------------------------------------------------

Na equação \eqref{eq:regra_backward_mqo}, assume-se que \(\pi_t\) e \(\tilde y_t\) são, no horizonte trimestral, predeterminados em relação a \(u_t\), de forma que \(E(u_t|\pi_t,\tilde y_t,i_{t-1},\mathcal F_{t-1})=0\). Sob essa hipótese, MQO fornece estimadores consistentes, e a correção HAC (\textit{Heteroskedasticity- and autocorrelation-consistent estimators}) garante inferência robusta à autocorrelação serial na série da Selic. Os estimadores HAC , também conhecidos como \textit{Estimadores de variância Newey-West}, propostos pela primeira vez por \citeonline{neweyWest}, estima a matriz de covariância do modelo corrigindo para autocorrelação e heterocedasticidade.
\textcolor{magenta}{<-> Procurar artigo que justifica/prova a existência de autocorrelação e heterocedasticidade em curvas de juros para acrescentá-lo neste parágrafo. <->}

% -----------------------------------------------------------
\subsubsection{GMM para Regras \textit{Forward} e Híbrida.}
\label{subsubsec:gmm_regras_forward_hibrida}
% -----------------------------------------------------------

Nas equações \eqref{eq:regra_forward_gmm} e \eqref{eq:regra_hibrida_gmm}, as variáveis de expectativas \(E_t\{\pi_{t+1}\}\) e \(E_t\{\tilde y_{t+1}\}\) são potencialmente endógenas, tanto por incorporarem erros de previsão como por refletirem a reação simultânea do mercado às decisões do Comitê de Política Monetária. Seguindo \citeonline{clarida2000monetary} e a crítica de \citeonline{orphanides2001monetary}, a estimação é realizada por Método dos Momentos Generalizados (GMM), com condições de momento
\begin{equation}
  E\!\big[Z_t\,e_t(\theta)\big] = 0
  \quad\text{e}\quad
  E\!\big[Z_t\,v_t(\theta)\big] = 0,
  \label{eq:momentos_gmm_simplificado}
\end{equation}
em que \(Z_t\) é um vetor de instrumentos formado por defasagens de variáveis observáveis:
\begin{equation}
  Z_t
  =
  \big(
    1,\,
    i_{t-1}, i_{t-2},\,
    \pi_{t-1}, \pi_{t-2},\,
    \tilde y_{t-1}, \tilde y_{t-2},\,
    ICBr_{t-1},\dots
  \big)',
  \label{eq:instrumentos_simplificado}
\end{equation}
e \(\theta\) reúne os parâmetros de interesse em cada especificação. 

O termo \(ICBr_{t-1}\) corresponde a um indicador sintético de condições macroeconômicas e financeiras do Brasil, utilizado como instrumento adicional por pertencer ao conjunto informacional disponível em \(t-1\) (\(ICBr_{t-1}\in\mathcal F_{t-1}\)) \footnote{Enquanto \(\mathcal F_{t-1}\) denota o conjunto informacional abstrato relevante para a exogeneidade condicional no modelo \textit{backward}, no contexto das estimações \textit{forward-looking} esse conjunto é aproximado empiricamente por um vetor finito de instrumentos \(Z_t \subset \mathcal F_{t-1}\), incluindo variáveis macroeconômicas defasadas e indicadores sintéticos como \(ICBr_{t-1}\).
}, sendo correlacionado com as expectativas formadas pelo mercado, mas exógeno às inovações contemporâneas da política monetária. A inclusão do indicador sintético \(ICBr_{t-1}\) baseia-se na hipótese de exclusão de que, por sua construção e defasagem temporal, o índice não incorpora as inovações contemporâneas da política monetária no período \(t\), sendo plausivelmente exógeno às condições de erro da equação estimada. 

A validade dos instrumentos e a sobreidentificação são avaliadas pelo teste \(J\) de Hansen, padrão em aplicações de GMM com regras de Taylor \textit{forward-looking}.

% -----------------------------------------------------------
\subsubsection{Princípio de Taylor e Condição de Bullard--Mitra}
\label{subsubsec:principio_taylor_bullard_mitra}
% -----------------------------------------------------------

Os parâmetros de reação de longo prazo \(\phi_\pi\) e \(\phi_y\), derivados das equações \eqref{eq:regra_backward_mqo}--\eqref{eq:regra_hibrida_gmm}, são fundamentais para testar a aderência da política monetária às condições de estabilidade do modelo Novo-Keynesiano.

Contudo, como esses parâmetros representam transformações não-lineares dos coeficientes estimados (p. ex.: \(\phi_\pi = \gamma_2/(1-\gamma_1)\)), a inferência estatística direta não é trivial. Para contornar isso, emprega-se o Método Delta, que aproxima a matriz de covariância assintótica dos parâmetros estruturais via expansão de Taylor de primeira ordem, viabilizando a construção dos erros-padrão e a realização dos testes de Wald subsequentes \cite{greene2003econometric}.

Sendo assim, então se avalia o cumprimento do Princípio de Taylor por meio do teste de Wald unilateral \footnote{Registre-se que o teste de Wald avalia evidência estatística, enquanto Bullard–Mitra avalia consistência estrutural do modelo.}:

\begin{equation}
  H_0:\ \phi_\pi \le 1
  \qquad \text{,} \qquad
  H_1:\ \phi_\pi > 1.
  \label{eq:test_taylor_simplificado}
\end{equation}

Logo depois, a análise de estabilidade é aprofundada inserindo os pares estimados \((\hat\phi_\pi,\hat\phi_y)\) na condição de Bullard--Mitra para o modelo NK canônico:

\begin{equation}
  \kappa(\hat\phi_\pi -1) + (1-\beta)\hat\phi_y > 0.
  \label{eq:bullard_mitra_empirica_simplificada}
\end{equation}

Nesta desigualdade, os parâmetros estruturais \(\kappa\) (inclinação da Curva de Phillips) e \(\beta\) (fator de desconto) são calibrados conforme os valores discutidos na Seção~\ref{subsec:condicoes_determinancia}. Essa verificação permite classificar se, em cada especificação e subperíodo amostral, a condução da política monetária foi compatível com a unicidade do equilíbrio (determinância) e com a estabilidade sob aprendizagem (\textit{E-stability}).

% -----------------------------------------------------------
\subsubsection{Testes de Hipótese, Determinância e Instabilidade Paramétrica}
\label{subsubsec:testes_determinancia_quebras}
% -----------------------------------------------------------

Em todas as especificações com suavização da taxa de juros, os coeficientes relevantes para os testes de determinância são interpretados em termos de resposta de longo prazo da política monetária. Em particular, os parâmetros \( \phi_\pi \) e \( \phi_y \) utilizados na avaliação do Princípio de Taylor e da condição de Bullard--Mitra correspondem às elasticidades de longo prazo implícitas na regra estimada, obtidas a partir das transformações não lineares dos coeficientes estimados.

Isto posto, os parâmetros estruturais \(\phi_\pi\) e \(\phi_y\), recuperados via Método Delta conforme discutido anteriormente, são submetidos a três procedimentos de validação empírica.

Primeiro, avalia-se o cumprimento estrito do Princípio de Taylor via teste de Wald unilateral ($H_0: \phi_\pi \le 1$). Segundo, verifica-se a condição necessária e suficiente para a determinância do equilíbrio no modelo NK, dada pela desigualdade de \citeonline{bullard2002learning}:
\begin{equation}
    \kappa(\hat{\phi}_\pi - 1) + (1-\beta)\hat{\phi}_y > 0.
    \label{eq:bullard_mitra_empirica}
\end{equation}
Nesta expressão, os parâmetros profundos são calibrados para a economia brasileira seguindo \citeonline{kanczuk2004choques} e \citeonline{machado2014phillips}, assumindo-se valores consistentes com a frequência trimestral dos dados ($\beta \approx 0.985$, $\kappa \approx 0.05$).

Por fim, para investigar a estabilidade dos parâmetros ao longo do tempo e validar a segmentação da amostra proposta na Seção~\ref{sec:politica_monetaria_brasil_1999_2024}, aplica-se o teste de quebras estruturais endógenas de \citeonline{bai1998estimating} e \citeonline{bai2003computation}. Esse procedimento permite identificar, sem imposição a priori de datas, se as mudanças na condução da política monetária coincidem estatisticamente com as trocas de comando no Banco Central (Fraga, Meirelles, Tombini, Goldfajn, Campos Neto), oferecendo robustez à análise comparativa dos regimes de política.

% ***********************************************************

% PRINCÍPIOS DE TAYLOR
% ------------------------------
% * Funções de reação da Selic
%     - Modelo retrospectivo (MQO)
%         + Correção HAC (benchmark)
%     - Modelo prospectivo (GMM)
%     - Modelo híbrido com suavização (GMM)
%         + Validade dos instrumentos
%             / Teste J de Hansen (sobreidentificação)
% % * Parâmetros de reação de longo prazo ($\phi_\pi$, $\phi_y$)
%     - Avaliação do Princípio de Taylor
%         + Teste estatístico de $\phi_\pi > 1$
% % DETERMINÂNCIA (MODELO NK)
% ------------------------------
% * Condição de Bullard--Mitra (estabilidade local)
%     - Parâmetros estimados:
%         + $\hat{\phi}_\pi$
%         + $\hat{\phi}_y$
%     - Verificação da condição:
%         / $\kappa(\hat{\phi}_\pi - 1) + (1 - \beta)\hat{\phi}_y > 0$
% % * Extensão / Robustez temporal (se der tempo)
%     - Testes de quebras estruturais endógenas
%       (Bai--Perron, 1998, 2003)

% ***********************************************************

% ===========================================================
\section{Resultados}
\label{sec:resultados}
% ===========================================================

Esta Seção apresenta os resultados empíricos da estimação das diferentes especificações da Regra de Taylor para a economia brasileira, bem como os testes associados ao Princípio de Taylor e à condição de determinância do equilíbrio no modelo Novo-Keynesiano. Em consonância com a estratégia empírica delineada na Seção~\ref{sec:metodologia}, os resultados são organizados de forma a \textbf{(a)} distinguir entre as diferentes especificações econométricas e \textbf{(b)} avaliar a robustez das estimativas a alternativas de mensuração do hiato do produto.

A análise abrange o período de consolidação do Regime de Metas (1999--2024) para as especificações retrospectivas. Para as especificações prospectivas e híbridas, a amostra inicia-se em 2004, garantindo a consistência das séries de expectativas de mercado (Relatório Focus) e evitando a volatilidade excessiva do período de transição cambial (2002--2003).\footnote{A restrição da amostra para as estimações GMM justifica-se pela necessidade de séries longas e estáveis para os instrumentos de expectativas, conforme ponderado na Seção~\ref{subsec:bases_de_dados}.}

A análise adota como referência o hiato do produto estimado pelo Banco Central do Brasil, por refletir a informação efetivamente disponível à autoridade monetária no momento da tomada de decisões. Contudo, com o objetivo de avaliar a robustez dos resultados e mitigar potenciais problemas de endogeneidade, a análise é replicada com medidas alternativas, incluindo as estimativas da \cite{ifi_hiato_produto} e aquelas obtidas por meio dos filtros de Hamilton \cite{hamilton2018filtering} e Hodrick–Prescott \cite{hodrick1997postwar}, contextualizadas na subseção~\ref{subsec:indicadores_economicos}.

A Seção organiza-se em duas partes: \textbf{(i)} estimações das regras de Taylor (Seção~\ref{subsec:estimacoes_regra_taylor}); e \textbf{(ii)} implicações para a determinância do equilíbrio no modelo NK (Seção~\ref{subsec:determinancia_bullard_mitra}), culminando na análise dinâmica via simulação.

% PARTE 1: ESTIMATIVAS DAS REGRAS DE TAYLOR
% -----------------------------------------------------------
\subsection{Estimações da Regra de Taylor}
\label{subsec:estimacoes_regra_taylor}
% -----------------------------------------------------------

Inicialmente, apresentam-se as estimações das regras de Taylor \textit{backward-looking}, \textit{forward-looking} e híbrida, conforme especificado nos tópicos \ref{subsec:regra_taylor_backward}, \ref{subsubsec:regra_forward_gmm} e \ref{subsubsec:regra_hibrida_suavizacao_gmm}. Essas três especificações são analisadas de forma comparável, com ênfase na resposta sistemática da taxa de juros à inflação e ao hiato do produto, bem como no grau de persistência da política monetária. Em todas as estimações, a leitura dos resultados privilegia os coeficientes de curto prazo, que capturam diretamente a reação contemporânea da autoridade monetária às condições macroeconômicas observadas ou esperadas.

As estimações \textit{backward-looking}, obtidas por MQO com correção HAC, fornecem um ponto de partida empírico para caracterizar a resposta retrospectiva da política monetária brasileira, permitindo identificar regularidades básicas associadas à persistência da taxa de juros e à reação à inflação e ao hiato do produto. As especificações \textit{forward-looking}, estimadas via GMM, incorporam explicitamente expectativas racionais e permitem avaliar em que medida a autoridade monetária reage a variáveis prospectivas, em particular à inflação esperada. Por fim, a regra híbrida combina esses dois elementos ao incluir um termo de suavização da taxa de juros, oferecendo uma representação mais flexível e empiricamente plausível do comportamento da política monetária, especialmente em contextos de elevada persistência observada na Selic.

Para todas as especificações, a análise principal concentra-se nos coeficientes de curto prazo $(\hat{\phi}_\pi,\hat{\phi}_y)$, que governam a dinâmica local do sistema e são diretamente relevantes para a interpretação econômica das regras estimadas. Medidas implícitas de longo prazo são reportadas apenas como exercício de robustez, uma vez que, em ambientes caracterizados por elevada persistência da taxa de juros $(\rho \approx 1)$, tais medidas tornam-se altamente sensíveis e potencialmente pouco informativas. Essa estratégia permite distinguir com maior clareza entre evidência estatística de reação sistemática à inflação e ao hiato do produto e inferências estruturais que dependem criticamente da interpretação dos termos associados à persistência, preparando o terreno para a análise de determinância desenvolvida na subseção seguinte (\cite{clarida2000monetary}, \cite{woodford2003interest}, \cite{rudebusch2005monetary}).

% ----------------------------------------
\subsubsection{\textbf{Backward-Looking (MQO)}}
\label{mqo_backward_estimacao}

A Tabela~\ref{tab:mqo_backward_resultados} sintetiza as estimações da regra de Taylor \textit{backward-looking} para o período 1999--2024. A análise dos coeficientes revela padrões distintos quanto à inércia da política monetária e à sensibilidade do Banco Central a diferentes medidas de atividade econômica.

\paragraph{Persistência e Resposta à Inflação}
Um resultado comum a todas as especificações é a elevada suavização da taxa de juros. O coeficiente de persistência $\hat{\rho}$ situa-se sistematicamente no intervalo entre $0{,}92$ e $0{,}95$, com elevada significância estatística ($p < 0{,}01$). Essa evidência confirma a natureza gradualista da política monetária brasileira, sugerindo que a autoridade monetária evita movimentos abruptos na taxa Selic, possivelmente para mitigar riscos à estabilidade financeira.

Quanto ao mandato principal, a resposta de curto prazo à inflação ($\hat{\phi}_{\pi}$) mostra-se robusta e estável. Independentemente da medida de hiato utilizada ou da defasagem temporal da atividade ($t, t-1$ ou $t-2$), os coeficientes oscilam entre $0{,}31$ e $0{,}36$. Isso indica que, no curto prazo, a cada aumento de 1 p.p. na inflação acumulada, o Banco Central elevou a taxa nominal contemporânea em aproximadamente 0,34 p.p., \textit{ceteris paribus}. Embora menor que a unidade, esse coeficiente de curto prazo deve ser interpretado à luz da alta inércia: a resposta total é distribuída ao longo do tempo.

% ----------------------------------------------
% TABELA: Regras de Taylor backward-looking (MQO)
% ----------------------------------------------
\begin{table}[H]
\centering
\caption{Regras de Taylor backward-looking estimadas por MQO (HAC): persistência da Selic e respostas de curto prazo.}
\label{tab:mqo_backward_resultados}

\scriptsize
\setlength{\tabcolsep}{3.5pt}
\renewcommand{\arraystretch}{1.20}

% coluna X alinhada à esquerda
\newcolumntype{Y}{>{\raggedright\arraybackslash}X}
% macro para coeficiente (erro-padrão)
\newcommand{\coefse}[2]{\begin{tabular}[c]{@{}r@{}}#1\\\scriptsize(#2)\end{tabular}}

\begin{tabularx}{\textwidth}{l Y r r r r r r r}
\toprule
ID & Spec & N & $\rho$ & $\phi_{\pi}$ & $\phi_{y}$ & $p(\rho)$ & $p(\phi_{\pi})$ & $p(\phi_{y})$ \\
\midrule

\multicolumn{9}{l}{\textit{Hiato: Filtro HP}}\\
1  & \texttt{HP gap} ($\tilde y_{t}$)        & 82 & \coefse{0.9477}{0.017} & \coefse{0.3429}{0.038} & \coefse{0.1485}{0.054} & 0.0000 & 0.0000 & 0.0060 \\
2  & \texttt{HP gap} ($\tilde y_{t-1}$)      & 82 & \coefse{0.9435}{0.017} & \coefse{0.3334}{0.041} & \coefse{0.1636}{0.050} & 0.0000 & 0.0000 & 0.0010 \\
3  & \texttt{HP gap} ($\tilde y_{t-2}$)      & 82 & \coefse{0.9389}{0.020} & \coefse{0.3375}{0.046} & \coefse{0.0871}{0.037} & 0.0000 & 0.0000 & 0.0190 \\
\addlinespace[1mm]

\multicolumn{9}{l}{\textit{Hiato: BCB}}\\
4  & \texttt{BCB gap} ($\tilde y_{t}$)       & 82 & \coefse{0.9281}{0.021} & \coefse{0.3221}{0.037} & \coefse{0.1599}{0.053} & 0.0000 & 0.0000 & 0.0030 \\
5  & \texttt{BCB gap} ($\tilde y_{t-1}$)     & 82 & \coefse{0.9213}{0.023} & \coefse{0.3164}{0.041} & \coefse{0.1687}{0.055} & 0.0000 & 0.0000 & 0.0020 \\
6  & \texttt{BCB gap} ($\tilde y_{t-2}$)     & 82 & \coefse{0.9233}{0.025} & \coefse{0.3275}{0.048} & \coefse{0.1150}{0.054} & 0.0000 & 0.0000 & 0.0330 \\
\addlinespace[1mm]

\multicolumn{9}{l}{\textit{Hiato: Filtro Hamilton}}\\
7  & \texttt{Hamilton gap} ($\tilde y_{t}$)  & 82 & \coefse{0.9378}{0.018} & \coefse{0.3545}{0.028} & \coefse{0.0618}{0.015} & 0.0000 & 0.0000 & 0.0000 \\
8  & \texttt{Hamilton gap} ($\tilde y_{t-1}$)& 82 & \coefse{0.9351}{0.019} & \coefse{0.3575}{0.031} & \coefse{0.0617}{0.018} & 0.0000 & 0.0000 & 0.0010 \\
9  & \texttt{Hamilton gap} ($\tilde y_{t-2}$)& 82 & \coefse{0.9358}{0.021} & \coefse{0.3548}{0.037} & \coefse{0.0385}{0.023} & 0.0000 & 0.0000 & 0.0900 \\
\addlinespace[1mm]

\multicolumn{9}{l}{\textit{Hiato: IFI}}\\
10 & \texttt{IFI gap} ($\tilde y_{t}$)       & 82 & \coefse{0.9281}{0.019} & \coefse{0.3433}{0.034} & \coefse{0.1617}{0.052} & 0.0000 & 0.0000 & 0.0020 \\
11 & \texttt{IFI gap} ($\tilde y_{t-1}$)     & 82 & \coefse{0.9217}{0.019} & \coefse{0.3362}{0.036} & \coefse{0.1776}{0.048} & 0.0000 & 0.0000 & 0.0000 \\
12 & \texttt{IFI gap} ($\tilde y_{t-2}$)     & 82 & \coefse{0.9236}{0.021} & \coefse{0.3396}{0.042} & \coefse{0.1228}{0.042} & 0.0000 & 0.0000 & 0.0030 \\
\bottomrule
\end{tabularx}

\vspace{1.5mm}
\begin{minipage}{\textwidth}
\scriptsize
\textit{Notas}: Estimações por MQO com erros-padrão robustos a heterocedasticidade e autocorrelação (HAC, 4 defasagens, sem correção de pequena amostra). A variável dependente é a Selic média trimestral. Reportam-se coeficientes e, entre parênteses, erros-padrão HAC.

\vspace{5mm}
\textit{\underline{Dicionário das variáveis/colunas}}:
\begin{itemize}
  \item $\#$: Identificador da especificação estimada.
  \item \textit{Spec}: Medida do hiato do produto utilizada na regra de Taylor e seu horizonte temporal.
  \item $N$: Número de observações trimestrais utilizadas na estimação.
  \item $\rho$: Coeficiente de persistência da taxa de juros (coeficiente da Selic defasada $i_{t-1}$).
  \item $\phi_{\pi}$: Resposta de curto prazo da taxa de juros à inflação corrente (IPCA acumulado em 12 meses no fim do trimestre).
  \item $\phi_{y}$: Resposta de curto prazo da taxa de juros ao hiato do produto.
  \item $p(\rho)$: Valor-p associado ao teste de significância do coeficiente de persistência da taxa de juros.
  \item $p(\phi_{\pi})$: Valor-p associado ao teste de significância da resposta à inflação.
  \item $p(\phi_{y})$: Valor-p associado ao teste de significância da resposta ao hiato do produto.
\end{itemize}

\end{minipage}
\end{table}
% ----------------------------------------------
% Fim da tabela MQO Backward-Looking

\paragraph{Robustez às Medidas de Hiato}
A resposta à atividade econômica ($\hat{\phi}_{y}$) exibe heterogeneidade marcante dependendo da métrica de ciclo utilizada, evidenciando a sensibilidade da regra de Taylor à mensuração do hiato.

Nas especificações que utilizam os hiatos do BCB, IFI e Filtro HP (Modelos 1-6 e 10-12), a resposta é positiva e economicamente relevante, com coeficientes situados majoritariamente entre $0{,}09$ e $0{,}18$, e significativos a 1\% (com exceção do modelo com defasagem t-2 do HP). Isso sugere que a autoridade monetária reagiu à ociosidade calculada por métricas convencionais.

Em contraste, as especificações com o Filtro de Hamilton (Modelos 7-9) apresentam coeficientes substancialmente menores ($0{,}03$ a $0{,}06$) e perda de significância estatística nas defasagens mais longas. Dado que o filtro de Hamilton tende a produzir ciclos mais voláteis e curtos, esse resultado sugere que o BCB "filtra" \space ruídos de alta frequência na atividade, reagindo apenas a movimentos mais persistentes do ciclo econômico.

\paragraph{Especificações e Defasagens}
A comparação entre as estruturas de defasagem revela que a informação contemporânea ($t$) e a defasada em um período ($t-1$) dominam a função de reação. As especificações com hiato em $t-2$ (Modelos 3, 6, 9 e 12) apresentam sistematicamente menor poder explicativo para o coeficiente $\phi_y$ e, no caso do hiato IFI e BCB, uma ligeira redução na magnitude da resposta à inflação. Isso corrobora a hipótese de que o Comitê utiliza o conjunto de informações mais recente disponível (nowcasting) para calibrar o ajuste fino da Selic.

\paragraph{Resumo dos Resultados}
Em síntese, as estimativas MQO apontam para um Banco Central que (i) opera com alto grau de inércia, (ii) reage consistentemente à inflação no curto prazo e (iii) considera o hiato do produto, desde que mensurado por métricas que suavizem a volatilidade de curto prazo (como HP ou estimativas oficiais). A magnitude dos coeficientes, contudo, levanta dúvidas sobre a determinância estrutural (longo prazo), tema que será explorado via GMM nos tópicos a seguir.

% ----------------------------------------
\subsubsection{\textbf{Forward-Looking (GMM)}}
\label{gmm_forward_estimacao}

A Tabela~\ref{tab:gmm_forward_resultados} apresenta os resultados da estimação da regra de Taylor \textit{forward-looking} via Método dos Momentos Generalizados (GMM). Diferentemente da análise retrospectiva (MQO), esta etapa concentra-se em uma única especificação de referência. Essa restrição metodológica deve-se à indisponibilidade de séries históricas longas e consistentes para as expectativas de mercado do hiato do produto, o que impôs o uso do hiato HP referente a $t+1$ como \textit{proxy} para a atividade esperada (vide \ref{subsec:indicadores_economicos} e \ref{subsubsec:gmm_regras_forward_hibrida}).

\paragraph{Dinâmica de Curto Prazo e Expectativas}
A introdução de componentes prospectivos altera a magnitude da resposta da autoridade monetária em comparação ao modelo retrospectivo.

Primeiro, a resposta de curto prazo à inflação esperada ($\hat{\phi}_{\pi}$) é estimada em $0{,}548$, substancialmente superior aos coeficientes obtidos via MQO ($\approx 0{,}35$). Isso sugere que o Banco Central reage de forma mais agressiva às projeções de desvios da meta do que à inflação passada, característica condizente com um regime de metas focado na ancoragem de expectativas.

% ----------------------------------------------
% TABELA: Regra Forward-Looking (GMM)
% ----------------------------------------------
\begin{table}[H]
\centering
\caption{Regra de Taylor Forward-Looking estimada por GMM (2004--2024).}
\label{tab:gmm_forward_resultados}

\small
\setlength{\tabcolsep}{8pt}
\renewcommand{\arraystretch}{1.2}

\begin{tabular}{l c c c c}
\toprule
\textbf{Variável} & \textbf{Coeficiente} & \textbf{Erro-Padrão} & \textbf{Estatística Z} & \textbf{Valor-p} \\
\midrule
Constante ($c$)             & $-1{,}4849$ & $0{,}271$ & $-5{,}47$ & $0{,}000$ \\
Inércia ($\rho$)            & $0{,}8648$  & $0{,}021$ & $42{,}14$ & $0{,}000$ \\
Inflação Esperada ($\phi_{\pi}$) & $0{,}5483$  & $0{,}058$ & $9{,}41$  & $0{,}000$ \\
Hiato Esperado ($\phi_{y}$) & $0{,}2963$  & $0{,}085$ & $3{,}47$  & $0{,}001$ \\
\midrule
\multicolumn{5}{l}{\textit{Estatísticas de Diagnóstico:}} \\
Observações ($N$)           & 82          &           &           &           \\
Teste J de Hansen           & $10{,}26$   &           &           & $0{,}0165$ \\
\bottomrule
\end{tabular}

\vspace{2mm}
\begin{minipage}{\textwidth}
\scriptsize
\textit{Notas}: Estimação por GMM (IV-GMM) com matriz de ponderação robusta (HAC). \textbf{Variável dependente}: Taxa Selic média trimestral. Conforme Subsubseção~\ref{subsubsec:gmm_regras_forward_hibrida}: \textbf{Instrumentos}: Defasagens da taxa de juros, inflação, hiato e variáveis cambiais; O \textbf{Teste J de Hansen} avalia a validade das restrições de sobreidentificação.
\end{minipage}
\end{table}
% ----------------------------------------------
% Fim da tabela GMM Forward-Looking

Segundo, a reação ao hiato do produto esperado ($\hat{\phi}_{y}$) é de $0{,}296$, também superior às estimativas \textit{backward-looking}. Esse resultado indica que a política monetária atua de forma preventiva (preemptiva) em relação ao ciclo econômico, elevando os juros antecipadamente quando se projeta um aquecimento da atividade, antes mesmo que ele se materialize plenamente nos índices de preços correntes.

\paragraph{Persistência da Taxa de Juros}
O grau de inércia, medido por $\hat{\rho} \approx 0{,}86$, permanece elevado e estatisticamente significativo a 1\%. Contudo, nota-se uma redução em relação aos patamares observados no MQO ($>0{,}92$). Economicamente, isso pode sinalizar que parte da "inércia aparente" observada nas regras de bolso é, na verdade, uma resposta sistemática a expectativas que se movem lentamente. Ainda assim, a magnitude do coeficiente confirma que o alisamento da taxa de juros é uma característica intrínseca da regra, independentemente do horizonte temporal das variáveis explicativas.

\paragraph{Considerações sobre os resultados}
Em resumo, apesar da significância individual dos coeficientes, a validade estrutural desta especificação requer cautela. O teste de sobreidentificação de Hansen rejeita a hipótese nula de validade dos instrumentos ao nível de 5\% ($J=10{,}26$; valor-p $= 0{,}0165$).

Essa rejeição estatística sugere que a dinâmica prospectiva pura, tal como especificada, pode não capturar adequadamente todas as correlações entre os instrumentos e o termo de erro, ou que a forma funcional omite mecanismos relevantes de ajustamento. Do ponto de vista econômico, isso implica que inferências sobre o cumprimento do Princípio de Taylor baseadas exclusivamente nesta equação (calculando o longo prazo como $\phi/(1-\rho)$) seriam frágeis, pois o modelo não passa no crivo de validade das condições de momento. Esse diagnóstico reforça a necessidade de se investigar a regra híbrida, apresentada na sequência, que busca corrigir potenciais erros de especificação ao modelar conjuntamente a formação de expectativas e a estrutura de defasagens da política monetária.

% ----------------------------------------
\subsubsection{\textbf{Híbrida (GMM)}}
\label{gmm_hibrida_estimacao}

A Tabela~\ref{tab:gmm_hibrido_resultados} apresenta as estimações da regra de Taylor híbrida via GMM, combinando componentes prospectivos (expectativas) e retrospectivos (inércia). Esta especificação oferece a caracterização mais completa da função de reação, permitindo isolar o efeito do alisamento da taxa de juros da resposta estrutural às variáveis macroeconômicas, consoante o explanado em \ref{par:taylor_hibrida}, \ref{subsubsec:gmm_regras_forward_hibrida} e \ref{subsubsec:regra_hibrida_suavizacao_gmm}.

\paragraph{Persistência e Raiz Unitária}
De pronto, o resultado mais marcante destas estimações é a magnitude do coeficiente de persistência $\hat{\rho}$. Em todas as especificações, o parâmetro situa-se próximo ou ligeiramente acima da unidade (entre $0{,}96$ e $1{,}04$). Estatisticamente, valores de $\rho \geq 1$ sugerem que a taxa Selic exibe comportamento de raiz unitária ou quase-raiz unitária. Economicamente, isso implica um grau extremo de gradualismo: os choques de política monetária são altamente persistentes e a reversão à média é lenta.

Essa evidência impõe severas restrições à interpretação dos coeficientes de longo prazo. Como o denominador $(1-\rho)$ tende a zero, os multiplicadores implícitos tornam-se assintoticamente indefinidos. Assim sendo, a análise da conduta do Banco Central deste trabalho limita-se estritamente à dinâmica de curto prazo representada pelos coeficientes diretos $\phi_\pi$ e $\phi_y$.

\paragraph{Resposta à Inflação Esperada}
Controlando pela inércia, a resposta de curto prazo à inflação esperada ($\hat{\phi}_{\pi}$) mostra-se robusta e positiva, variando no intervalo $[0{,}73; 0{,}92]$ com elevada significância estatística ($p < 0{,}01$).
Comparativamente aos modelos anteriores, a regra híbrida revela uma reação contemporânea mais vigorosa, o que significa que, uma vez descontado o componente inercial, o Banco Central ajusta a taxa de juros nominal de forma quase proporcional (mas ainda inferior a 1 no curto prazo) às expectativas de inflação, validando a importância do canal prospectivo no regime de metas.

\paragraph{Resposta ao Hiato e Heterogeneidade}
A reação à atividade econômica ($\hat{\phi}_{y}$) apresenta resultados mistos e sensíveis à especificação, refletindo a dificuldade de se identificar uma resposta sistemática ao ciclo em tempo real.

% ----------------------------------------------
% TABELA: Regras de Taylor híbridas (GMM)
% ----------------------------------------------
\begin{table}[H]
\centering
\caption{Regras de Taylor híbridas estimadas por GMM (one-step): coeficientes de curto prazo e persistência.}
\label{tab:gmm_hibrido_resultados}

\scriptsize
\setlength{\tabcolsep}{3.5pt}
\renewcommand{\arraystretch}{1.20}

% coluna X alinhada à esquerda (como na MQO)
\newcolumntype{Y}{>{\raggedright\arraybackslash}X}

\begin{tabularx}{\textwidth}{r Y r r r r r r r r}
\toprule
\# & Spec & N & \#Z & $\rho$ & $\phi_{\pi}$ & $\phi_{y}$ &
$p(\rho)$ & $p(\phi_{\pi})$ & $p(\phi_{y})$ \\
\midrule

\multicolumn{10}{l}{\textit{Hiato: BCB}}\\
1 & \texttt{BCB gap} ($\tilde y_{t-1}$) & 82 & 5 & 1.0031 & 0.8367 & -0.1222 & 0.0000 & 0.0000 & 0.2576 \\
2 & \texttt{BCB gap} ($\tilde y_{t-2}$) & 82 & 5 & 1.0394 & 0.9166 & -0.2594 & 0.0000 & 0.0000 & 0.0218 \\
\addlinespace[1mm]

\multicolumn{10}{l}{\textit{Hiato: Filtro Hamilton}}\\
3 & \texttt{Hamilton gap} ($\tilde y_{t-1}$) & 82 & 5 & 0.9595 & 0.7359 & 0.0495 & 0.0000 & 0.0000 & 0.0604 \\
4 & \texttt{Hamilton gap} ($\tilde y_{t-2}$) & 82 & 5 & 0.9700 & 0.7643 & 0.0214 & 0.0000 & 0.0000 & 0.4076 \\
\addlinespace[1mm]

\multicolumn{10}{l}{\textit{Hiato: Filtro HP}}\\
5 & \texttt{HP gap} ($\tilde y_{t-1}$) & 82 & 5 & 0.9704 & 0.7341 & 0.0719 & 0.0000 & 0.0000 & 0.3990 \\
6 & \texttt{HP gap} ($\tilde y_{t-2}$) & 82 & 5 & 0.9817 & 0.7843 & -0.0074 & 0.0000 & 0.0000 & 0.9145 \\
\addlinespace[1mm]

\multicolumn{10}{l}{\textit{Hiato: IFI}}\\
7 & \texttt{IFI gap} ($\tilde y_{t-1}$) & 82 & 5 & 0.9861 & 0.7851 & -0.0274 & 0.0000 & 0.0000 & 0.7701 \\
8 & \texttt{IFI gap} ($\tilde y_{t-2}$) & 82 & 5 & 1.0139 & 0.8383 & -0.1218 & 0.0000 & 0.0000 & 0.1566 \\
\bottomrule
\end{tabularx}

\vspace{1.5mm}
\begin{minipage}{\textwidth}
\scriptsize
\textit{Notas}: Estimações via GMM one-step com erros-padrão HAC. $\rho$ denota a persistência da taxa de juros. $\phi_{\pi}$ e $\phi_{y}$ representam respostas de curto prazo à inflação esperada e ao hiato do produto. Valores de $\rho \approx 1$ indicam alta persistência, impossibilitando o cálculo direto de multiplicadores de longo prazo robustos.

\vspace{5mm}
\textit{\underline{Dicionário das variáveis/colunas}}:
\begin{itemize}\setlength{\itemsep}{1pt}
\item \#: Identificador da especificação estimada.
\item \textit{Spec}: Medida do hiato utilizada e seu horizonte temporal ($t-1$ ou $t-2$).
\item \#Z: Número de instrumentos utilizados na estimação GMM.
\item $\rho$: Coeficiente de persistência da taxa de juros.
\item $\phi_{\pi}$: Resposta de curto prazo da taxa de juros à inflação esperada.
\item $\phi_{y}$: Resposta de curto prazo da taxa de juros ao hiato do produto.
\item $p(\cdot)$: Valor-p associado ao teste de significância do coeficiente respectivo.
\end{itemize}

\end{minipage}
\end{table}
% ----------------------------------------------
% Fim da tabela GMM Híbrida

Nas especificações com hiatos do BCB e IFI (Modelos 1, 2, 7 e 8), os coeficientes são estatisticamente nulos ou, no caso do Modelo 2, negativos e significativos. Um coeficiente negativo ($\phi_y < 0$) é contra-intuitivo pela teoria padrão, mas pode ocorrer em períodos de estagflação (choques de oferta), onde o Banco Central eleva os juros para combater a inflação mesmo diante de uma desaceleração do produto.

Apenas nas especificações com o Filtro de Hamilton (Modelo 3), observa-se um coeficiente positivo e marginalmente significativo ($0{,}05$, $p=0{,}06$), sugerindo uma resposta fraca, mas alinhada à teoria, quando o ciclo é mensurado por métodos que isolam componentes cíclicos de maior frequência.

\paragraph{Síntese da Evidência Híbrida}
Em suma, a regra híbrida confirma que a política monetária brasileira é caracterizada por (i) dominância da estabilização inflacionária sobre a estabilização do produto e (ii) inércia extremada. A combinação de $\rho \approx 1$ com $\phi_\pi < 1$ no curto prazo sugere que, neste caso, o cumprimento do Princípio de Taylor é um fenômeno dependente da acumulação de ajustes ao longo do tempo, e não uma resposta imediata. Essa dinâmica levanta a questão central da próxima seção: dados esses parâmetros estimados, o equilíbrio macroeconômico resultante é determinado e único, ou a economia operou em regiões de instabilidade potencial? A avaliação da condição de Bullard-Mitra a seguir endereça essa questão, oferecendo um critério de estabilidade robusto mesmo na presença de alta persistência.

% PARTE 2: DETERMINÂNCIA DO EQUILÍBRIO
% -----------------------------------------------------------
\subsection{Determinância e Condição de Bullard--Mitra}
\label{subsec:determinancia_bullard_mitra}

A avaliação da consistência da política monetária com a estabilidade macroeconômica, objetivo central deste estudo, requer o confronto dos parâmetros estimados com as condições de unicidade do equilíbrio no modelo Novo-Keynesiano. Tradicionalmente, conforme explanado em \ref{subsec:especificacao_econometrica}, essa verificação baseia-se na resposta de longo prazo da autoridade monetária à inflação ($\phi_{\pi}^{LP}$), calculada algebricamente a partir dos coeficientes da regra de Taylor com suavização:
\begin{equation}
\phi_{\pi}^{LP} = \frac{\phi_{\pi}}{1 - \rho}
\label{eq:long_run_multiplier}
\end{equation}
Conforme discutido neste artigo em \ref{subsec:regras_politica_monetaria}, \ref{subsec:determinancia_estabilidade}, \ref{subsec:evidencias_internacionais}, \ref{subsec:evidencias_brasil}, \ref{subsubsec:principio_taylor}, \ref{sec:metodologia} e \ref{subsec:determinancia_bullard_mitra}, a condição canônica para a determinância exige que $\phi_{\pi}^{LP} > 1$, assegurando que a taxa de juros real se eleve em resposta a choques inflacionários (\cite{taylor1993}, \cite{clarida2000monetary}).

Entretanto, os resultados empíricos apresentados na subseção anterior impõem uma restrição severa à aplicação direta dessa métrica. As estimações da regra híbrida (Tabela~\ref{tab:gmm_hibrido_resultados}), corroboradas por testes de robustez via Mínimos Quadrados Generalizados (GLS) com erros autorregressivos, indicam que o parâmetro de persistência situa-se sistematicamente na vizinhança da unidade ($\hat{\rho} \approx 1$), exibindo comportamento de raiz unitária ou quase-raiz unitária. Matematicamente, à medida que $\rho \to 1$, o denominador da Equação \ref{eq:long_run_multiplier} tende a zero, fazendo com que o multiplicador calculado assuma valores explosivos e economicamente não-interpretáveis.

Esse fenômeno estatístico sugere que o Banco Central do Brasil operou sob um regime que a literatura denomina de \textbf{super-inercial}. Neste contexto, a análise de determinância não pode se restringir ao cálculo estático do multiplicador de longo prazo, devendo recorrer às condições dinâmicas estabelecidas por \cite{bullard2002learning} e \cite{rotemberg1997optimization} para regras com elevada suavização.

% -----------------------------------------------------------
\subsubsection{Estabilidade em Regimes Super-Inerciais}

Segundo \citeonline[p.~255--256]{woodford2003interest}, em modelos Novo-Keynesianos com regras de juros inerciais, a condição de determinância deve ser interpretada em termos da resposta cumulativa da política monetária a um choque inflacionário permanente. O autor demonstra (Proposição 4.4) que, quando a regra exibe persistência elevada ou ``super-inércia'' ($\rho \geq 1$), a condição necessária e suficiente para a unicidade do equilíbrio é satisfeita mesmo com coeficientes de reação contemporânea moderados ($\phi_{\pi} < 1$), uma vez que a promessa de manter as taxas de juros elevadas por um longo horizonte garante a elevação necessária da taxa real \textit{ex-ante}.

A lógica econômica subjacente é intertemporal: em uma regra super-inercial, um aumento na inflação hoje eleva a taxa de juros nominal corrente em apenas $\phi_{\pi}$ (estimado neste trabalho entre $0{,}73$ e $0{,}92$, portanto $< 1$). No entanto, devido ao coeficiente $\rho \approx 1$, esse aperto inicial não é revertido nos períodos subsequentes; ele é carregado e amplificado pela dinâmica da regra. Agentes racionais, antecipando que a taxa de juros nominal permanecerá elevada por um horizonte longo, revisam suas expectativas de inflação para baixo imediatamente, elevando a taxa de juros real \textit{ex-ante} acumulada.

Nesse arcabouço, a evidência econométrica de que $\hat{\phi}_{\pi} \in (0, 1)$ combinada com $\hat{\rho} \approx 1$ é compatível, em tese, com a determinância do equilíbrio. O Banco Central do Brasil teria logrado ancorar as expectativas não por meio de choques abruptos na taxa Selic, mas através de um compromisso implícito de que os apertos monetários, uma vez iniciados, seriam duradouros (\cite{coibion2012target}).

% -----------------------------------------------------------
\subsubsection{Limitações da Análise Algébrica}

Embora a teoria sugira que a super-inércia possa compensar uma resposta contemporânea moderada, a verificação empírica dessa hipótese não é trivial. A desigualdade de Bullard-Mitra apresentada em \ref{subsec:condicoes_determinancia} depende implicitamente de multiplicadores finitos de longo prazo. Diante da singularidade matemática gerada pela raiz unitária estimada, a simples substituição dos parâmetros na inequação analítica torna-se inconclusiva tanto para a unicidade do Equilíbrio de Expectativas Racionais (REE) quanto para a estabilidade sob aprendizagem (\textit{E-stability}).

Essa distinção é crucial: ainda que um equilíbrio exista (determinância), a presença de inércia extrema pode dificultar a convergência das expectativas dos agentes para esse equilíbrio caso a regra não seja suficientemente clara (\cite{bullard2002learning}). Para superar essa limitação e responder definitivamente ao problema de pesquisa, faz-se necessário avaliar as \textbf{implicações dinâmicas} dos parâmetros estimados. 

A subseção a seguir utiliza um modelo DSGE Novo-Keynesiano calibrado para a economia brasileira e simulado via Dynare. Confome explicado em \ref{subsec:determinancia_estabilidade}, o objetivo é submeter o sistema de equações estruturais (NKPC, IS Dinâmica e Regras Estimadas) ao critério de estabilidade de \citeonline{blanchard1980solution}, verificando computacionalmente se a combinação de parâmetros $(\hat{\rho} \approx 1, \hat{\phi}_{\pi} < 1)$ obtida empiricamente é suficiente para gerar trajetórias convergentes, validando o cumprimento do Princípio de Taylor por vias dinâmicas.

% -----------------------------------------------------------
\subsection{Implicações Dinâmicas e Simulações (Dynare)}
\label{subsec:simulacoes_dynare}

\textcolor{magenta}{<-> APERFEIÇOAR TEXTO + TENTAR FAZER LOOP NO DYNARE COM VÁRIOS RESULTADOS DAS TABELAS ANTERIORES + GRÁFICOS EM PNG E NÃO PDF <->}

Diante da impossibilidade de validar a determinância apenas por critérios algébricos estáticos dada a super-inércia estimada, esta subseção explora o comportamento dinâmico do modelo Novo-Keynesiano calibrado para a economia brasileira. O objetivo é verificar computacionalmente se a combinação de parâmetros $(\hat{\rho} \approx 0{,}98, \hat{\phi}_{\pi} \approx 0{,}85)$ obtida empiricamente é suficiente para satisfazer as condições de estabilidade de \citeonline{blanchard1980solution}.

Para este exercício, utilizou-se um modelo DSGE canônico de pequena escala, calibrado com parâmetros estruturais padrão para a economia brasileira ($\beta=0{,}99$, $\sigma=1{,}5$, $\kappa=0{,}08$), e alimentado com a regra de Taylor híbrida estimada na Seção anterior.

\paragraph{Resultados de Estabilidade}
As simulações computacionais confirmam que o modelo satisfaz as condições de Blanchard-Kahn. O sistema converge para um equilíbrio único e estável, afastando a hipótese de indeterminação ou trajetórias explosivas. Esse resultado corrobora a tese teórica de \citeonline{woodford2003interest}: em regimes super-inerciais, a promessa de persistência da taxa de juros compensa uma resposta contemporânea moderada à inflação.

\paragraph{Funções de Resposta ao Impulso (IRFs)}
A análise das Funções de Resposta ao Impulso (Figura~\ref{fig:irf_dynare}) ilustra o mecanismo de transmissão. Diante de um choque inflacionário de custo ($e_{\pi}$), a taxa de juros nominal (linha inferior) eleva-se gradualmente, mas de forma persistente. Embora o aumento inicial seja menor que o choque inflacionário (dado $\phi_{\pi} < 1$), a trajetória da taxa de juros permanece acima do estado estacionário por um longo horizonte (mais de 10 trimestres). Essa inércia gera uma elevação acumulada da taxa de juros real \textit{ex-ante}, suficiente para contrair o hiato do produto (linha central) e forçar o retorno da inflação à meta (linha superior).

\begin{figure}[H]
    \centering
    \caption{Funções de Resposta ao Impulso: Choque de Inflação (Oferta)}
    \label{fig:irf_epi}
    % Caminho relativo para a figura do choque de inflação
    \includegraphics[width=0.85\textwidth]{../scripts/modelo_nk/graphs/modelo_nk_IRF_e_pi.pdf}
    \footnotesize
    \vspace{0.2cm}
    \textit{Fonte: Elaboração própria a partir de simulações no Dynare (2025).}
\end{figure}

\begin{figure}[H]
    \centering
    \caption{Funções de Resposta ao Impulso: Choque de Política Monetária}
    \label{fig:irf_ei}
    % Caminho relativo para a figura do choque de juros
    \includegraphics[width=0.85\textwidth]{../scripts/modelo_nk/graphs/modelo_nk_IRF_e_i.pdf}
    \footnotesize
    \vspace{0.2cm}
    \textit{Fonte: Elaboração própria a partir de simulações no Dynare (2025).}
\end{figure}


Portanto, a evidência econométrica e a simulação dinâmica convergem para a mesma conclusão: a política monetária brasileira no período analisado operou em uma região de determinância, sustentada fundamentalmente pelo canal da inércia e pela gestão de expectativas, mitigando a necessidade de choques nominais abruptos.

% ***********************************************************

% ===========================================================
\section{Considerações Finais}
% ===========================================================
\label{sec:consideracoes_finais}

\begin{itemize}
    \item Responder ao problema de pesquisa, sintetizando em que medida a política monetária brasileira satisfez o Princípio de Taylor e a condição de determinância.
    \item Avaliar a compatibilidade do regime de metas com a determinância ao longo dos diferentes regimes de política monetária.
    \item Destacar a contribuição teórica e empírica do artigo para o debate sobre estabilidade nominal, credibilidade e dominância fiscal.
    \item Sugerir extensões (DSGE estimado, interação fiscal-monetária, comparações internacionais).
\end{itemize}

Acrescentar:
- Limitações: inserir dólar, preços de produtos agrícolas, outros choques\dots
- Refazer os testes econométricos por períodos dos mandatos dos presidentes do BC.

\clearpage
\pagenumbering{Roman}
\bibliographystyle{abntex2-alf}
\bibliography{library}

\end{document}